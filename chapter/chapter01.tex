\section{绪论与信号基础}

% --- 模块一:傅里叶变换 ---
\begin{kbox}{信号分析基础:傅里叶变换}
    \begin{enumerate}
        \item \textbf{定义} \pptpage{6}
        \begin{itemize}
            \item 正变换:$F(\omega)=\int_{-\infty}^{\infty} f(t) e^{-j \omega t} d t$
            \item 反变换:$f(t)=\frac{1}{2 \pi} \int_{-\infty}^{\infty} F(\omega) e^{j \omega t} d \omega$
        \end{itemize}

        \item \textbf{常用运算特性} \pptpage{7-8}
        \begin{itemize}
            \item \textbf{共轭对称性}(补充):
            实信号 $f(t)$ 的频谱满足 $F(\omega) = F^*(-\omega)$。
            即:\textbf{幅度谱为偶函数,相位谱为奇函数}。
            \item \textbf{标度换算}:
            $f(a t) \leftrightarrow \frac{1}{|a|} F\left(\frac{\omega}{a}\right)$
            \item \textbf{时移特性}:
            $f(t-t_{0}) \leftrightarrow e^{-j \omega t_{0}} F(\omega)$
            \item \textbf{频移特性}:
            $e^{j \omega_{0} t} f(t) \leftrightarrow F(\omega-\omega_{0})$
            \item \textbf{调制特性}:
            $f(t) \cos \omega_{0} t \leftrightarrow \frac{1}{2}[F(\omega+\omega_{0})+F(\omega-\omega_{0})]$
            \item \textbf{时域卷积}:
            $f_{1}(t) * f_{2}(t) \leftrightarrow F_{1}(\omega) \cdot F_{2}(\omega)$
            \item \textbf{微分特性}:
            $\frac{d^{n}}{d t^{n}} f(t) \leftrightarrow(j \omega)^{n} F(\omega)$
        \end{itemize}

        \item \textbf{常用信号对与推论} \pptpage{10-11}
        \begin{itemize}
            \item \textbf{矩形脉冲}(重要):
            宽度为 $\tau$ 的门函数 $g_\tau(t) \leftrightarrow \tau \operatorname{Sa}\left(\frac{\omega \tau}{2}\right)$
            \begin{itemize}
                \item[$\circ$] 注:$Sa(x)=\frac{\sin x}{x}$, $sinc(x)=\frac{\sin \pi x}{\pi x}$
                \item[$\circ$] 关系:$Sa(\frac{\omega\tau}{2}) = sinc(f\tau)$
                \item[$\circ$] \textbf{对偶特性(笔记补充)}:
                $Sa(\tau t) \leftrightarrow \frac{\pi}{\tau} g_{2\tau}(\omega)$
                \\ (含义:时域 Sa 函数对应频域宽度为 $2\tau$ 的矩形谱)
            \end{itemize}
            \item \textbf{三角脉冲}:
            等腰三角脉冲可分解为两个矩形脉冲的卷积。
            
            % --- 开始插入 TikZ 绘图 ---
            \begin{center}
            \begin{tikzpicture}[scale=0.55, >=latex, font=\scriptsize]
                % 定义样式:坐标轴和波形颜色
                \tikzstyle{axis}=[->, thin, gray]
                \tikzstyle{signal}=[thick, mainblue] 

                % --- 图1:三角波 ---
                % 坐标轴
                \draw[axis] (-1.8,0) -- (1.8,0);
                \draw[axis] (0,-0.2) -- (0,1.8);
                % 波形:从 -tau 到 tau,高度为 A^2 tau
                \draw[signal] (-1.2, 0) -- (0, 1.4) node[right, black]{$A^2\tau$} -- (1.2, 0);
                % 刻度标签
                \node[below] at (-1.2, 0) {$-\tau$};
                \node[below] at (1.2, 0) {$\tau$};

                % --- 等号 ---
                \node at (2.2, 0.7) {\large $=$};

                % --- 图2:矩形波1 (利用 scope 平移坐标系) ---
                \begin{scope}[shift={(4.5,0)}]
                    \draw[axis] (-1.2,0) -- (1.2,0);
                    \draw[axis] (0,-0.2) -- (0,1.8);
                    % 波形:从 -tau/2 到 tau/2,高度 A
                    \draw[signal] (-0.6, 0) -- (-0.6, 1) -- (0.6, 1) node[midway, above, black, xshift=7pt]{$A$} -- (0.6, 0);
                    % 刻度标签
                    \node[below] at (-0.6, 0) {$-\frac{\tau}{2}$};
                    \node[below] at (0.6, 0) {$\frac{\tau}{2}$};
                \end{scope}

                % --- 卷积符号 ---
                \node at (6.2, 0.7) {\large $*$};

                % --- 图3:矩形波2 (再次平移) ---
                \begin{scope}[shift={(8,0)}]
                    \draw[axis] (-1.2,0) -- (1.2,0);
                    \draw[axis] (0,-0.2) -- (0,1.8);
                    % 波形
                    \draw[signal] (-0.6, 0) -- (-0.6, 1) -- (0.6, 1) node[midway, above, black, xshift=7pt]{$A$} -- (0.6, 0);
                    % 刻度标签
                    \node[below] at (-0.6, 0) {$-\frac{\tau}{2}$};
                    \node[below] at (0.6, 0) {$\frac{\tau}{2}$};
                \end{scope}
            \end{tikzpicture}
            \end{center}
            % --- 结束插入 ---

            \item \textbf{双边指数信号}:
            $f(t) = e^{-\alpha t}u(t) - e^{\alpha t}u(-t) \leftrightarrow \frac{-2j\omega}{\alpha^2+\omega^2} \quad (\alpha>0)$
            \item \textbf{符号函数}:
            由上式令 $\alpha \to 0$ 可得:$\operatorname{sgn}(t) \leftrightarrow \frac{2}{j\omega}$
            \item \textbf{单位冲激串}:
            $\delta_{T}(t) \leftrightarrow \omega_{0} \sum_{n=-\infty}^{\infty} \delta(\omega-n \omega_{0})$
        \end{itemize}
    \end{enumerate}
\end{kbox}

% --- 新增模块:常用变换速查表 ---
\begin{kbox}{速查表:常用函数傅里叶变换对}
    \begin{center}
    \renewcommand{\arraystretch}{1.5}
    \begin{tabular}{c|c}
        \hline
        \textbf{时域信号} $f(t)$ & \textbf{频域频谱} $F(\omega)$ \\
        \hline
        $\delta(t)$ & $1$ \\
        $1$ & $2\pi \delta(\omega)$ \\
        $u(t)$ & $\pi \delta(\omega) + \frac{1}{j\omega}$ \\
        $e^{-\alpha t}u(t) \ (\alpha > 0)$ & $\frac{1}{\alpha + j\omega}$ \\
        $e^{-\alpha |t|} \ (\alpha > 0)$ & $\frac{2\alpha}{\alpha^2 + \omega^2}$ \\
        $e^{j \omega_0 t}$ & $2\pi \delta(\omega - \omega_0)$ \\
        $\cos(\omega_0 t)$ & $\pi [\delta(\omega - \omega_0) + \delta(\omega + \omega_0)]$ \\
        $\sin(\omega_0 t)$ & $j\pi [\delta(\omega + \omega_0) - \delta(\omega - \omega_0)]$ \\
        $\operatorname{sgn}(t)$ & $\frac{2}{j\omega}$ \\
        \hline
    \end{tabular}
    \end{center}
\end{kbox}

% --- 新增模块:w与f域性质对比 (本次添加) ---
\begin{kbox}{深度对比:$\omega$ 与 $f$ 域性质差异}
    \textbf{核心差异速记}:
    \begin{itemize}
        \item \textbf{积分相关}(逆变换、卷积、能量):$\omega$ 域通常需乘 $\frac{1}{2\pi}$ 修正。
        \item \textbf{微分相关}:$f$ 域通常需乘 $2\pi$。
    \end{itemize}
    
    \begin{center}
    \renewcommand{\arraystretch}{1.3}
    % 使用 resizebox 确保表格适应双栏宽度,防止溢出
    \resizebox{\linewidth}{!}{
    \begin{tabular}{c|c|c}
        \hline
        \textbf{性质} & \textbf{$\omega$ 域 (rad/s)} & \textbf{$f$ 域 (Hz)} \\
        \hline
        \textbf{逆变换} & $\frac{1}{2\pi} \int F(\omega)e^{j\omega t}d\omega$ & $\int F(f)e^{j2\pi ft}df$ \\
        \hline
        \textbf{频域卷积} & $\frac{1}{2\pi} [F_1 * F_2]$ & $F_1 * F_2$ \\
        {\scriptsize (对应时域相乘)} & {\scriptsize (需除系数)} & {\scriptsize (无系数,对称)} \\
        \hline
        \textbf{能量} & $\frac{1}{2\pi} \int |F(\omega)|^2 d\omega$ & $\int |F(f)|^2 df$ \\
        \hline
        \textbf{时域微分} & $(j\omega)^n F(\omega)$ & $(j2\pi f)^n F(f)$ \\
        \hline
        \textbf{直流 DC} & $2\pi\delta(\omega)$ & $\delta(f)$ \\
        \hline
    \end{tabular}
    }
    \end{center}
\end{kbox}
% --------------------------------

% --- 模块二:确定信号与系统 ---
\begin{kbox}{确定信号性质与线性系统}
    \begin{enumerate}
        \item \textbf{能量与功率} \pptpage{14}
        \begin{itemize}
            \item 能量:$E=\int_{-\infty}^{\infty} s^{2}(t) d t$
            \item 功率:$P=\lim _{T \rightarrow \infty} \frac{1}{T} \int_{-T / 2}^{T / 2} s^{2}(t) d t$
        \end{itemize}

        \item \textbf{带宽定义} \pptpage{15}
        \begin{equation*}
            B_{3}=\frac{\int_{-\infty}^{\infty} E(f) d f}{2 E(0)} \quad (\text{等效矩形带宽})
        \end{equation*}

        \item \textbf{线性系统响应} \pptpage{16-17}
        \begin{itemize}
            \item 频域:$Y(f)=H(f) X(f)$
            \item 时域:$y(t)=\int_{-\infty}^{\infty} h(t-\tau) x(\tau) d \tau$
        \end{itemize}
        
        \item \textbf{无失真传输条件} \pptpage{18} 
        \begin{itemize}
            \item 时域条件:$y(t)=k x(t-t_{d})$
            \item 频域条件:$H(\omega)=K e^{-j \omega t_{d}}$
            \item \textbf{群时延}:$\tau(\omega)=-\frac{d \varphi(\omega)}{d \omega} = \text{常数}$
        \end{itemize}
    \end{enumerate}
\end{kbox}

% --- 模块三:信息论基础 ---
\begin{kbox}{信息及其度量}
    \begin{enumerate}
        \item \textbf{离散消息信息量} \pptpage{50}
        \begin{equation*}
            I = -\ln P(x) \quad (\text{单位 nat})
        \end{equation*}
        \textbf{单位换算}:
        \begin{itemize}
            \item $I_{bit} = -\log_2 P(x) = \frac{1}{\ln 2} I_{nat}$
            \item $1 \text{ nat} \approx 1.44 \text{ bit}$
        \end{itemize}

        \item \textbf{离散信源熵 (平均信息量)} \pptpage{52}
        \begin{equation*}
            H=\sum_{i=1}^{M} P(x_{i}) \log _{2} \frac{1}{P(x_{i})} \quad (\text{bit/符号})
        \end{equation*}
        \textbf{最大熵}:当等概率出现时,$H_{\max }=\log _{2} M$。

        \item \textbf{总信息量} \pptpage{55}
        \begin{equation*}
            I_{\text{总}} = m \cdot H \quad (m \text{为符号总数})
        \end{equation*}
    \end{enumerate}
\end{kbox}

% --- 模块四:信道模型 ---
\begin{kbox}{信道与衰落}
    \begin{enumerate}
        \item \textbf{调制信道模型} \pptpage{59}
        \begin{equation*}
            e_{o}(t)=k(t) e_{i}(t)+n(t)
        \end{equation*}

        \item \textbf{多径效应与正交分解} \pptpage{70}
        接收信号 $R(t)$ 为多径叠加:
        \begin{align*}
            R(t) &= \sum_{i=1}^{n} \mu_i(t) \cos[\omega_c (t - \tau_i(t))] \\
                 & \qquad \text{\small (原始多径公式)} \\ % <--- 修改点:将说明文字移到下一行,并略微缩小
                 &= \sum_{i=1}^{n} \mu_i(t) \cos[\omega_c t + \varphi_i(t)] \\
                 & \quad {\color{red}\text{其中 } \varphi_i(t) = -\omega_c \tau_i(t) } \\ 
                 &= \sum_{i=1}^{n} \mu_i(t) \cos(\omega_c t) \cos(\varphi_i(t)) \\
                 &\quad - \sum_{i=1}^{n} \mu_i(t) \sin(\omega_c t) \sin(\varphi_i(t)) \\
                 &= X_c(t) \cos(\omega_c t) - X_s(t) \sin(\omega_c t) \\
                 &= V(t) \cos[\omega_c t + \varphi(t)]
        \end{align*}
        \textbf{参数含义}:
        \begin{itemize}
            \item $\mu_i(t)$:第 $i$ 条路径接收信号的\textbf{振幅}。
            \item $\tau_i(t)$:第 $i$ 条路径接收信号的\textbf{时延}。
            \item $\varphi_i(t)$:第 $i$ 条路径接收信号的\textbf{相位},且 $\varphi_i(t) = -\omega_c \tau_i(t)$。
            \item $X_c(t), X_s(t)$:同相与正交分量,大量路径叠加时视为高斯随机过程。
            \item $V(t)$:包络,服从瑞利分布;$\varphi(t)$:合成相位,服从均匀分布。
        \end{itemize}

        \item \textbf{衰落类型与相干带宽} \pptpage{74}
        \begin{itemize}
            \item 相干带宽:$\Delta f \approx 1 / \tau_{m}$ ($\tau_{m}$ 为最大多径时延差)
            \item \textbf{平坦衰落}:信号带宽 $B < \Delta f$
            \item \textbf{频率选择性衰落}:信号带宽 $B > \Delta f$
        \end{itemize}
    \end{enumerate}
\end{kbox}

% --- 模块五:性能指标 ---
\begin{kbox}{通信系统性能指标}
    \begin{enumerate}
        \item \textbf{模拟系统} \pptpage{87}
        \begin{equation*}
            \text{SNR}(\text{dB}) = 10 \lg (S_{i} / N_{i})
        \end{equation*}

        \item \textbf{补充定义:码元}
        \begin{itemize}
            \item 定义:承载信息的基本单位,是通信的“字符”或“信号”。
            \item 形式:固定时长的脉冲或波形(二进制2种,四进制4种)。
        \end{itemize}

        \item \textbf{数字系统:有效性 (速率)} \pptpage{88-89}
        \begin{itemize}
            \item \textbf{码元速率 ($R_B$)}:波特率(Baud),每秒传送的波形数(“货车”数量)。
            \item \textbf{信息速率 ($R_b$)}:比特率(bps),每秒传送的信息总量(“货物”总量)。
            \item \textbf{关系}:$R_{b} = R_{B} \log _{2} M$
                \begin{itemize}
                    \item 仅当各码元等概出现时,单码元信息量 $I = \log_2 M$。
                \end{itemize}
            \item \textbf{频带利用率}:$\eta = \frac{R_{B}}{B}$ (Baud/Hz) 或 $\frac{R_{b}}{B}$ (bps/Hz)。
        \end{itemize}

        \item \textbf{数字系统:可靠性 (误码/误信)} \pptpage{90-91}
        % --- 修改处:添加 raggedright 避免稀疏排版 ---
        {\raggedright
        \begin{itemize}
            \item \textbf{误码率 ($P_e$)}:错误码元数占比,$P_e = n_{eB} / n_B$。
                \begin{itemize}
                    \item 二进制平均误码率:$P_e = P(0)P(1|0) + P(1)P(0|1)$。
                \end{itemize}
            \item \textbf{误信率 ($P_b$)}:错误比特数占比,$P_b = n_{eb} / n_b$。
            \item \textbf{关系}:
            \begin{itemize}
                \item 二进制 ($M=2$):$P_{b} = P_{e}$
                \item 多进制 ($M>2$):$P_{b} < P_{e}$
            \end{itemize}
        \end{itemize}
        }
    \end{enumerate}
\end{kbox}