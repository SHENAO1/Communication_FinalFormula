% chapter/chapter03.tex

\section{信道}

\begin{kbox}{1. 信道模型}
    \textbf{调制信道 (Modulation Channel)}:
    \begin{itemize}
        \item 入出关系:
        \[ r(t) = k(t) \cdot s_i(t) + n(t) \]
        \item $k(t)$:乘性干扰 (反映信道时变特性)
        \item $n(t)$:加性噪声
    \end{itemize}
    
    \tcbline
    
    \textbf{编码信道 (Encoding Channel)} (二进制):
    \begin{itemize}
        \item 正确传输概率:$P(0/0) + P(1/0) = 1$
        \item \textbf{误码率}:
        \[ P_e = P(0)P(1/0) + P(1)P(0/1) \]
    \end{itemize}
\end{kbox}

\begin{kbox}{2. 恒参信道特性}
    \textbf{传输函数}:$H(\omega) = |H(\omega)| e^{j\varphi(\omega)}$
    
    \textbf{无失真传输条件}:
    \begin{enumerate}
        \item \textbf{幅频特性}:$|H(\omega)| = K$ (常数)
        \item \textbf{相频特性}:$\varphi(\omega) = \omega t_d$ (通过原点的直线)
        \item \textbf{群迟延}:$\tau(\omega) = \frac{d\varphi(\omega)}{d\omega} = t_d$ (常数)
    \end{enumerate}
    
    \textbf{时域响应}:
    \[ h(t) = K\delta(t - t_d) \Longrightarrow s_o(t) = K s(t - t_d) \]
\end{kbox}

\begin{kbox}{3. 随参信道与多径效应}
    \textbf{接收信号模型} ($n$ 条路径):
    \[ r(t) = \sum_{i=1}^n a_i(t) \cos \omega_c [t - \tau_i(t)] = V(t) \cos[\omega_c t + \varphi(t)] \]
    多径效应导致信号包络 $V(t)$ 和相位 $\varphi(t)$ 随机起伏。
    
    \tcbline
    
    \textbf{频率选择性衰落}:
    \begin{itemize}
        \item \textbf{相关带宽}:$\Delta f = 1/\tau_m$ ($\tau_m$ 为最大多径时延差)
        \item \textbf{工程经验公式} (避免频选衰落):
        \begin{itemize}
            \item 信号带宽:$B_s \leq (1/3 \sim 1/5) \Delta f$
            \item 码元宽度:$T_s \geq (3 \sim 5) \tau_m$
        \end{itemize}
    \end{itemize}
\end{kbox}

\begin{kbox}{4. 信道噪声 (高斯白噪声)}
    \textbf{统计特性}:
    \begin{itemize}
        \item 双边 PSD:$P_n(f) = n_0 / 2$ (W/Hz)
        \item 自相关:$R_n(\tau) = \frac{n_0}{2} \delta(\tau)$
        \item 一维 PDF (正态分布):
        \[ f_n(v) = \frac{1}{\sqrt{2\pi}\sigma_n} \exp\left(-\frac{v^2}{2\sigma_n^2}\right) \]
    \end{itemize}
    
    \textbf{功率计算}:
    \begin{itemize}
        \item 噪声功率:$N = \int_{-\infty}^{\infty} P_n(f) df$
        \item 等效带宽:$B_n = \frac{\int_0^{\infty} P_n(f) df}{P_n(f_0)}$
    \end{itemize}
\end{kbox}

\begin{kbox}{5. 信道容量 (Channel Capacity)}
    \textbf{离散信道}:
    \[ C = \max_{P(x)} [H(x) - H(x/y)] \quad (\text{b/符号}) \]
    其中 $H(x/y)$ 为损失信息量(平均条件熵)。
    
    \tcbline
    
    \textbf{连续信道 (香农公式 Shannon Formula)}:
    \[ C = B \log_2 \left( 1 + \frac{S}{N} \right) = B \log_2 \left( 1 + \frac{S}{n_0 B} \right) \]
    \begin{itemize}
        \item $B$:带宽 (Hz),$S$:信号功率 (W)
        \item $n_0$:噪声单边 PSD (W/Hz)
    \end{itemize}
    
    \textbf{带宽无限大极限}:
    \[ \lim_{B \to \infty} C \approx 1.44 \frac{S}{n_0} \]
    这意味着信道容量有极限值,不能通过无限增加带宽来无限增加容量。
\end{kbox}