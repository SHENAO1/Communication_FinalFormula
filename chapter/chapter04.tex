\section{信源编码}

\begin{kbox}{抽样定理 (Sampling Theorem)}
    \begin{itemize}
        \item \textbf{低通模拟信号抽样定理} \pptpage{12}
        \begin{itemize}
            \item \textbf{条件}:为了无失真恢复原信号,抽样速率 $f_s$ 需满足:
            \begin{equation}
                f_s \ge 2f_H \quad \text{且} \quad T_s \le \frac{1}{2f_H}
            \end{equation}
            其中 $f_H$ 为信号最高频率。
            \item \textbf{奈奎斯特速率}:$f_s = 2f_H$
            \item \textbf{奈奎斯特间隔}:$T_s = \frac{1}{2f_H}$
        \end{itemize}
        
        \item \textbf{理想抽样 (Ideal Sampling)} \pptpage{13}
        \begin{itemize}
            \item \textbf{时域}:利用单位冲激序列 $\delta_T(t)$ 进行乘积:
            % --- 修正:使用 split 环境将长公式换行对齐 ---
            \begin{equation}
            \begin{split}
                m_s(t) &= m(t)\sum_{n=-\infty}^{\infty}\delta(t-nT_s) \\
                       &= \sum_{n=-\infty}^{\infty}m(nT_s)\delta(t-nT_s)
            \end{split}
            \end{equation}
            \item \textbf{频域}:频谱以 $f_s$ 为周期搬移:
            \begin{equation}
                M_s(f) = \frac{1}{T_s}\sum_{n=-\infty}^{\infty}M(f-nf_s)
            \end{equation}
        \end{itemize}

        \item \textbf{平顶抽样与孔径失真} \pptpage{26}
        \begin{itemize}
            \item \textbf{产生}:理想抽样脉冲经过形状为 $h(t)$ 的保持电路。
            \item \textbf{频域关系}:
            \begin{equation}
                M_H(f) = M_s(f) \cdot H(f)
            \end{equation}
            \item \textbf{孔径失真}:由 $H(f) = T_s \text{Sa}(\pi f T_s)$ (矩形脉冲) 引起的高频衰减,需在接收端使用 \textbf{修正滤波器} 进行补偿。
        \end{itemize}
    \end{itemize}
\end{kbox}

\begin{kbox}{量化 (Quantization)}
    \begin{itemize}
        \item \textbf{均匀量化 (Uniform Quantization)} \pptpage{32}
        \begin{itemize}
            \item \textbf{量化间隔}:$\Delta v = \frac{b-a}{M}$
            \item \textbf{量化噪声功率} (重要结论):
            \begin{equation}
                N_q = \frac{(\Delta v)^2}{12}
            \end{equation}
            \item \textbf{信号量噪比 ($S/N_q$)} \pptpage{34}:
            % 使用 multline 确保长公式能够适应窄栏
            \begin{multline}
                (S/N_q)_{\text{dB}} \approx \\ 4.8 + 6n + 10\lg(S/x_{max}^2)
            \end{multline}
            \textbf{结论}:编码位数 $n$ 每增加 1 bit,信噪比提高约 \textbf{6dB}。
        \end{itemize}

        \item \textbf{非均匀量化 (Non-uniform)} \pptpage{38}
        \begin{itemize}
            \item \textbf{原理}:先压缩 (Compress),再均匀量化,最后扩张 (Expand)。目的:提高\textbf{小信号}的量噪比。
            \item \textbf{A律压缩特性} (A-Law, 欧/中标准, $A=87.6$) \pptpage{44}:
            % 微调格式,确保分数在窄栏中不溢出
            \begin{equation}
                y = \begin{cases} 
                \frac{Ax}{1+\ln A}, & 0 < x \le \frac{1}{A} \\[6pt] % 增加行间距
                \frac{1+\ln(Ax)}{1+\ln A}, & \frac{1}{A} \le x \le 1 
                \end{cases}
            \end{equation}
            \item \textbf{$\mu$律压缩特性} (美/日标准, $\mu=255$):
            \begin{equation}
                y = \frac{\ln(1+\mu x)}{\ln(1+\mu)}, \quad 0 \le x \le 1
            \end{equation}
            \item \textbf{改善度}:$[Q]_{\text{dB}} = 20\lg y'$,即取决于压缩曲线斜率。
        \end{itemize}
    \end{itemize}
\end{kbox}

\begin{kbox}{脉冲编码调制 (PCM)}
    \begin{itemize}
        \item \textbf{基本参数} \pptpage{72}
        设模拟信号最高频率为 $f_H$,抽样率为 $f_s$,编码位数为 $N$。
        \item \textbf{信息传输速率 (比特率)}:
        \begin{equation}
            R_b = f_s \cdot N \ge 2f_H \cdot N
        \end{equation}
        \item \textbf{传输带宽} (第一零点带宽):
        \begin{equation}
            B = R_b = f_s \cdot N \quad (\text{NRZ矩形})
        \end{equation}
        \item \textbf{典型应用:数字电话}
        \begin{itemize}
            \item 话音带宽 $B \approx 3.4\text{kHz} \to f_s = 8000\text{Hz}$
            \item 量化位数 $N=8$ (A律13折线)
            \item \textbf{比特率}:$R_b = 8000 \times 8 = \mathbf{64\text{kbps}}$
        \end{itemize}
    \end{itemize}
\end{kbox}