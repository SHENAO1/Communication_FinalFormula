\section{数字基带传输}

\subsection{数字基带信号的表示与编码}

\begin{kbox}{基带信号一般表达式 (随机脉冲序列) \pptpage{15}}
    \begin{equation*}
        s(t) = \sum_{n=-\infty}^{\infty} a_n g(t - nT_B)
    \end{equation*}
    \begin{itemize}
        \item $a_n$:第 $n$ 个码元的电平取值(统计独立的随机量)
        \item $g(t)$:单个脉冲波形;$T_B$:码元持续时间
    \end{itemize}
\end{kbox}

% 【修改点1】添加垂直间距,解决挨得太近的问题
\vspace{1em} 

\begin{enumerate}
    \item \textbf{差分编码与译码(相对码)} \pptpage{13}
    \begin{itemize}
        \item \textbf{编码}:$b_n = a_n \oplus b_{n-1}$ (克服相位模糊)
        \item \textbf{译码}:$a_n = b_n \oplus b_{n-1}$
    \end{itemize}
    
    \item \textbf{多电平(四电平)定义} \pptpage{14}
    \begin{itemize}
        \item 映射关系:$00 \to +3E,\; 01 \to +E,\; 10 \to -E,\; 11 \to -3E$
    \end{itemize}
\end{enumerate}

\subsection{基带信号的功率谱密度 (PSD)}

\begin{kbox}{功率谱分解通用公式 \pptpage{17-18}}
    \begin{equation*}
        P_s(f) = P_u(f) + P_v(f)
    \end{equation*}
    \tcblower
    \begin{itemize}
        \item \textbf{连续谱(交变波)}:
        \[ P_u(f) = f_B p(1-p) \left| G_1(f) - G_2(f) \right|^2 \]
        
        \item \textbf{离散谱(稳态波)}:
        % 【修改点2】使用 aligned 环境换行,防止公式超出分栏宽度
        \[
        \begin{aligned}
            P_v(f) = & \sum_{m=-\infty}^{\infty} \Big| f_B [  p G_1(mf_B) \\
            & + (1-p)G_2(mf_B)] \Big|^2 \delta(f - mf_B)
        \end{aligned}
        \]
        
        \item \textbf{离散谱消失条件}:$p = \frac{1}{1 - g_1(t)/g_2(t)}$
    \end{itemize}
\end{kbox}

\noindent\textbf{典型波形的功率谱密度:}
\begin{itemize}
    \item \textbf{单极性 NRZ} \pptpage{19}:
    $P_s(f) = \frac{1}{4} T_s \mathrm{Sa}^2(\pi f T_s) + \frac{1}{4} \delta(f)$
    
    \item \textbf{单极性 RZ (占空比1/2)} \pptpage{20}:
    $P_s(f) = \frac{1}{16} T_s \mathrm{Sa}^2(\frac{\pi f T_s}{2}) + \frac{1}{16} \sum_{m} \mathrm{Sa}^2(\frac{m\pi}{2}) \delta(f - mf_s)$
    
    \item \textbf{双极性等概 NRZ ($p=1/2$)} \pptpage{21}:
    $P_s(f) = T_s \mathrm{Sa}^2(\pi f T_s)$ (\textcolor{alertred}{无离散谱、无直流})
\end{itemize}

\subsection{码间干扰 (ISI) 与奈奎斯特准则}

\begin{kbox}{接收抽样点信号分解 \pptpage{41-43}}
    % 【修改点3】分三行对齐,解决长公式和下括号导致的溢出
    \[
    \begin{aligned}
        r(kT_s + t_0) = & \underbrace{a_k h(t_0)}_{\text{有用信号}} \\
        & + \underbrace{\sum_{n \neq k} a_n h(kT_s + t_0 - nT_s)}_{\text{码间干扰 ISI}} \\
        & + \underbrace{n_R(kT_s + t_0)}_{\text{噪声}}
    \end{aligned}
    \]
    其中 $h(t) = \mathcal{F}^{-1}\{G_T(\omega)C(\omega)G_R(\omega)\}$ 为系统总响应。
\end{kbox}

\subsubsection*{1. 奈奎斯特第一准则 (无ISI准则)}
\begin{itemize}
    \item \textbf{时域条件} \pptpage{50}:
    \[ h(mT_s) = \begin{cases} 1, & m=0 \\ 0, & m \neq 0 \end{cases} \]
    \item \textbf{频域条件} \pptpage{52, 55}:
    \[ \sum_{i=-\infty}^{\infty} H\left(\omega + \frac{2\pi i}{T_s}\right) = T_s, \quad |\omega| \leq \frac{\pi}{T_s} \]
\end{itemize}

\subsubsection*{2. 系统带宽指标}
\begin{enumerate}
    \item \textbf{理想低通系统} \pptpage{58-59}:
    \begin{itemize}
        \item 奈奎斯特带宽:$B = \frac{1}{2T_B} = f_N$
        \item 最高频带利用率:$\eta = R_B/B = 2$ (Baud/Hz)
    \end{itemize}
    \item \textbf{余弦滚降系统} \pptpage{61-62}:
    \begin{itemize}
        \item 带宽:$B = (1+\alpha)f_N$ ($\alpha$ 为滚降系数)
        \item 利用率:$\eta = \frac{2}{1+\alpha}$ (Baud/Hz)
    \end{itemize}
\end{enumerate}

\subsection{基带系统的抗噪声性能}

\begin{itemize}
    \item \textbf{噪声方差} \pptpage{73}:$\sigma_n^2 = \int_{-\infty}^{\infty} \frac{n_0}{2} |G_R(f)|^2 df$
    \item \textbf{误码率 $P_e$ 与最佳门限 $V_d^*$}:
\end{itemize}

\begin{kbox}{误码率公式速查}
    \textbf{1. 双极性基带系统} \pptpage{78-79}
    \begin{itemize}
        \item 最佳门限:$V_d^* = \frac{\sigma_n^2}{2A} \ln \frac{P(0)}{P(1)}$
        \item \textbf{等概时 ($P(0)=P(1)$)}:
        \[ V_d^* = 0, \quad P_e = \frac{1}{2} \operatorname{erfc}\left(\frac{A}{\sqrt{2}\sigma_n}\right) \]
    \end{itemize}
    
    \tcblower
    
    \textbf{2. 单极性基带系统} \pptpage{81-82}
    \begin{itemize}
        \item 最佳门限:$V_d^* = \frac{A}{2} + \frac{\sigma_n^2}{A} \ln \frac{P(0)}{P(1)}$
        \item \textbf{等概时}:
        \[ V_d^* = \frac{A}{2}, \quad P_e = \frac{1}{2} \operatorname{erfc}\left(\frac{A}{2\sqrt{2}\sigma_n}\right) \]
    \end{itemize}
\end{kbox}

\subsection{部分响应系统与时域均衡}

\subsubsection*{1. 第I类部分响应 (奈奎斯特第二准则)}
\begin{itemize}
    \item \textbf{编码}:$c_k = a_k + a_{k-1}$
    \item \textbf{频域} \pptpage{87}:$G(f) = 2T_s \cos(\pi f T_s), \quad |f| \leq \frac{1}{2T_s}$
    \item \textbf{预编码} \pptpage{90}:$b_k = a_k \oplus b_{k-1}$ (防止差错传播)
\end{itemize}

\subsubsection*{2. 时域均衡 (TDE)}
\begin{itemize}
    \item \textbf{横向滤波器输出} \pptpage{98}:$y_k = \sum_{n=-N}^{N} C_n x_{k-n}$
    \item \textbf{设计目标}:使 $y_k$ 满足无 ISI 条件。
    \item \textbf{均衡器频率响应} \pptpage{96}:
    \[ T(\omega) = \frac{T_B}{\sum_i H(\omega + \frac{2\pi i}{T_B})} \]
\end{itemize}

\subsection{实验评估:眼图 \pptpage{103}}
\begin{itemize}
    \item \textbf{眼睛张开度}:反映 ISI 和噪声强弱(开口大则ISI小)。
    \item \textbf{最佳抽样时刻}:眼睛张开最大的时刻。
    \item \textbf{斜边斜率}:反映对定时误差的灵敏度(斜率越大越敏感)。
\end{itemize}