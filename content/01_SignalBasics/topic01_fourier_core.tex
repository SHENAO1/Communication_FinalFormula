\section{绪论与信号基础}

% --- 模块一:傅里叶变换 (核心理论) ---
\begin{kbox}{信号分析基础:傅里叶变换}
    \begin{enumerate}
        \item \textbf{定义} \pptpage{6}
        \begin{itemize}
            \item 正变换:$F(\omega)=\int_{-\infty}^{\infty} f(t) e^{-j \omega t} d t$
            \item 反变换:$f(t)=\frac{1}{2 \pi} \int_{-\infty}^{\infty} F(\omega) e^{j \omega t} d \omega$
        \end{itemize}

        \item \textbf{常用运算特性} \pptpage{7-8}
        \begin{itemize}
            \item \textbf{共轭对称性}(补充):
            实信号 $f(t)$ 的频谱满足 $F(\omega) = F^*(-\omega)$。
            即:\textbf{幅度谱为偶函数,相位谱为奇函数}。
            \item \textbf{标度换算}:
            $f(a t) \leftrightarrow \frac{1}{|a|} F\left(\frac{\omega}{a}\right)$
            \item \textbf{时移特性}:
            $f(t-t_{0}) \leftrightarrow e^{-j \omega t_{0}} F(\omega)$
            \item \textbf{频移特性}:
            $e^{j \omega_{0} t} f(t) \leftrightarrow F(\omega-\omega_{0})$
            \item \textbf{调制特性}:
            $f(t) \cos \omega_{0} t \leftrightarrow \frac{1}{2}[F(\omega+\omega_{0})+F(\omega-\omega_{0})]$
            \item \textbf{时域卷积}:
            $f_{1}(t) * f_{2}(t) \leftrightarrow F_{1}(\omega) \cdot F_{2}(\omega)$
            \item \textbf{微分特性}:
            $\frac{d^{n}}{d t^{n}} f(t) \leftrightarrow(j \omega)^{n} F(\omega)$
        \end{itemize}

        \item \textbf{常用信号对与推论} \pptpage{10-11}
        \begin{itemize}
            \item \textbf{矩形脉冲}(重要):
            宽度为 $\tau$ 的门函数 $g_\tau(t) \leftrightarrow \tau \operatorname{Sa}\left(\frac{\omega \tau}{2}\right)$
            \begin{itemize}
                \item[$\circ$] 注:$Sa(x)=\frac{\sin x}{x}$, $sinc(x)=\frac{\sin \pi x}{\pi x}$
                \item[$\circ$] 关系:$Sa(\frac{\omega\tau}{2}) = sinc(f\tau)$
                \item[$\circ$] \textbf{对偶特性(笔记补充)}:
                $Sa(\tau t) \leftrightarrow \frac{\pi}{\tau} g_{2\tau}(\omega)$
                \\ (含义:时域 Sa 函数对应频域宽度为 $2\tau$ 的矩形谱)
            \end{itemize}
            \item \textbf{三角脉冲}:
            等腰三角脉冲可分解为两个矩形脉冲的卷积。
            
            % --- TikZ 绘图 ---
            \begin{center}
            \begin{tikzpicture}[scale=0.55, >=latex, font=\scriptsize]
                \tikzstyle{axis}=[->, thin, gray]
                \tikzstyle{signal}=[thick, mainblue] 

                % --- 图1:三角波 ---
                \draw[axis] (-1.8,0) -- (1.8,0);
                \draw[axis] (0,-0.2) -- (0,1.8);
                \draw[signal] (-1.2, 0) -- (0, 1.4) node[right, black]{$A^2\tau$} -- (1.2, 0);
                \node[below] at (-1.2, 0) {$-\tau$};
                \node[below] at (1.2, 0) {$\tau$};

                % --- 等号 ---
                \node at (2.2, 0.7) {\large $=$};
                % --- 图2:矩形波1 ---
                \begin{scope}[shift={(4.5,0)}]
                    \draw[axis] (-1.2,0) -- (1.2,0);
                    \draw[axis] (0,-0.2) -- (0,1.8);
                    \draw[signal] (-0.6, 0) -- (-0.6, 1) -- (0.6, 1) node[midway, above, black, xshift=7pt]{$A$} -- (0.6, 0);
                    \node[below] at (-0.6, 0) {$-\frac{\tau}{2}$};
                    \node[below] at (0.6, 0) {$\frac{\tau}{2}$};
                \end{scope}

                % --- 卷积符号 ---
                \node at (6.2, 0.7) {\large $*$};
                % --- 图3:矩形波2 ---
                \begin{scope}[shift={(8,0)}]
                    \draw[axis] (-1.2,0) -- (1.2,0);
                    \draw[axis] (0,-0.2) -- (0,1.8);
                    \draw[signal] (-0.6, 0) -- (-0.6, 1) -- (0.6, 1) node[midway, above, black, xshift=7pt]{$A$} -- (0.6, 0);
                    \node[below] at (-0.6, 0) {$-\frac{\tau}{2}$};
                    \node[below] at (0.6, 0) {$\frac{\tau}{2}$};
                \end{scope}
            \end{tikzpicture}
            \end{center}

            \item \textbf{双边指数信号}:
            $f(t) = e^{-\alpha t}u(t) - e^{\alpha t}u(-t) \leftrightarrow \frac{-2j\omega}{\alpha^2+\omega^2} \quad (\alpha>0)$
            \item \textbf{符号函数}:
            由上式令 $\alpha \to 0$ 可得:$\operatorname{sgn}(t) \leftrightarrow \frac{2}{j\omega}$
            \item \textbf{单位冲激串}:
            $\delta_{T}(t) \leftrightarrow \omega_{0} \sum_{n=-\infty}^{\infty} \delta(\omega-n \omega_{0})$
        \end{itemize}
    \end{enumerate}
\end{kbox}