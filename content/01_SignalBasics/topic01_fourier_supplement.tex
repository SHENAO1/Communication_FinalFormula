% --- 模块一补充:笔记推导细节 ---
\begin{examplebox}{笔记补充:Sa函数与门函数的对偶性推导}
    \textbf{1. 基础变换回顾}
    \begin{itemize}
        \item 门函数 $\to$ Sa函数:
        \[ g_{\tau}(t) \longleftrightarrow \tau Sa\left(\frac{\omega \tau}{2}\right) \]
        \item Sa函数 $\to$ 门函数(利用对偶性):
        \[ Sa(\Omega t) \longleftrightarrow \frac{\pi}{\Omega} g_{2\Omega}(\omega) \]
    \end{itemize}
    
    \tcbline % 分割线
    
    \textbf{2. 频域 $f$ 下的对应关系推导}
    
    \textit{问题:如何将 $Sa(\Omega t)$ 的频谱表示为关于 $f$ 的门函数?}
    
    \begin{itemize}
        \item \textbf{变量代换}:
        已知角频率与频率关系为 $w = 2\pi f$。
        \item \textbf{带宽边界映射}:
        对于频域门函数 $g_{2\Omega}(\omega)$,其截止角频率为 $w = \pm \Omega$(宽度为 $2\Omega$)。
        \item \textbf{计算 $f$ 域宽度}:
        当 $w = \Omega$ 时,对应的频率 $f$ 为:
        \[ f = \frac{w}{2\pi} = \frac{\Omega}{2\pi} \]
        因此,总宽度(即门函数的下标)为:
        \[ \text{Width}_f = \frac{\Omega}{2\pi} - \left(-\frac{\Omega}{2\pi}\right) = \frac{\Omega}{\pi} \]
        \item \textbf{最终结论}:
        \[ \frac{\Omega}{\pi} Sa(\Omega t) \longleftrightarrow g_{\frac{\Omega}{\pi}}(f) \]
    \end{itemize}
\end{examplebox}