% content/01_SignalBasics/topic01_time_freq_relation.tex

\begin{intuitionbox}{核心直观:时域越窄,频域越宽}
    \textbf{现象描述}:脉冲信号的持续时间(脉宽 $\tau$)与它所占据的频带宽度(带宽 $B$)成反比关系。
    \[ \tau \downarrow \quad \Longrightarrow \quad B \uparrow \]
    
    \tcbline
    
    \textbf{物理直观解释(变化的快慢)}:
    傅里叶变换的本质是将信号分解为不同频率正弦波的叠加。
    \begin{itemize}
        \item \textbf{长脉冲(宽)}:电平变化缓慢。要合成一个缓慢变化的波形,主要需要\textbf{低频}分量,不需要高频分量参与。因此频谱集中在低频,带宽窄。
        \item \textbf{短脉冲(窄)}:电平在极短瞬间从 0 跳变到 1 再跳回 0。这种\textbf{极其陡峭}的斜率变化,意味着信号变化极快。只有\textbf{高频}正弦波才能提供如此快的变化率。因此需要大量高频分量叠加,带宽极宽。
    \end{itemize}
\end{intuitionbox}

\begin{kbox}{数学推导:矩形脉冲与 Sinc 函数}
    设矩形脉冲信号 $x(t) = \text{rect}(t/\tau)$,其宽度为 $\tau$。
    根据傅里叶变换对:
    \begin{equation}
        x(t) = \text{rect}\left(\frac{t}{\tau}\right) \stackrel{\mathcal{F}}{\longleftrightarrow} X(f) = \tau \cdot \text{sinc}(f\tau)
    \end{equation}
    其中 $\text{sinc}(x) = \frac{\sin(\pi x)}{\pi x}$。

    \begin{itemize}
        \item \textbf{第一零点带宽}:Sinc 函数的第一个零点出现在 $f\tau = 1$ 处。
        \item \textbf{带宽公式}:
        \[ B = \frac{1}{\tau} \]
    \end{itemize}
    
    \tcbline
    
    \textbf{图解演示}:
    
    \begin{center}
    \begin{tikzpicture}
        % --- 定义通用样式以保持整洁 ---
        \pgfplotsset{
            miniplot/.style={
                width=3.6cm, height=2.8cm,  % [优化] 尺寸微调,防止重叠
                axis lines=middle,
                xtick=\empty, ytick=\empty,
                ymin=-0.3, ymax=1.6,        % [优化] 增加上方留白,防止标签撞线
                xmin=-3.5, xmax=3.5,
                clip=false,                 % 允许标签超出绘图区
                % 统一坐标轴标签位置:末端外侧
                every axis x label/.style={at={(current axis.east)}, anchor=west, xshift=2pt},
                every axis y label/.style={at={(current axis.north)}, anchor=south, yshift=2pt}
            }
        }

        % ==========================================
        % 左侧组:宽脉冲 -> 窄频谱
        % ==========================================
        \begin{scope}[xshift=0cm]
            % 标题
            \node[font=\bfseries\scriptsize, align=center, anchor=south] at (0, 2.2) {情形 A: 宽脉冲 ($\tau$ 大)};
            
            % 1. 时域图 (A)
            \begin{axis}[
                miniplot,
                at={(0,1.2cm)},
                xlabel={$t$}, ylabel={$x(t)$},
                xmin=-2.5, xmax=2.5,
            ]
                % 绘制脉冲
                \draw[thick, mainblue] (axis cs:-1,0) -- (axis cs:-1,1) -- (axis cs:1,1) -- (axis cs:1,0);
                % 标注宽度 (内部)
                \draw[<->, >=stealth, thin] (axis cs:-1, 0.5) -- (axis cs:1, 0.5) node[midway, above, font=\tiny] {$\tau$};
            \end{axis}

            % 2. 频域图 (A)
            \begin{axis}[
                miniplot,
                at={(0,-1.5cm)}, % [优化] 增加垂直间距,防止上下打架
                xlabel={$f$}, ylabel={$|X(f)|$},
                xmin=-4.5, xmax=4.5,
            ]
                % 绘制窄 Sinc
                \addplot[thick, alertred, domain=-4.2:4.2, samples=100] {abs(sin(deg(x*1.5))/(x*1.5+0.001))}; 
                % 标注 (移动到侧边,避免遮挡)
                \draw[->, thin, alertred] (axis cs:2.5, 0.6) -- (axis cs:0.8, 0.4);
                \node[font=\tiny, alertred, anchor=west] at (axis cs:2.5, 0.6) {频带窄};
            \end{axis}
        \end{scope}

        % ==========================================
        % 右侧组:窄脉冲 -> 宽频谱
        % ==========================================
        \begin{scope}[xshift=4.2cm]
            % 标题
            \node[font=\bfseries\scriptsize, align=center, anchor=south] at (0, 2.2) {情形 B: 窄脉冲 ($\tau$ 小)};
            
            % 3. 时域图 (B)
            \begin{axis}[
                miniplot,
                at={(0,1.2cm)},
                xlabel={$t$},
                xmin=-2.5, xmax=2.5,
            ]
                % 绘制窄脉冲
                \draw[thick, mainblue] (axis cs:-0.3,0) -- (axis cs:-0.3,1) -- (axis cs:0.3,1) -- (axis cs:0.3,0);
                % 标注宽度 (外部箭头,避免挤在中间)
                \draw[->, >=stealth, thin] (axis cs:-0.8, 0.5) -- (axis cs:-0.3, 0.5);
                \draw[->, >=stealth, thin] (axis cs:0.8, 0.5) -- (axis cs:0.3, 0.5);
                \node[font=\tiny] at (axis cs:1.0, 0.5) {$\tau$};
            \end{axis}

            % 4. 频域图 (B)
            \begin{axis}[
                miniplot,
                at={(0,-1.5cm)},
                xlabel={$f$},
                xmin=-4.5, xmax=4.5,
            ]
                % 绘制宽 Sinc
                \addplot[thick, alertred, domain=-4.2:4.2, samples=100] {abs(sin(deg(x*0.5))/(x*0.5+0.001))};
                % 标注 (放在曲线下方空旷处)
                \node[font=\tiny, alertred] at (axis cs:2.2, 0.4) {频带宽};
            \end{axis}
        \end{scope}
    \end{tikzpicture}
    \end{center}
\end{kbox}

\begin{kbox}{极限情况:记忆锚点}
    \begin{itemize}
        \item \textbf{极端窄($\delta$ 冲击)}:
        时域宽度 $\tau \to 0$。为了制造这样一个无限陡峭的瞬间冲击,需要全宇宙所有的频率共同作用。
        \[ \delta(t) \stackrel{\mathcal{F}}{\longleftrightarrow} 1 \quad (\text{带宽 } B \to \infty) \]
        
        \item \textbf{极端宽(直流 DC)}:
        时域宽度 $\tau \to \infty$(永恒不变)。没有任何变化,不需要任何非零频率。
        \[ 1 \stackrel{\mathcal{F}}{\longleftrightarrow} \delta(f) \quad (\text{带宽 } B \to 0) \]
    \end{itemize}
\end{kbox}