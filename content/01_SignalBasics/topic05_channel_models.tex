% --- 模块四:信道模型 ---
\begin{kbox}{信道与衰落}
    \begin{enumerate}
        \item \textbf{调制信道模型} \pptpage{59}
        \begin{equation*}
            e_{o}(t)=k(t) e_{i}(t)+n(t)
        \end{equation*}

        \item \textbf{多径效应与正交分解} \pptpage{70}
        接收信号 $R(t)$ 为多径叠加:
        \begin{align*}
            R(t) &= \sum_{i=1}^{n} \mu_i(t) \cos[\omega_c (t - \tau_i(t))] \\
                 & \qquad \text{\small (原始多径公式)} \\
                 &= \sum_{i=1}^{n} \mu_i(t) \cos[\omega_c t + \varphi_i(t)] \\
                 & \quad {\color{red}\text{其中 } \varphi_i(t) = -\omega_c \tau_i(t) } \\ 
                 &= \sum_{i=1}^{n} \mu_i(t) \cos(\omega_c t) \cos(\varphi_i(t)) \\
                 &\quad - \sum_{i=1}^{n} \mu_i(t) \sin(\omega_c t) \sin(\varphi_i(t)) \\
                 &= X_c(t) \cos(\omega_c t) - X_s(t) \sin(\omega_c t) \\
                 &= V(t) \cos[\omega_c t + \varphi(t)]
        \end{align*}
        \textbf{参数含义}:
        \begin{itemize}
            \item $\mu_i(t)$:第 $i$ 条路径接收信号的\textbf{振幅}。
            \item $\tau_i(t)$:第 $i$ 条路径接收信号的\textbf{时延}。
            \item $\varphi_i(t)$:第 $i$ 条路径接收信号的\textbf{相位},且 $\varphi_i(t) = -\omega_c \tau_i(t)$。
            \item $X_c(t), X_s(t)$:同相与正交分量,大量路径叠加时视为高斯随机过程。
            \item $V(t)$:包络,服从瑞利分布;$\varphi(t)$:合成相位,服从均匀分布。
        \end{itemize}

        \item \textbf{衰落类型与相干带宽} \pptpage{74}
        \begin{itemize}
            \item 相干带宽:$\Delta f \approx 1 / \tau_{m}$ ($\tau_{m}$ 为最大多径时延差)
            \item \textbf{平坦衰落}:信号带宽 $B < \Delta f$
            \item \textbf{频率选择性衰落}:信号带宽 $B > \Delta f$
        \end{itemize}
    \end{enumerate}
\end{kbox}