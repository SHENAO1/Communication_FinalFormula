% --- 模块五:性能指标 ---
\begin{kbox}{通信系统性能指标}
    \begin{enumerate}
        \item \textbf{模拟系统} \pptpage{87}
        \begin{equation*}
            \text{SNR}(\text{dB}) = 10 \lg (S_{i} / N_{i})
        \end{equation*}

        \item \textbf{补充定义:码元}
        \begin{itemize}
            \item 定义:承载信息的基本单位,是通信的“字符”或“信号”。
            \item 形式:固定时长的脉冲或波形(二进制2种,四进制4种)。
        \end{itemize}

        \item \textbf{数字系统:有效性 (速率)} \pptpage{88-89}
        \begin{itemize}
            \item \textbf{码元速率 ($R_B$)}:波特率(Baud),每秒传送的波形数(“货车”数量)。
            \item \textbf{信息速率 ($R_b$)}:比特率(bps),每秒传送的信息总量(“货物”总量)。
            \item \textbf{关系}:$R_{b} = R_{B} \log _{2} M$
                \begin{itemize}
                    \item 仅当各码元等概出现时,单码元信息量 $I = \log_2 M$。
                \end{itemize}
            \item \textbf{频带利用率}:$\eta = \frac{R_{B}}{B}$ (Baud/Hz) 或 $\frac{R_{b}}{B}$ (bps/Hz)。
        \end{itemize}

        \item \textbf{数字系统:可靠性 (误码/误信)} \pptpage{90-91}
        {\raggedright
        \begin{itemize}
            \item \textbf{误码率 ($P_e$)}:错误码元数占比,$P_e = n_{eB} / n_B$。
                \begin{itemize}
                    \item 二进制平均误码率:$P_e = P(0)P(1|0) + P(1)P(0|1)$。
                \end{itemize}
            \item \textbf{误信率 ($P_b$)}:错误比特数占比,$P_b = n_{eb} / n_b$。
            \item \textbf{关系}:
            \begin{itemize}
                \item 二进制 ($M=2$):$P_{b} = P_{e}$
                \item 多进制 ($M>2$):$P_{b} < P_{e}$
            \end{itemize}
        \end{itemize}
        }
    \end{enumerate}
\end{kbox}