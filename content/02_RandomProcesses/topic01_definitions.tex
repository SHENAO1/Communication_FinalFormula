\begin{kbox}{1. 随机过程定义与统计特性}
    \textbf{定义}:
    随时间作随机变化的一系列随机变量的集合。
        \begin{itemize}[leftmargin=3em]
            \item[(1)] 结构上是:所有样本函数 $\xi_i(t)$ 的集合;
            \item[(2)] 也是随机变量 $\xi(t)$ 的集合。
        \end{itemize}

    \tcbline

    \textbf{统计特性与物理意义}:
    \begin{itemize}
        \item \textbf{一维分布函数}:
        \[ F_1(x_1, t_1) = P\{\xi(t_1) \leq x_1\} \]
        \item \textbf{一维概率密度函数 (PDF)}:
        \[ f_1(x_1, t_1) = \frac{\partial F_1(x_1, t_1)}{\partial x_1} \]
        \item \textbf{均值 (数学期望)} —— \textcolor{mainblue}{直流分量}:
        \[ \mu(t) = E[\xi(t)] = \int_{-\infty}^{\infty} x f_1(x, t) dx \]
        \item \textbf{方差} —— \textcolor{mainblue}{交流功率}:
        \[ \sigma^2(t) = D[\xi(t)] = E\left[(\xi(t) - \mu(t))^2\right] \]
        \item \textbf{平均功率 (二阶原点矩)}:
        由 $DX = E(X^2) - (EX)^2$ 可知功率关系:
        \[ \underbrace{E[\xi^2(t)]}_{\text{平均功率}} = \underbrace{\sigma^2(t)}_{\text{交流功率}} + \underbrace{\mu^2(t)}_{\text{直流功率}} \]
        (当 $\mu(t)=0$ 时,平均功率等于交流功率)
        
        \item \textbf{自相关函数}:
        描述同一过程在任意两个时刻随机变量间的关联程度。
        \begin{align*}
            R(t_1, t_2) &= E[\xi(t_1)\xi(t_2)] \\
            &= \iint_{-\infty}^{\infty} x_1 x_2 f_2(x_1, x_2; t_1, t_2) dx_1 dx_2
        \end{align*}
        (其中 $f_2$ 为二维概率密度函数)
        
        \item \textbf{互相关函数}:
        \[ R_{\xi\eta}(t_1, t_2) = E[\xi(t_1)\eta(t_2)] \]
    \end{itemize}
\end{kbox}

% --- 核心概念辨析 ---
\begin{kbox}{核心概念辨析:样本函数 vs 随机变量}
    \textbf{误区}:随机过程的“样本”不是随机变量,二者观察维度不同。
    
    对于随机过程 $\xi(t, \omega)$:
    \begin{itemize}
        \item \textbf{样本函数 (Sample Function)}:\textcolor{alertred}{固定 $\omega$,变化 $t$}。
        \\ 指随机过程的\textbf{一次具体实现}(如做了一次实验记录下的波形)。既然实验已发生,它就是一条\textbf{确定的时间函数} $x(t)$,不再具有随机性。
        \item \textbf{随机变量 (Random Variable)}:\textcolor{alertred}{固定 $t$,变化 $\omega$}。
        \\ 指随机过程在\textbf{某一特定时刻}的状态。它包含该时刻所有可能的取值及其概率分布,是\textbf{随机的}。
        \item \textbf{总结}:样本函数是“时间的函数”,随机变量是“概率的函数”。二者同时固定时,唯一确定一个数值。
    \end{itemize}
    
    \tcbline
    
    \textbf{直观总结}:
    \begin{itemize}
        \item “样本”是\textbf{纵向的历史轨迹}(确定的曲线)。
        \item “随机变量”是\textbf{横向的时间切片}(统计分布)。
    \end{itemize}
\end{kbox}