\begin{kbox}{2. 平稳随机过程}
    \textbf{广义平稳 (WSS) 条件}:
    \begin{enumerate}
        \item 均值与时间无关:$\mu(t) = \mu$
        \item 自相关仅与间隔 $\tau$ 有关:$R(t_1, t_1 + \tau) = R(\tau)$
    \end{enumerate}

    \tcbline

    \textbf{狭义平稳 (SSS) 与 广义平稳 (WSS) 对比}:
    \begin{itemize}
        \item \textbf{定义区别}:SSS 要求所有统计特性不变;WSS 仅要求均值和自相关不变。
        \item \textbf{关系}:SSS (二阶矩有限) $\Rightarrow$ WSS;反之通常不成立
        \item \textbf{特例}:对于\textbf{高斯随机过程},WSS $\Leftrightarrow$ SSS。
    \end{itemize}
    
    \tcbline 
    
    \textbf{各态历经性 (遍历性)}:
    \begin{itemize}
        \item \textbf{定义}:若平稳过程的\textbf{统计平均}等于样本的\textbf{时间平均},则称其具有各态历经性。
        \item \textbf{意义}:\textbf{用一次实现的时间平均取代过程的统计平均}。简化测量,因为实际往往只有一个样本波形。
        \item \textbf{数学条件}:$\mu = \bar{\mu}$ 且 $R(\tau) = \bar{R}(\tau)$。
    \end{itemize}
    
    \tcbline
    
    \textbf{深入理解:各态历经性与平稳性的关系}
    \begin{itemize}
        \item \textbf{为什么任一样本能代表整体?} \\
        各态历经的物理含义是:\textbf{一个样本在无限长的时间内,遍历了随机过程所有可能的状态}。因此,观测这一个样本足够久,就等同于观测了整个集合。
        
        \item \textbf{为什么“各态历经 $\Rightarrow$ 平稳”?} \\
        因为“时间平均”是一个常数(积分区间是无穷大),若它等于“统计平均”,则统计平均也必须是常数(不随 $t$ 变化),这正是平稳过程的定义。
        
        \item \textbf{为什么“平稳 $\nRightarrow$ 各态历经”?} \\
        平稳只要求整体统计特性不变,但允许样本之间存在永久性差异(“老死不相往来”)。
        \\ \textit{反例}:$X(t)=C$($C$为随机常数)。样本要么恒为1,要么恒为-1。单个样本的时间平均(1或-1)不等于整体统计平均(0),它没有遍历所有状态。
    \end{itemize}
\end{kbox}