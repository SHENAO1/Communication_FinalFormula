\begin{kbox}{4. 高斯随机过程与误差函数 (基础结论)}
    \textbf{1. 概率密度与分布函数}
    \begin{itemize}
        \item \textbf{概率密度 (PDF)}:
        \[ f(x) = \frac{1}{\sqrt{2\pi}\sigma} \exp\left[-\frac{(x-\mu)^2}{2\sigma^2}\right] \]
        \item \textbf{分布函数 (CDF) 定义}:
        分布函数是 PDF 的积分(即 $x$ 落在 $(-\infty, b]$ 的概率):
        \[ F(b) = P(x \leq b) = \int_{-\infty}^{b} \frac{1}{\sqrt{2\pi}\sigma} \exp\left[-\frac{(x-\mu)^2}{2\sigma^2}\right] dx \]
    \end{itemize}
    
    \tcbline
    
    \textbf{2. 误差函数的由来}
    上述积分无法用初等函数直接表示,引入定义:
    \[ \mathrm{erf}(z) \triangleq \frac{2}{\sqrt{\pi}} \int_0^z e^{-u^2} du \]
    {\small \textcolor{gray}{注:系数 $2/\sqrt{\pi}$ 是为了归一化,确保 $\mathrm{erf}(\infty)=1$。}}
    
    \tcbline
    
    \textbf{3. $F(x)$ 的分段表达式}
    \[
    F(x) = 
    \begin{cases}
        \displaystyle \frac{1}{2} + \frac{1}{2} \mathrm{erf}\left( \frac{x-\mu}{\sqrt{2}\sigma} \right), & x \geq \mu \\[1.2em]
        \displaystyle 1 - \frac{1}{2} \mathrm{erfc}\left( \frac{x-\mu}{\sqrt{2}\sigma} \right), & x < \mu
    \end{cases}
    \]
\end{kbox}

% --- 插入详细推导部分 ---
\begin{kbox}{补充推导:高斯分布与误差函数详解}
    \textbf{1. 高斯随机过程定义} \\
    随机变量 $\xi(t)$ 的任意 $n$ 维分布服从正态分布,称为高斯过程。

    \tcbline

    % --- 新增部分:来自图片内容 ---
    \textbf{2. 高斯过程的重要性质}
    \begin{itemize}
        \item 高斯过程的 $n$ 维概率密度函数仅取决于它的\textcolor{red}{数学期望}、\textcolor{red}{方差}和\textcolor{red}{相关函数}。
        \item 对高斯过程,\textcolor{red}{广义平稳} $\Leftrightarrow$ \textcolor{red}{严平稳}。
        \item 高斯过程,在不同时刻的随机变量值,\textcolor{red}{互不相关} $\Leftrightarrow$ \textcolor{red}{互相独立}。
        \item 高斯过程(变量)的\textcolor{red}{线性变换}仍为高斯过程(变量)。
    \end{itemize}

    \tcbline
    % ---------------------------

    \textbf{3. 互补误差函数性质}
    \begin{align*}
        \text{erfc}(-z) &= 1 - \text{erf}(-z) \\
        &= 1 + \text{erf}(z) \\
        &= 2 - \text{erfc}(z)
    \end{align*}
    即:\textcolor{mainblue}{$\text{erfc}(z) + \text{erfc}(-z) = 2$}

    \tcbline

    \textbf{4. 分布函数 $F(x)$ 的积分推导细节} \\
    设正态分布概率密度函数为 $f(t)$,则分布函数 $F(x) = \int_{-\infty}^{x} f(t) dt$。

    \vspace{4pt}
    \textbf{(1) 当 $x > \mu$ 时 (落在均值右侧)}
    \[ F(x) = \int_{-\infty}^{\mu} f(t) dt + \int_{\mu}^{x} \frac{1}{\sqrt{2\pi}\sigma} e^{-\frac{(t-\mu)^2}{2\sigma^2}} dt \]
    显然由高斯分布对称性知:$\int_{-\infty}^{\mu} f(t) dt = \frac{1}{2}$。
    
    \textbf{换元法}:对后式,令 $y = \frac{t-\mu}{\sqrt{2}\sigma}$,则 $dy = \frac{1}{\sqrt{2}\sigma} dt$。
    \begin{itemize}
        \item 当 $t=x$ 时,$y=\frac{x-\mu}{\sqrt{2}\sigma}$;
        \item 当 $t=\mu$ 时,$y=0$。
    \end{itemize}
    代入积分项:
    \[ \int_{\mu}^{x} f(t) dt = \int_{0}^{\frac{x-\mu}{\sqrt{2}\sigma}} \frac{1}{\sqrt{\pi}} e^{-y^2} dy \]
    根据 $\text{erf}(z) = \frac{2}{\sqrt{\pi}} \int_{0}^{z} e^{-u^2} du$ 的定义,上式即为 $\frac{1}{2}\text{erf}(\frac{x-\mu}{\sqrt{2}\sigma})$。
    
    \noindent \textbf{故当 $x > \mu$ 时:}
    \[ \boxed{F(x) = \frac{1}{2} + \frac{1}{2} \text{erf}\left(\frac{x-\mu}{\sqrt{2}\sigma}\right)} \]

    \vspace{4pt}
    \textbf{(2) 当 $x < \mu$ 时 (落在均值左侧)}
    \[ F(x) = \int_{-\infty}^{\mu} f(t) dt - \int_{x}^{\mu} \frac{1}{\sqrt{2\pi}\sigma} e^{-\frac{(t-\mu)^2}{2\sigma^2}} dt \]
    同样利用对称性 $\int_{-\infty}^{\mu} f(t) dt = \frac{1}{2}$。
    
    \textbf{换元法}:对后式,令 $y = \frac{\mu-t}{\sqrt{2}\sigma}$,则 $dt = -\sqrt{2}\sigma dy$。
    \begin{itemize}
        \item 当 $t=x$ 时,$y=\frac{\mu-x}{\sqrt{2}\sigma}$;
        \item 当 $t=\mu$ 时,$y=0$。
    \end{itemize}
    (注意积分限变换抵消了负号),代入有:
    \[ \int_{x}^{\mu} f(t) dt = \int_{0}^{\frac{\mu-x}{\sqrt{2}\sigma}} \frac{1}{\sqrt{\pi}} e^{-y^2} dy = \frac{1}{2}\text{erf}\left(\frac{\mu-x}{\sqrt{2}\sigma}\right) \]
    
    \noindent \textbf{故当 $x < \mu$ 时:}
    \[ F(x) = \frac{1}{2} - \frac{1}{2} \text{erf}\left(\frac{\mu-x}{\sqrt{2}\sigma}\right) \]

    \tcbline

    \textbf{5. 统一用 $\text{erfc}$ 表示}
    
    又 $\text{erf}(x) = 1 - \text{erfc}(x)$,且 $\text{erfc}(-x) = 2 - \text{erfc}(x)$。
    
    对于 $x < \mu$ 的情况,直接代入:
    \[ 
    \begin{aligned}
        F(x) &= \frac{1}{2} \left[1 - \text{erf}\left(\frac{\mu-x}{\sqrt{2}\sigma}\right)\right] \\
             &= \textcolor{mainblue}{\frac{1}{2} \text{erfc}\left(\frac{\mu-x}{\sqrt{2}\sigma}\right)} 
    \end{aligned}
    \]
    
    对于 $x > \mu$ 的情况:
    \[ 
    \begin{aligned}
        F(x) &= \frac{1}{2} \left[2 - \text{erfc}\left(\frac{x-\mu}{\sqrt{2}\sigma}\right)\right] \\
             &= \textcolor{mainblue}{1 - \frac{1}{2} \text{erfc}\left(\frac{x-\mu}{\sqrt{2}\sigma}\right)} 
    \end{aligned}
    \]
\end{kbox}