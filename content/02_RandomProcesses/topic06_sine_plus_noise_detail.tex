\begin{kbox}{正弦波加窄带高斯噪声 (详细推导与模型)}
    
    % --- 第一部分:系统模型图解 ---
    \textbf{1. 物理模型}:
    
    \begin{center}
    \begin{tikzpicture}[
        % 【关键修改1】缩小节点间距,适应双栏宽度 (8cm左右)
        node distance=1.5cm, 
        auto,
        >=Latex, 
        font=\small\sffamily
    ]
        % 【关键修改2】收紧背景框坐标,确保总宽不超过 8cm
        % 左边界 -3.4, 右边界 4.2 -> 总宽 7.6cm (安全范围)
        \fill[boxbg] (-3.4,-1.5) rectangle (4.2, 1.2);
        
        % 顶部标题
        \node[text=procblue, anchor=north west] at (-3.4, 1.2) {接收端};

        % --- 节点定义 ---
        % 起点稍微向右偏移,保证居中
        \node (input) at (-2.6, 0) {}; 
        
        % 滤波器框 (稍微改小一点宽度)
        \node [draw=procblue, thick, fill=white, minimum height=0.9cm, minimum width=2.0cm, right=of input] (filter) {窄带滤波器};
        
        % 输出节点 (缩短距离)
        \node [right=2.2cm of filter] (output) {}; 

        % --- 连线与公式 ---
        % 输入信号
        \draw [->, thick, procblue] (input) -- node[name=inlabel, align=center, yshift=3pt] {\small $s(t) {+} n_w(t)$} (filter);
        
        % 输出信号
        \draw [->, thick, procblue] (filter) -- node[name=outlabel, align=center, yshift=3pt] {\small $r(t) {=} s(t) {+} n(t)$} (output);

        % --- 底部红色注释 (完全还原图示布局) ---
        
        % 1. 左侧文字节点 (调整坐标以适应新布局)
        \node[text=procblue, anchor=north, font=\footnotesize] (txt_signal) at (-2.3, -0.7) {正弦波已调信号};
        \node[text=procblue, anchor=north, font=\footnotesize] (txt_noise)  at (-0.1, -0.7) {高斯白噪声};
        
        % 2. 左侧红线 (使用相对坐标计算,精准指向)
        % 指向 s(t)
        \draw[red, thin] (txt_signal.north) -- ($(inlabel.south west)!0.4!(inlabel.south)$);
        % 指向 nw(t)
        \draw[red, thin] (txt_noise.north) -- ($(inlabel.south)!0.4!(inlabel.south east)$);

        % 3. 右侧文字与红线
        \node[text=procblue, anchor=north, font=\footnotesize] (txt_nbnoise) at (3.0, -0.7) {窄带高斯噪声};
        % 指向 n(t)
        \draw[red, thin] (txt_nbnoise.north) -- ($(outlabel.south)!0.7!(outlabel.south east)$);

    \end{tikzpicture}
    \end{center}

    \tcbline

    % --- 第二部分:数学推导 (保持简洁优化版) ---
    \textbf{2. 合成信号推导}:
    
    设输入信号为正弦波,噪声为窄带高斯过程:
    \begin{itemize}
        \item $s(t) = A \cos(\omega_c t + \theta)$ \quad ($\theta \sim U[0, 2\pi]$)
        \item $n(t)$:均值0,方差 $\sigma_\xi^2$ 的窄带高斯噪声
    \end{itemize}

    \vspace{0.5em}
    \textbf{正交分解步骤}:
    将 $n(t) = n_c(t)\cos\omega_c t - n_s(t)\sin\omega_c t$ 代入:
    
    {\small
    \begin{align*}
        r(t) &= A \cos(\omega_c t + \theta) + n(t) \\
             &= A[\cos\omega_c t \cos\theta - \sin\omega_c t \sin\theta] \\
             &\quad + [n_c(t)\cos\omega_c t - n_s(t)\sin\omega_c t] \\
             &= \underbrace{[A\cos\theta + n_c(t)]}_{\textcolor{textred}{z_c(t)}} \cos\omega_c t - \underbrace{[A\sin\theta + n_s(t)]}_{\textcolor{textgreen}{z_s(t)}} \sin\omega_c t
    \end{align*}
    }%
    
    \textbf{最终包络相位形式}:
    \[ r(t) = z(t) \cos[\omega_c t + \varphi(t)] \]

    \tcbline

    % --- 第三部分:分量与统计特性 ---
    \begin{minipage}[t]{0.48\textwidth}
        \textbf{同相与正交分量}:
        \begin{tcolorbox}[colback=subbg, boxrule=0pt, frame hidden, left=2pt, top=2pt, bottom=2pt]
        \small
        $z_c(t) = A \cos \theta + n_c(t)$ \\
        $z_s(t) = A \sin \theta + n_s(t)$
        \end{tcolorbox}
    \end{minipage}%
    \hfill
    \begin{minipage}[t]{0.48\textwidth}
        \textbf{合成包络与相位}:
        \begin{tcolorbox}[colback=subbg, boxrule=0pt, frame hidden, left=2pt, top=2pt, bottom=2pt]
        \small
        $z = \sqrt{z_c^2 + z_s^2}$ \\
        $\varphi = \arctan (z_s/z_c)$
        \end{tcolorbox}
    \end{minipage}
    
    \vspace{0.5em}
    \textbf{$z(t)$ 的统计特性}:
    \begin{itemize}
        \item $z(t)$ 服从 \textbf{莱斯分布 (Ricean)}。
        \item $A=0 \Rightarrow$ \textbf{瑞利分布} (纯噪声)。
        \item $A \gg \sigma_n \Rightarrow$ 近似 \textbf{高斯分布}。
    \end{itemize}

\end{kbox}