% content/03_Channels/topic01_models.tex

\begin{kbox}{1. 信道模型定义}
    \textbf{1. 调制信道 (Modulation Channel)}:研究模拟波形的传输
    \begin{itemize}
        \item \textbf{范围}:调制器输出端 $\to$ 解调器输入端
        \item \textbf{模型}:
        \[ r(t) = k(t) \cdot s_i(t) + n(t) \]
        \item $k(t)$:乘性干扰 (反映信道特性如衰落,若 $k(t)=C$ 则为恒参信道)
        \item $n(t)$:加性噪声 (通常假设为高斯白噪声)
    \end{itemize}
    
    \tcbline
    
    \textbf{2. 编码信道 (Coding Channel)}:研究数字序列的转移
    \begin{itemize}
        \item \textbf{范围}:编码器输出端 $\to$ 译码器输入端 (包含调制信道)
        \item \textbf{模型}:用转移概率描述 (如二进制对称信道 BSC)
        \item \textbf{误码率}:
        \[ P_e = P(0)P(1|0) + P(1)P(0|1) \]
    \end{itemize}
\end{kbox}

\begin{kbox}{2. 核心辨析:调制信道 vs 编码信道}
    \textbf{1. 层级与包含关系}
    \[ \text{编码} = \text{调制器} + \underbrace{\text{物理媒介} + \text{噪声}}_{\text{调制信道}} + \text{解调器} \]
    
    \tcbline
    
    \textbf{2. 关键区别对比}
    \begin{center}
    % 【优化】缩小字号,减小列间距,使用p列允许换行
    \footnotesize 
    \setlength{\tabcolsep}{2pt} 
    \renewcommand{\arraystretch}{1.3} % 稍微增加行高,避免换行后文字挤在一起
    
    % 计算逻辑:第一列占15%,后两列各占42%,总和约100%
    \begin{tabular}{c|p{0.4\linewidth}|p{0.4\linewidth}}
        \hline
        \textbf{维度} & \centering\textbf{调制信道}\par(内层/物理) & \centering\textbf{编码信道}\par(外层/逻辑) \tabularnewline
        \hline
        \textbf{信号} & 连续模拟波形 $s(t)$ & 离散数字序列 $0, 1$ \\
        \textbf{对象} & 物理特性\par (噪声、带宽) & 逻辑特性\par (误码率、容量) \\
        \textbf{参数} & 信噪比 $S/N$, $N_0$ & 误码率 $P_e$, 转移概率 \\
        \hline
    \end{tabular}
    \end{center}

    \tcbline
    
    \textbf{3. 因果关系 (为什么错码归咎于调制信道?)}
    \begin{itemize}
        \item \textbf{调制信道是“病因”}:物理世界的噪声 $n(t)$ 和衰落 $k(t)$ 导致模拟波形畸变。
        \item \textbf{编码信道是“症状”}:解调器对畸变波形进行判决,产生逻辑错误。
        \item \textbf{数学联系}:编码信道的 $P_e$ 是调制信道 $N_0$ 的函数。
        \[ \text{BPSK:} \quad P_e = Q\left(\sqrt{\frac{2E_b}{N_0}}\right) \]
    \end{itemize}
\end{kbox}