% content/03_Channels/topic02_constant_param.tex

\begin{kbox}{1. 恒参信道特性定义}
    \textbf{1. 传输函数}
    \[ H(\omega) = |H(\omega)| e^{j\varphi(\omega)} \]

    \textbf{2. 无失真传输条件}
    \begin{itemize}
        \item \textbf{幅频特性}:$|H(\omega)| = K$ (常数)
        \item \textbf{相频特性}:$\varphi(\omega) = \omega t_d$ (过原点的直线)
        \item \textbf{群迟延}:$\tau(\omega) = \frac{d\varphi(\omega)}{d\omega} = t_d$ (常数)
    \end{itemize}
    
    \textbf{3. 时域响应 (物理意义)}
    \begin{itemize}
        \item 冲激响应仅是延迟和缩放:$h(t) = K\delta(t - t_d)$
        \item 输出波形无形状改变,仅有滞后:$s_o(t) = K s(t - t_d)$
    \end{itemize}
\end{kbox}

\begin{examplebox}{2. 失真类型辨析}
    \begin{itemize}
        \item \textbf{幅频失真}:$|H(\omega)| \neq K$。
        \par 会导致模拟信号波形畸变。对数字信号影响相对较小(数字通信主要看判决时刻的电平)。
        
        \item \textbf{相频失真}:$\varphi(\omega)$ 非线性(即 $\tau(\omega)$ 不恒定)。
        \par 不同频率分量“跑”得快慢不一,导致波形在时间轴上“散开”(色散)。
        \par \textbf{后果}:产生严重的\textbf{码间串扰 (ISI)},是高速数字传输的大敌。
    \end{itemize}
\end{examplebox}