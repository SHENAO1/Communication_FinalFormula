% content/03_Channels/topic02_constant_param.tex

\begin{kbox}{1. 恒参信道特性定义}
    \textbf{1. 传输函数}
    \[ H(\omega) = |H(\omega)| e^{j\varphi(\omega)} \]

    \textbf{2. 无失真传输条件}
    \begin{itemize}
        \item \textbf{幅频特性}:$|H(\omega)| = K$ (常数)
        \item \textbf{相频特性}:$\varphi(\omega) = \omega t_d$ (过原点的直线)
        \item \textbf{群迟延}:$\tau(\omega) = \frac{d\varphi(\omega)}{d\omega} = t_d$ (常数)
    \end{itemize}
    
    \textbf{3. 时域响应}
    \par 冲激响应仅是延迟和缩放:$h(t) = K\delta(t - t_d)$
\end{kbox}

\begin{kbox}{2. 线性畸变:幅频失真与相频失真}
    % --- 幅频失真部分 ---
    \textbf{1. 幅频失真}
    \begin{itemize}
        \item \textbf{定义}:$|H(\omega)| \neq K$(信道对不同频率分量的增益不同)。
        \item \textbf{影响}:
            \begin{itemize}
                \item \textbf{模拟信号}:波形畸变 $\longrightarrow$ 信噪比 ($S/N$) 下降。
                \item \textbf{数字信号}:波形拖尾 $\longrightarrow$ 码间串扰 (ISI) $\longrightarrow$ 误码率增大。
            \end{itemize}
        \item \textbf{措施}:利用线性网络(均衡器)补偿,使其合成幅频特性在频带内为\textbf{水平直线}。
    \end{itemize}

    \tcbline % 分割线

    % --- 相频失真部分 ---
    \textbf{2. 相频失真 (群迟延失真)}
    \begin{itemize}
        \item \textbf{定义}:$\varphi(\omega)$ 非线性 $\iff$ 群迟延 $\tau(\omega)$ 不恒定。
        \item \textbf{影响}:
            \begin{itemize}
                \item \textbf{模拟信号}:语音影响较小(人耳不敏感),视频信号会产生重影/畸变。
                \item \textbf{数字信号}:不同频率分量传输快慢不一 $\longrightarrow$ \textbf{严重的码间串扰 (ISI)}。
            \end{itemize}
        \item \textbf{措施}:使用相位均衡器进行校正。
    \end{itemize}
\end{kbox}

\begin{examplebox}{3. 深度分析:为什么失真会导致串扰?本质是什么?}
    \textbf{1. 核心机制:从频域到时域的连锁反应}
    \par \textbf{频域视角(成因):} 数字信号(如方波)包含丰富的高频谐波。若信道带宽受限(幅频失真,通常为低通特性),高频分量被衰减或切除。
    \par \textbf{时域视角(现象):} 缺失了高频分量,方波的陡峭边缘变缓,脉冲在时间轴上\textbf{“展宽” (Spreading)}并出现\textbf{“拖尾” (Tailing)}。
    \par \textbf{结果(ISI):} 前一个码元的“拖尾”延伸到了当前码元的时隙内,叠加在当前信号上。若干扰过大,判决器会将 "0" 误判为 "1"(或反之),导致\textbf{误码率增大}。

    \tcbline

    \textbf{2. 本质辨析:同一物理过程,不同工程视角}
    \par “模拟信号的波形失真”和“数字信号的码间串扰”本质上是\textbf{同一物理过程}(线性系统对频谱的改变),只是关注的指标(KPI)不同:
    \begin{itemize}
        \item \textbf{模拟通信}关注\textbf{保真度 (Fidelity)}:
        \par 我们在乎波形“像不像”。高频丢失导致声音变“闷”、图像变“糊”,这直接破坏了信息本身,故称“失真”。
        \item \textbf{数字通信}关注\textbf{可分性 (Separability)}:
        \par 我们不在乎波形是否方正,只在乎\textbf{抽样时刻}能否分清 0 和 1。
        \par 只有当失真导致波形\textbf{展宽}并干扰到相邻码元(即“拖尾”打架)时,才构成问题。因此,我们特意用\textbf{码间串扰 (ISI)} 来强调这种相邻脉冲重叠的工程后果。
    \end{itemize}
\end{examplebox}