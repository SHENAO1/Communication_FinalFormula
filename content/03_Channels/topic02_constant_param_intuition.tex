% content/03_Channels/topic02_constant_param_intuition.tex

% 第一部分保持不变,已经是绿色盒子
\begin{intuitionbox}{1. 核心概念辨析:频率、角频率与速度}
    \textbf{1. 三个易混淆概念的物理本质}
    \begin{itemize}
        \item \textbf{频率 ($f$)}:描述\textbf{“重复的快慢”} (Hz)。
        \item \textbf{角频率 ($\omega$)}:描述\textbf{“旋转的角度速度”} (rad/s)。$\omega = 2\pi f$。
        \item \textbf{波速 ($v$) vs 群迟延 ($t_d$)}:
        \par \textbf{误区}:“频率越高,速度越快”。\textbf{错!}在同一介质中,波速 $v$ 通常与频率无关。
    \end{itemize}

    \tcbline

    \textbf{2. “为了保持相同的时间延迟”的理解}
    \begin{itemize}
        \item 既然速度相同,走过相同的距离,\textbf{耗时(群迟延 $t_d$)必然相同}。
        \item 公式:$\text{相位偏移} (\varphi) = \text{角速度}(\omega) \times \text{时间}(t_d)$。
        \item 因为 $t_d$ 固定(大家都跑1秒):
        \begin{itemize}
            \item \textbf{低频} ($\omega$ 小) $\to$ 跑的圈数少 $\to$ \textbf{相位 $\varphi$ 小}。
            \item \textbf{高频} ($\omega$ 大) $\to$ 跑的圈数多 $\to$ \textbf{相位 $\varphi$ 大}。
        \end{itemize}
    \end{itemize}
\end{intuitionbox}

% --- 修改点:将 kbox 改为 intuitionbox ---
\begin{intuitionbox}{2. 核心深度解析:群时延与相位的物理本质}
    \textbf{1. 深度辨析:为什么说相位不是“尺子”?}
    \begin{itemize}
        \item \textbf{尺子 (绝对性)}:刻度统一。1米永远是1米,1秒永远是1秒。
        \item \textbf{相位 (相对性)}:相位的“时间刻度”是弹性的,不能仅看读数。
        \begin{itemize}
            \item 对于低频 ($1\text{Hz}$):转半圈 ($180^\circ$) 代表走了 \textbf{0.5秒}。
            \item 对于高频 ($100\text{Hz}$):转半圈 ($180^\circ$) 仅代表走了 \textbf{0.005秒}。
        \end{itemize}
        \item \textit{结论:相位标尺随频率变化,不能仅凭“度数大小”来判断时间长短。}
    \end{itemize}

    \tcbline

    \textbf{2. 本质回归:相位是“圈数” }
    \par 相位本质是计数器:$\text{周期数} = \varphi / 2\pi$。
    \par \textbf{场景}:大家都要经历 \textbf{1秒} ($t_d$, 时间同步)。
    \begin{itemize}
        \item \textbf{慢车 (低频)}:车轮转得慢。1秒钟只够它开 \textbf{1圈} ($\varphi=2\pi$)。
        \item \textbf{快车 (高频)}:车轮转得快。1秒钟足够它开 \textbf{100圈} ($\varphi=200\pi$)。
    \end{itemize}
    \textit{结论:高频分量相位大,不是因为它用的时间长,而是因为它“车轮转得快”,必须在相同时间里积累更多的“圈数”。}

    \tcbline

    \textbf{3. 视觉类比:螺旋弹簧}
    \par 想象将两根弹簧拉长到同样的 \textbf{1米} (代表时间同步 $t_d=1\text{s}$):
    \begin{itemize}
        \item \textbf{弹簧A (低频,疏松)}:绕得很松。拉到1米长,它可能只绕了 \textbf{2圈}。
        \item \textbf{弹簧B (高频,紧密)}:绕得很密。拉到1米长,它可能绕了 \textbf{200圈}。
    \end{itemize}
    \textbf{对应关系}:
    \begin{itemize}
        \item \textbf{1米} = \textbf{绝对时间 $t_d$} (这是尺子,大家都一样长)。
        \item \textbf{圈数} = \textbf{相位 $\varphi$} (这是结构密度,随频率变化)。
    \end{itemize}
    \textit{警示:不要因为弹簧B的圈数多100倍,就误以为它比A长100倍。它们其实一样长!}

    \tcbline

    \textbf{4. 另一种视角:钟表的“秒针 vs 分针”}
    \par 设想两根指针都需要经历相同的\textbf{物理时间 (1分钟)}:
    \begin{itemize}
        \item \textbf{分针 (低频)}:转速慢。只挪动一小格 ($6^\circ$) $\to$ 相位小。
        \item \textbf{秒针 (高频)}:转速快。必须扫过整整一圈 ($360^\circ$) $\to$ 相位大。
    \end{itemize}
\end{intuitionbox}