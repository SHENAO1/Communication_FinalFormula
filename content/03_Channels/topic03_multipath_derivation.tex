% content/03_Channels/topic03_multipath_derivation.tex

% --- 深度解析:多径效应的机理 ---

\begin{intuitionbox}{深度解析:两径模型与频率选择性衰落}
    为了深入理解频率选择性衰落,我们建立一个具体的\textbf{两径信道模型}进行推导。
    
    \textbf{1. 模型建立}:
    设发射信号为 $f(t)$,两条路径衰减均为 $K$,时延分别为 $\tau_1, \tau_2$。
    接收信号 $f_o(t)$ 为:
    \[ f_o(t) = K f(t - \tau_1) + K f(t - \tau_2) \]
    
    \textbf{2. 频域传输函数推导}:
    令相对时延差 \textcolor{alertred}{$\tau = \tau_2 - \tau_1$}。
    对 $f_o(t)$ 做傅里叶变换:
    \begin{align*}
        F_o(\omega) &= K F(\omega)e^{-j\omega\tau_1} + K F(\omega)e^{-j\omega\tau_2} \\
                    &= K F(\omega)e^{-j\omega\tau_1} \left[ 1 + e^{-j\omega(\tau_2 - \tau_1)} \right]
    \end{align*}
    信道传输函数 $H(\omega) = F_o(\omega)/F(\omega)$:
    \begin{equation}
        H(\omega) = \underbrace{K e^{-j\omega\tau_1}}_{\text{固定时延/衰减}} \cdot \underbrace{(1 + e^{-j\omega\tau})}_{\text{\textcolor{alertred}{频率选择性项}}}
    \end{equation}

    \textbf{3. 幅频特性 $|H(\omega)|$}:
    利用欧拉公式 $|1+e^{-j\theta}| = |e^{-j\theta/2}(e^{j\theta/2}+e^{-j\theta/2})| [cite_start]= 2|\cos(\theta/2)|$ [cite: 10]:
    \begin{equation}
        |H(\omega)| = 2K \left| \cos\left( \frac{\omega\tau}{2} \right) \right|
    \end{equation}
    这表明衰减不仅与 $K$ 有关,还与\textbf{频率 $\omega$} 和\textbf{时延差 $\tau$} 密切相关。
\end{intuitionbox}

\begin{kbox}{多径信道幅频特性曲线与分析}
    根据 $|H(\omega)| [cite_start]= 2K |\cos(\omega\tau/2)|$ [cite: 12] 绘制幅频特性曲线:

    \begin{center}
    \begin{tikzpicture}
        \begin{axis}[
            width=\linewidth, height=5cm,
            axis lines=middle,
            xlabel={$\omega$}, ylabel={$|H(\omega)|$},
            xmin=0, xmax=13, ymin=0, ymax=2.3,
            xtick={3.14, 9.42},
            xticklabels={$\frac{\pi}{\tau}$, $\frac{3\pi}{\tau}$},
            ytick={2}, yticklabels={$2K$},
            every axis x label/.style={at={(current axis.right of origin)},anchor=north west},
            label style={font=\footnotesize},
            tick label style={font=\footnotesize}
        ]
        
        % 1. 绘制余弦模值曲线
        \addplot[thick, mainblue, domain=0:12.5, samples=100] {2*abs(cos(deg(x/2)))};
        
        % 2. 绘制传输零点 (红点)
        \node[circle,fill=alertred,inner sep=1.5pt] at (axis cs:3.14,0) {};
        \node[circle,fill=alertred,inner sep=1.5pt] at (axis cs:9.42,0) {};
        
        % 3. 绘制垂直虚线
        \draw[dashed, alertred] (axis cs:3.14, 0) -- (axis cs:3.14, 1.2);
        \draw[dashed, alertred] (axis cs:9.42, 0) -- (axis cs:9.42, 1.2);

        % 4. 绘制相关带宽水平双向箭头
        \draw[<->, alertred, thick] (axis cs:3.14, 1.2) -- (axis cs:9.42, 1.2) node[midway, above, font=\scriptsize] {$\Delta f = 1/\tau$};
    \end{axis}
    \end{tikzpicture}
    \end{center}

    \begin{itemize}
        \item \textbf{频率选择性衰落}:信道对不同频率成分衰减不同。
        \item \textbf{传输零点}:当 $\omega = (2n+1)\pi/\tau$ 时,信号完全无法通过。
        \item \textbf{相关带宽} $\Delta f = 1/\tau_m$:相邻传输零点的频率间隔。
    \end{itemize}
\end{kbox}

\begin{kbox}{抗频率选择性衰落的措施}
    为了使信号基本不受多径影响(即让信号频谱落在信道特性的平坦区域),需满足以下条件:

    \tcbline
    
    \textbf{1. 频域条件:限制信号带宽}
    \begin{itemize}
        \item 要求信号带宽 $B_s$ \textbf{远小于} 信道相关带宽 $\Delta f$。
        \item \textbf{工程经验公式}:
        \[ \boxed{ B_s = \left( \frac{1}{3} \sim \frac{1}{5} \right) \Delta f } \]
    \end{itemize}

    \tcbline

    \textbf{2. 时域条件:限制码元速率}
    多径效应在时域表现为\textbf{码间串扰 (ISI)}。
    \begin{itemize}
        \item 为减小 ISI,数字信号码元宽度 $T_s$ 应远大于最大多径时延 $\tau_m$。
        \item \textbf{工程经验公式}:
        \[ \boxed{ T_s = (3 \sim 5) \tau_m } \]
        \item \textbf{推论}:$T_s \uparrow \Rightarrow$ 码元速率 $R_B \downarrow \Rightarrow$ 信号带宽 $B_s \downarrow \Rightarrow$ 多径影响 $\downarrow$。
    \end{itemize}
\end{kbox}

\begin{intuitionbox}{核心辨析:为什么要选 $1/3 \sim 1/5$?如果不幸掉进“波谷”怎么办?}
    \textbf{Q: 如果信号带宽很窄 ($B_s \ll \Delta f$),但恰好落在信道的“谷底”(传输零点附近),信号岂不是衰减很严重?}
    
    \textbf{A: 是的,信号会严重衰减(深衰落),但这是“两害相权取其轻”的选择。}

    \begin{center}
    \begin{tikzpicture}
        \begin{axis}[
            width=\linewidth, height=6.5cm, % [微调] 稍微增加高度以适应更高的纵轴
            axis lines=middle, 
            xtick=\empty, ytick=\empty,
            xlabel={$\omega$}, ylabel={$|H(\omega)|$},
            xmin=0, xmax=13, 
            ymin=0, ymax=4.0, % [修改] 纵轴上限拉高到 4.0,给文字留足空间
            every axis y label/.style={at={(current axis.above origin)},anchor=south},
            every axis x label/.style={at={(current axis.right of origin)},anchor=north west},
            clip=false
        ]
        % 背景信道曲线
        \addplot[thick, gray!50, domain=0:12.5, samples=100] {2*abs(cos(deg(x/2)))};
        
        % 情况1:宽带信号(红色区域)
        \fill[red, opacity=0.3] (axis cs: 1.5, 0) rectangle (axis cs: 4.7, 2);
        
        % [修改] 红色文字:上移至 2.9 (比之前的 2.4 高,比 3.3 低)
        % 配合 ymax=4.0,这里有充足的空间,既不压框也不撞轴
        \node[red, align=center, font=\scriptsize] at (axis cs: 3.14, 2.9) {\textbf{宽带信号} ($B_s > \Delta f$)\\ 部分强、部分弱\\ $\downarrow$\\ \textbf{波形畸变 (ISI)}};
        
        % 情况2:窄带信号(落在波峰)
        \fill[mainblue, opacity=0.6] (axis cs: 6.0, 0) rectangle (axis cs: 6.6, 2);
        
        % 蓝色文字
        \node[mainblue, font=\scriptsize] at (axis cs: 6.28, 2.3) {窄带(波峰)};

        % 情况3:窄带信号(落在波谷)
        \fill[mainblue, opacity=0.6] (axis cs: 9.1, 0) rectangle (axis cs: 9.7, 0.4);
        \draw[->, mainblue] (axis cs: 10.5, 0.8) -- (axis cs: 9.7, 0.4);
        \node[mainblue, align=center, font=\scriptsize, anchor=west] at (axis cs: 10.0, 1.2) {\textbf{窄带(波谷)}\\ 整体变弱\\ 但\textbf{无畸变}};
        \end{axis}
    \end{tikzpicture}
    \end{center}

    \textbf{原理分析}:
    \begin{itemize}
        \item \textbf{现状 A (宽带信号)}:如果不限制带宽,信号频谱一部分在波峰,一部分在波谷。接收到的频谱形状被“扭曲”,时域上产生严重的\textbf{码间串扰 (ISI)}。这种非线性畸变极难修复。
        \item \textbf{现状 B (窄带信号)}:如果我们限制 $B_s < \Delta f$:
        \begin{itemize}
            \item \textbf{好处}:信号在通带内受到的衰落是“均匀”的(即\textbf{平坦衰落}),波形不畸变,无 ISI。
            \item \textbf{代价}:如果落在谷底,能量会整体变弱。
        \end{itemize}
        \item \textbf{解决方案}:工程上,“能量弱”比“波形畸变”好解决。
        \begin{itemize}
            \item 使用 \textbf{AGC (自动增益控制)} 放大信号。
            \item 使用 \textbf{分集接收 (Diversity)}:这根天线在谷底,另一根天线可能就在波峰。
        \end{itemize}
    \end{itemize}
\end{intuitionbox}


\begin{intuitionbox}{深度辨析:为什么“降低速率”能减轻多径效应?}
    这是一个非常经典的通信原理问题。简单来说,降低码元速率之所以能减轻多径效应,是因为它让信号在\textbf{时域上“变长”}了,在\textbf{频域上“变窄”}了。
    
    我们将从频域和时域两个角度来解析这个逻辑链条:

    \tcbline

    \textbf{1. 为什么降低码元速率,带宽会随之减小?}
    
    这是信号处理的一个基本物理规律:\textbf{时域和频域是反比关系}。
    \begin{itemize}
        \item \textbf{码元速率 ($R_B$)}:每秒传输符号的个数。
        \item \textbf{码元宽度 ($T_s$)}:$T_s = 1 / R_B$。
        \item \textbf{信号带宽 ($B$)}:根据傅里叶变换原理,脉冲信号带宽与持续时间成反比,即 $B \approx 1 / T_s$。
    \end{itemize}
    \textbf{结论}:降低速率 $R_B \downarrow$ $\Rightarrow$ 码元变长 $T_s \uparrow$ $\Rightarrow$ \textbf{带宽变窄 $B \downarrow$}。

    \tcbline

    \textbf{2. 为什么带宽减小(或码元变长)能减轻多径效应?}

    \textbf{A. 频域解释(躲避深坑)}
    \begin{itemize}
        \item \textbf{宽带信号}:带宽 > 相关带宽 ($\Delta f$)。信号频谱很宽,容易跨越信道的“深坑”(频率选择性衰落),导致波形严重失真。
        \item \textbf{窄带信号}:带宽 < 相关带宽 ($\Delta f$)。整个信号频谱可以“躲”在信道频率响应平坦的区域。虽然可能整体衰减(平坦衰落),但\textbf{没有频率畸变},ISI 很小。
    \end{itemize}

    \textbf{B. 时域解释(直观演示:回声变成“边角料”)}
    
    假设环境决定的多径最大时延为固定值 $\tau_m$。多径效应表现为:\textbf{上一码元的“回声”延迟到达,干扰了当前码元}。
    
    \begin{center}
    \begin{tikzpicture}[>=Stealth, scale=0.9, transform shape, font=\footnotesize]
        
        % =========================================
        % Case 1: 高速率 (短码元)
        % =========================================
        \node[anchor=west, text=alertred, font=\bfseries] at (-0.5, 3.7) {Case 1: 高速率 (短码元 $T_s \approx \tau_m$)};

        % 坐标定义
        \def\yBase{1.0}
        
        % 时间轴
        \draw[->, thick, gray] (-0.2, 0.8) -- (6.0, 0.8) node[right] {$t$};

        % 1. 当前码元 (主径) - 蓝色
        \filldraw[fill=mainblue!15, draw=mainblue, thick] (0, \yBase) rectangle (2.0, \yBase+1.0);
        \node[mainblue] at (1.0, \yBase+0.5) {当前码元};
        % Ts 辅助线
        \draw[dashed, mainblue] (0, \yBase) -- (0, 0.4);

        % 2. 上一码元的回声 (多径) - 灰色虚线
        \filldraw[fill=gray!20, draw=gray, dashed, opacity=0.8] (1.5, \yBase+0.2) rectangle (3.5, \yBase+1.2);
        
        % 回声说明
        \node[text=gray!90, font=\scriptsize, align=center] (echo1) at (3.5, 2.8) {上一码元\\的延迟回声};
        \draw[->, gray!90, thin] (echo1.south) -- (2.5, \yBase+1.2);

        % 3. 干扰区域 (ISI) - 红色斜线
        \fill[pattern=north east lines, pattern color=alertred] (1.5, \yBase) rectangle (2.0, \yBase+1.0);
        \draw[alertred, thick] (1.5, \yBase) rectangle (2.0, \yBase+1.0);

        % 4. 延迟标注
        \draw[->, thick, alertred] (0, 0.5) -- (1.5, 0.5) node[midway, below] {延迟 $\tau_m$};
        \draw[dashed, gray] (1.5, \yBase+0.2) -- (1.5, 0.4);

        % 5. ISI 说明
        \node[align=left, anchor=west] (desc1) at (3.3, 1.5) {
            \textbf{\textcolor{alertred}{严重 ISI}} \\
            \scriptsize 回声覆盖大部分区域,\\
            \scriptsize 导致严重判决错误。
        };
        \draw[->, alertred, thick, shorten >=2pt] (desc1.west) to[out=180, in=0] (2.0, 1.5);


        % =========================================
        % Case 2: 低速率 (长码元) —— 修改版
        % =========================================
        \begin{scope}[yshift=-4.0cm]
        
        \node[anchor=west, text=textgreen, font=\bfseries] at (-0.5, 3.2) {Case 2: 低速率 (长码元 $T_s \gg \tau_m$)};
        
        % 时间轴 (拉长以容纳更长的图示)
        \draw[->, thick, gray] (-0.2, 0.8) -- (7.5, 0.8) node[right] {$t$};

        % 1. 当前长码元 (主径) - 蓝色
        % [修改] 使用锯齿边效果表示“后面还有很长”
        % 绘制左、上、下三边
        \fill[mainblue!15] (0, \yBase) -- (6.5, \yBase) -- (6.5, \yBase+1.0) -- (0, \yBase+1.0) -- cycle;
        \draw[thick, mainblue] (0, \yBase) -- (6.5, \yBase);       % 下边
        \draw[thick, mainblue] (0, \yBase+1.0) -- (6.5, \yBase+1.0); % 上边
        \draw[thick, mainblue] (0, \yBase) -- (0, \yBase+1.0);       % 左边
        
        % [新增] 右侧绘制锯齿波 (代表截断/未完待续)
        \draw[thick, mainblue] (6.5, \yBase) -- (6.3, \yBase+0.2) -- (6.5, \yBase+0.4) -- (6.3, \yBase+0.6) -- (6.5, \yBase+0.8) -- (6.3, \yBase+1.0) -- (0, \yBase+1.0);
        % 注意:上面的 fill 已经覆盖了底色,这里只画边框线增加视觉效果

        % 辅助虚线
        \draw[dashed, mainblue] (0, \yBase) -- (0, 0.4);

        % 文字说明
        \node[mainblue] at (5.0, \yBase+0.5) {当前长码元 (未画完 $\cdots$)}; 

        % 2. 上一码元的回声 (多径)
        % [修改] 保持原有大小,对比之下显得很小
        \filldraw[fill=gray!20, draw=gray, dashed, opacity=0.8] (1.5, \yBase+0.2) rectangle (3.5, \yBase+1.2);
        
        % 回声说明
        \node[text=gray!90, font=\scriptsize] (echo2) at (2.5, 2.8) {上一码元回声};
        \draw[->, gray!90, thin] (echo2.south) -- (2.5, \yBase+1.2);

        % 3. 干扰区域 (ISI)
        % 这一块的大小是由“延迟”决定的,是物理客观存在的,不能变小,但放在巨大的蓝色背景下就显得小了
        \fill[pattern=north east lines, pattern color=alertred] (1.5, \yBase) rectangle (3.5, \yBase+1.0);
        \draw[alertred, thick] (1.5, \yBase) rectangle (3.5, \yBase+1.0);

        % 4. 延迟标注
        \draw[->, thick, alertred] (0, 0.5) -- (1.5, 0.5) node[midway, below] {延迟 $\tau_m$};
        \draw[dashed, gray] (1.5, \yBase+0.2) -- (1.5, 0.4);

        % 5. ISI 说明
        \node[align=left, anchor=north west] (desc2) at (2.5, 0.2) {
            \textbf{\textcolor{textgreen}{轻微 ISI}} \\
            \scriptsize 干扰仅占极小比例,\\ 
            \scriptsize 主体纯净且漫长。
        };
        \draw[->, textgreen, thick, shorten >=2pt] (desc2.north) -- (2.5, 1.0);

        \end{scope}

    \end{tikzpicture}
    \end{center}
\end{intuitionbox}