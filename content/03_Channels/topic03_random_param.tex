% content/03_Channels/topic03_random_param.tex

% --- 第一部分:随参信道特性与信号模型推导 ---
\begin{kbox}{3. 随参信道特性与模型推导}
    \textbf{随参信道定义}:传输特性随时间快速变化的信道。
    \begin{itemize}
        \item \textbf{三大特性}:
        \begin{enumerate}
            \item 衰减随时间变化 $a(t)$
            \item 时延随时间变化 $\tau(t)$
            \item 多径传播 (Multipath Propagation)
        \end{enumerate}
    \end{itemize}

    \tcbline

    \textbf{信号模型推导}:
    设发送信号为单频正弦波 $S(t) = A \cos(\omega_c t)$。
    经 $n$ 条路径传播,第 $i$ 条路径的衰减为 $a_i(t)$,时延为 $\tau_i(t)$。
    
    接收信号 $r(t)$ 为:
    \begin{align*}
        r(t) &= \sum_{i=1}^{n} a_i(t) \cos \{ \omega_c [t - \tau_i(t)] \} \\
             &= \sum_{i=1}^{n} a_i(t) \cos [\omega_c t + \varphi_i(t)]
    \end{align*}
    其中 $\varphi_i(t) = -\omega_c \tau_i(t)$ 为第 $i$ 径的随机相位。

    \textbf{同相-正交分解} (利用三角展开):
    \begin{align*}
        r(t) &= \sum_{i=1}^{n} a_i(t) [\cos \varphi_i(t) \cos \omega_c t - \sin \varphi_i(t) \sin \omega_c t] \\
             &= \underbrace{\left[\sum_{i=1}^{n} a_i(t) \cos \varphi_i(t)\right]}_{X(t)} \cos \omega_c t \\
             &\quad - \underbrace{\left[\sum_{i=1}^{n} a_i(t) \sin \varphi_i(t)\right]}_{Y(t)} \sin \omega_c t
    \end{align*}
    
    \textbf{合成包络与相位形式}:
    \[ r(t) = X(t) \cos \omega_c t - Y(t) \sin \omega_c t = V(t) \cos[\omega_c t + \varphi(t)] \]
    其中:
    \begin{itemize}
        \item $V(t) = \sqrt{X^2(t) + Y^2(t)}$ (合成包络)
        \item $\varphi(t) = \arctan[Y(t)/X(t)]$ (合成相位)
    \end{itemize}
\end{kbox}

% --- 第二部分:统计特性 (笔记中的红色重点) ---
\begin{kbox}{4. 接收信号的统计特性}
    接收信号是包络、相位随机缓慢变化的\textbf{窄带信号}。
    
    \textbf{中心极限定理分析}:
    \begin{itemize}
        \item 当路径数 $n$ 足够大时,由中心极限定理,同相分量 $X(t)$ 和正交分量 $Y(t)$ 均趋于\textbf{高斯分布 (正态分布)}。
    \end{itemize}

    \tcbline
    
    \textbf{结论}:
    \begin{itemize}
        \item \textbf{包络 $V(t)$}:服从 \textbf{瑞利分布 (Rayleigh)}。
        \[ f(v) = \frac{v}{\sigma^2} e^{-\frac{v^2}{2\sigma^2}}, \quad v \geq 0 \]
        \item \textbf{相位 $\varphi(t)$}:服从 \textbf{均匀分布 (Uniform)}。
        \[ f(\varphi) = \frac{1}{2\pi}, \quad 0 \le \varphi \le 2\pi \]
    \end{itemize}
    \textit{注:若存在视距分量(直射波),包络服从莱斯分布 (Rician)。}
\end{kbox}

% --- 第三部分:频率选择性衰落 (原文件内容) ---
\begin{kbox}{5. 频率选择性衰落}
    由于多径效应,不同频率分量的衰落不同。
    
    \begin{itemize}
        \item \textbf{相关带宽}:
        \[ \Delta f \approx \frac{1}{\tau_m} \]
        其中 $\tau_m$ 为最大多径时延差。
        
        \item \textbf{工程防衰落经验公式}:
        为使信号传输不发生严重的频率选择性衰落,需满足:
        \begin{itemize}
            \item 信号带宽:$B_s \leq (\frac{1}{3} \sim \frac{1}{5}) \Delta f$
            \item 码元宽度:$T_s \geq (3 \sim 5) \tau_m$
        \end{itemize}
    \end{itemize}
\end{kbox}