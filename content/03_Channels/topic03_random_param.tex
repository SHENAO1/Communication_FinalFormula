% content/03_Channels/topic03_random_param.tex

% --- 第一部分:随参信道特性与信号模型推导 ---
\begin{kbox}{3. 随参信道特性与信号模型}
    \textbf{随参信道定义}:传输特性随时间快速变化的信道(如短波电离层反射、移动通信信道)。
    \begin{itemize}
        \item \textbf{三大特性}:
        \begin{enumerate}
            \item 衰减随时间变化 $a(t)$
            \item 时延随时间变化 $\tau(t)$
            \item \textbf{多径传播 (Multipath)}:最核心特征。
        \end{enumerate}
    \end{itemize}

    \tcbline

    \textbf{多径信号模型}:
    设发送单频信号 $A \cos(\omega_c t)$,经 $n$ 条路径传输。
    接收信号 $r(t)$ 可表示为窄带随机过程:
    \begin{equation}
        r(t) = V(t) \cos[\omega_c t + \varphi(t)]
    \end{equation}
    其中 $V(t)$ 为随机包络,$\varphi(t)$ 为随机相位。
\end{kbox}

% --- 第二部分:统计特性 ---
\begin{kbox}{4. 接收信号的统计特性}
    根据中心极限定理,当路径数 $n$ 足够大时:
    
    \begin{itemize}
        \item \textbf{包络 $V(t)$}:服从 \textbf{瑞利分布 (Rayleigh)}。
        \[ f(v) = \frac{v}{\sigma^2} e^{-\frac{v^2}{2\sigma^2}}, \quad v \geq 0 \]
        \textit{注:若存在视距分量(LOS),则服从莱斯分布 (Rician)。}
        
        \item \textbf{相位 $\varphi(t)$}:服从 \textbf{均匀分布} $(0, 2\pi)$。
    \end{itemize}
\end{kbox}

% --- 原来的"第5部分"已被移除,由更详细的 derivation 文件替代 ---