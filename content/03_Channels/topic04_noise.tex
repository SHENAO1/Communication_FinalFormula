\begin{kbox}{4. 信道噪声与等效带宽}
    \textbf{1. 高斯白噪声统计特性}:
    \begin{itemize}
        \item 双边 PSD:$P_n(f) = \frac{n_0}{2}$ (W/Hz)
        \item 自相关:$R_n(\tau) = \frac{n_0}{2} \delta(\tau)$
        \item 一维 PDF (正态分布,均值为0):
        \[ f_n(v) = \frac{1}{\sqrt{2\pi}\sigma_n} \exp\left(-\frac{v^2}{2\sigma_n^2}\right) \]
    \end{itemize}
    
    \tcbline
    
    \textbf{2. 噪声等效带宽 ($B_n$)}:
    
    \begin{center}
    \begin{tikzpicture}[scale=0.85, >=stealth]
        % 定义坐标轴
        \draw[->] (-3.5,0) -- (3.5,0) node[right] {$f$};
        \draw[->] (0,0) -- (0,3) node[left] {$P_n(f)$};
        
        % 定义参数
        \def\fzero{2.0}
        \def\peak{2.0}
        \def\width{0.6} % 视觉上的半宽
        
        % 绘制实际噪声功率谱 (钟形曲线) - 右侧
        \draw[thick, mainblue] plot[domain=0.5:3.5, samples=50] (\x, {\peak * exp(-(\x-\fzero)^2 / (2*\width^2))});
        % 左侧
        \draw[thick, mainblue] plot[domain=-3.5:-0.5, samples=50] (\x, {\peak * exp(-(\x+\fzero)^2 / (2*\width^2))});
        
        % 绘制等效矩形 (虚线) - 右侧
        \draw[thick, dashed, alertred] (\fzero - 0.75, 0) -- (\fzero - 0.75, \peak) -- (\fzero + 0.75, \peak) -- (\fzero + 0.75, 0);
        % 左侧
        \draw[thick, dashed, alertred] (-\fzero - 0.75, 0) -- (-\fzero - 0.75, \peak) -- (-\fzero + 0.75, \peak) -- (-\fzero + 0.75, 0);
        
        % 标注
        \node[below] at (\fzero, 0) {$f_0$};
        \node[below] at (-\fzero, 0) {$-f_0$};
        \node[below left] at (0,0) {0};
        
        % 标注中心高度 Pn(f0)
        \draw[dotted] (\fzero, \peak) -- (0, \peak) node[left, font=\small] {$P_n(f_0)$};
        
        % 标注带宽 Bn
        \draw[<->] (\fzero - 0.75, \peak + 0.2) -- (\fzero + 0.75, \peak + 0.2) node[midway, above, font=\small] {$B_n$};
        
        % 箭头指向
        \node[font=\footnotesize, align=center, mainblue] at (0, 1.2) {实际谱\\ $P_n(f)$};
        \node[font=\footnotesize, align=center, alertred] at (0, 0.5) {等效矩形};
        
    \end{tikzpicture}
    \end{center}

    \begin{itemize}
        \item \textbf{定义公式}:保持噪声总功率不变,将实际滤波特性等效为理想矩形滤波器。
        \[
            B_n = \frac{\int_{-\infty}^{\infty} P_n(f) df}{2 P_n(f_0)} = \frac{\int_{0}^{\infty} P_n(f) df}{P_n(f_0)}
        \]
        \item \textbf{物理意义}:
        通过宽度为 $B_n$ 的理想矩形滤波器的噪声功率 \textbf{等于} 通过实际接收滤波器的噪声功率。
        \[ N = \int_{-\infty}^{\infty} P_n(f) df = \underbrace{2 \cdot B_n \cdot P_n(f_0)}_{\text{矩形面积}} \]
    \end{itemize}
\end{kbox}

% --- 新增部分:核心概念直观理解 (修正为垂直布局) ---

\begin{intuitionbox}{直观理解:为什么是 $0$ 均值与 $n_0/2$?}
    \textbf{1. 为什么总是研究 0 均值高斯噪声?}
    \begin{itemize}
        \item \textbf{分解视角}:任意非零均值噪声 $X(t)$ 均可分解为确定性直流分量 $\mu$ 与零均值随机分量 $N(t)$ 的叠加:
        \[ X(t) = \mu + N(t) \]
        \item \textbf{工程理由}:均值 $\mu$ 本质是直流偏置(DC Offset),不具备随机性,在接收端易于通过隔直电容或算法去除。通信系统核心对抗的是不可预测的随机波动(即方差),因此模型简化为仅研究 $N(t)$。
    \end{itemize}

    \tcbline

    \textbf{2. 为什么功率谱密度是 $n_0/2$?}
    
    $n_0$ (或 $N_0$) 代表\textbf{单边噪声功率谱密度}(物理热噪声 $N_0=kT$,其中 $k$ 为玻尔兹曼常数,系统温度为 $T$)。系数 $1/2$ 源于\textbf{物理实测}与\textbf{数学模型}为了保持\textbf{总功率守恒}所做的等效转换。

    \begin{center}
    \begin{tikzpicture}[scale=0.85, >=stealth]
        % --- 上图:物理单边谱 ---
        \begin{scope}[shift={(0, 3.2)}]
            % 坐标轴
            \draw[->] (-2.5,0) -- (2.5,0) node[right] {$f$};
            \draw[->] (0,0) -- (0,1.8) node[left] {$P_{single}$};
            
            % 图像 (单边)
            \fill[mainblue!20] (0,0) rectangle (2.0, 1.2);
            \draw[thick, mainblue] (0, 1.2) -- (2.0, 1.2);
            \draw[thick, mainblue] (0,0) -- (0,1.2); % y轴重合
            
            % 标注
            \node at (1.0, 0.6) {总功率 $P$};
            \node[left, mainblue] at (0, 1.2) {$n_0$};
            \node[right, mainblue, font=\small] at (1.5, 1.5) {\textbf{物理世界 (单边)}};
            \node[right, gray, font=\scriptsize] at (2.5, -0.3) {$f \ge 0$};
        \end{scope}

        % --- 转换箭头 ---
        % 从上图向下指
        \draw[->, thick, dashed, gray] (1.0, 2.9) -- (1.0, 1.9) node[midway, right, font=\scriptsize, color=black] {能量平分};
        % 为了直观,画出分叉效果
        \draw[->, thick, dashed, gray!50] (1.0, 1.9) -- (-1.0, 0.8);
        \draw[->, thick, dashed, gray!50] (1.0, 1.9) -- (1.0, 0.8);

        % --- 下图:数学双边谱 ---
        \begin{scope}[shift={(0, 0)}]
            % 坐标轴
            \draw[->] (-2.5,0) -- (2.5,0) node[right] {$f$};
            \draw[->] (0,0) -- (0,1.8) node[left] {$P_{double}$};
            
            % 负半轴
            \fill[alertred!20] (-2.0,0) rectangle (0, 0.6);
            \draw[thick, alertred] (-2.0, 0.6) -- (0, 0.6);
            
            % 正半轴
            \fill[alertred!20] (0,0) rectangle (2.0, 0.6);
            \draw[thick, alertred] (0, 0.6) -- (2.0, 0.6);
            
            % 标注
            \node at (-1.0, 0.3) {\footnotesize $P/2$};
            \node at (1.0, 0.3) {\footnotesize $P/2$};
            \node[left, alertred] at (0, 1) {$\frac{n_0}{2}$};
            \node[right, alertred, font=\small] at (1.5, 0.9) {\textbf{数学模型 (双边)}};
            \node[right, gray, font=\scriptsize] at (2.5, -0.3) {$f \in (-\infty, +\infty)$};
        \end{scope}
    \end{tikzpicture}
    \end{center}

    \begin{itemize}
        \item \textbf{积分守恒}:数学上引入负频率后,为了保证积分算出的总功率(面积)与物理真实功率一致,必须将高度减半:
        \[ P_{total} = \int_{0}^{\infty} n_0 \, df = \int_{-\infty}^{\infty} \frac{n_0}{2} \, df \]
    \end{itemize}
\end{intuitionbox}