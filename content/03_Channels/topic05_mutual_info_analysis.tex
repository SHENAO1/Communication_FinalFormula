% content/03_Channels/topic05_mutual_info_analysis.tex

% --- 1. 直观理解 ---
\begin{intuitionbox}{1. 互信息的物理直观:不确定性的消除}
    \textbf{核心概念}:信息量 = 不确定度的减小量。
    \begin{itemize}
        \item \textbf{发送前}:对发送端 $X$ 的不确定度是 $H(X)$。
        \item \textbf{接收后}:收到符号 $y$ 后,对 $X$ 依然存在的不确定度是 $H(X|Y)$(条件熵)。
        \item \textbf{获得的信息}(不确定性的消除):
        \[
            \begin{aligned}
            I(X;Y) &= \text{发送前的不确定度} - \text{接收后的不确定度} \\
                   &= H(X) - H(X|Y)
            \end{aligned}
        \]
    \end{itemize}
\end{intuitionbox}

% --- [新增] 互信息 Venn 图 (最终优化版) ---
% 调整:缩小中间字体,拉近圆心距离使交集更宽
\begin{center}
\begin{tikzpicture}[thick, scale=1.0]
    % 1. 定义参数 
    \def\R{1.8}   % 半径继续稍微加大
    \def\D{2.2}   % 圆心间距 (保持这个比例,交集会比较饱满)
    
    % --- 2. 填充颜色 ---
    % 左圆 H(X)
    \fill[mainblue!10] (0,0) circle (\R);
    % 右圆 H(Y)
    \fill[alertred!10] (\D,0) circle (\R);
    
    % 中间交集 I(X;Y)
    \begin{scope}
        \clip (0,0) circle (\R);
        \fill[mainblue!40] (\D,0) circle (\R); 
    \end{scope}

    % --- 3. 绘制轮廓 ---
    \draw[mainblue, thick] (0,0) circle (\R);
    \draw[alertred, thick] (\D,0) circle (\R);

    % --- 4. 内部标注 (字体缩小优化) ---
    % 中间:互信息 [修改] 使用 \small 缩小主文字,\tiny 缩小注释
    \node[align=center] at (\D/2, 0) {\small $\bm{I(X;Y)}$\\[1pt]{\tiny 有用信息}};
    
    % 左边:疑义度 [修改] 位置向左微调 (-0.4) 以适应更大的圆
    \node[align=center] at (-0.4, 0) {$H(X|Y)$ \\[-1pt] {\tiny (损失/疑义度)}};
    
    % 右边:噪声熵 [修改] 位置向右微调
    \node[align=center] at (\D+0.4, 0) {$H(Y|X)$ \\[-1pt] {\tiny (噪声熵)}};

    % --- 5. 外部标注 (向两侧展开) ---
    
    % H(X) 标注:向左上方引出箭头
    \draw[->, thick, gray] (-0.5, \R-0.3) -- (-1.5, \R+0.5) node[above left, black, align=right] {$\bm{H(X)}$ \\[-2pt] \scriptsize (信源熵)};
    
    % H(Y) 标注:向右上方引出箭头
    \draw[->, thick, gray] (\D+0.5, \R-0.3) -- (\D+1.5, \R+0.5) node[above right, black, align=left] {$\bm{H(Y)}$ \\[-2pt] \scriptsize (接收熵)};
    
    % H(X,Y) 联合熵 (底部横跨)
    \draw[<->, thin, gray] (-\R, -\R-0.3) -- (\D+\R, -\R-0.3) node[midway, fill=white] {$H(X,Y)$ \scriptsize (联合熵)};

\end{tikzpicture}
\end{center}

% --- 2. 详细推导 ---
\begin{kbox}{2. 条件期望视角的详细推导 (Handwritten Note)}
    \small
    此部分采用\textbf{分层理解}的方法,解释 $E[I(x_i|y_j)]$ 是如何推导为 $H(X|Y)$ 的。
    
    \tcbline
    
    \textbf{第一步:理解期望的层次}
    
    $E(\cdot)$ 是求期望。其物理含义是:在 $Y=y_j$ 发生的概率下,求 $X=x_i$ 的相关期望。\textbf{我们可以先固定 $Y$,再对外层求平均。}
    
    \textbf{第二步:构建“方框”结构}
    
    整个期望可以拆解为外层对 $Y$ 的加权和内层对 $X$ 的计算:
    
    \[ \sum_{j} P(y_j) \times \tikzmarknode{boxnode}{\framebox[1.2em]{\vphantom{M}}} \]
    
    \begin{itemize}
        \item \textbf{方框 $\square$ 中的内容}:
        这是在已知 $Y=y_j$ (确定) 的情况下,关于 $X$ 的信息量期望。
        \[ \text{方框内容} = \sum_{i} P(x_i|y_j) \cdot \log_2 \frac{1}{P(x_i|y_j)} = H(X|Y=y_j) \]
    \end{itemize}

    \begin{center}
    \begin{tikzpicture}[>=Stealth, node distance=0.8cm]
        % 样式定义
        \tikzstyle{block} = [rectangle, rounded corners, align=center, font=\scriptsize, drop shadow, inner sep=4pt, fill=white]
        
        % 外层节点
        \node (outer) [block, draw=mainblue, fill=mainblue!5] {
            $\displaystyle \text{外层:} \sum_{j} P(y_j) \times \left[ \mathbf{\dots} \right]$
        };

        % 内层节点
        \node (inner) [block, draw=alertred, fill=alertred!5, below=of outer] {
            \textbf{内层方框 (固定 $y_j$)}\\[3pt]
            $\displaystyle \sum_{i} P(x_i|y_j) \log_2 \frac{1}{P(x_i|y_j)}$ \\[3pt]
            $= H(X|Y=y_j)$ \quad (特定条件下的熵)
        };

        % 连接线
        \draw[->, thick, mainblue] (outer) -- node[right, font=\tiny, text=gray] {代入计算} (inner);
    \end{tikzpicture}
    \end{center}

    \textbf{第三步:整合得出结论}
    
    将方框内容代回原式,即可得:
    \[
    \begin{aligned}
        E[I] &= \sum_{j} P(y_j) \left[ \sum_{i} P(x_i|y_j) \log_2 \frac{1}{P(x_i|y_j)} \right] \\
             &= \sum_{j} P(y_j) \cdot H(X|Y=y_j) \\
             &= H(X|Y) \quad (\text{平均条件熵})
    \end{aligned}
    \]
\end{kbox}

% --- 3. 严谨推导 ---
\begin{kbox}{3. 严谨推导:联合概率法 (Derivation)}
    \small
    \textit{这是前文基于联合概率定义的另一种推导路径。}
    
    \[ I(X;Y) = E[I(x_i)] - E[I(x_i|y_j)] \]
    
    对于第二项 $E[I(x_i|y_j)]$,利用数学期望定义(双重求和):
    \[
    \begin{aligned}
        E[I(x_i|y_j)] 
        &= \sum_{j=1}^m \sum_{i=1}^n \underbrace{P(x_i, y_j)}_{\text{联合概率}} \log_2 \frac{1}{P(x_i|y_j)} \\
        &\quad \text{\scriptsize (代入贝叶斯公式 $P(x_i, y_j) = P(y_j)P(x_i|y_j)$)} \\
        &= \sum_{j=1}^m P(y_j) \underbrace{\left[ \sum_{i=1}^n P(x_i|y_j) \log_2 \frac{1}{P(x_i|y_j)} \right]}_{H(X|Y=y_j)} \\
        &= \sum_{j=1}^m P(y_j) H(X|Y=y_j) \\
        &= H(X|Y)
    \end{aligned}
    \]
    \textbf{最终结论}:$I(X;Y) = H(X) - H(X|Y)$
\end{kbox}

% --- 4. 对称性 ---
\begin{kbox}{4. 互信息的对称性 (Symmetry)}
    \[
    \begin{aligned}
        I(X;Y) &= H(X) - H(X|Y) \quad (\text{信源熵} - \text{信道疑义度}) \\
               &= H(Y) - H(Y|X) \quad (\text{接收熵} - \text{噪声熵})
    \end{aligned}
    \]
\end{kbox}