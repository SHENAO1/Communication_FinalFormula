% content/03_Channels/topic05_capacity.tex

\begin{kbox}{5. 离散信道容量 (Discrete Channel Capacity)}
    
    \textbf{1. 互信息与熵的关系}
    \begin{itemize}
        \item \textbf{发送熵 (Source Entropy)}:$H(x) = -\sum_{i} P(x_i)\log_2 P(x_i)$
        \item \textbf{损失熵 (Conditional Entropy)}:因噪声损失的信息量。
        \[ H(x/y) = -\sum_{j}\sum_{i} P(y_j)P(x_i/y_j)\log_2 P(x_i/y_j) \]
        \item \textbf{平均互信息}:接收端实际获得的平均信息量。
        \[ I(X;Y) = H(x) - H(x/y) \quad (\text{b/符号}) \]
    \end{itemize}

    \tcbline

    \textbf{2. 传输速率与信道容量}
    
    % [修改] 调整 minipage 宽度比例,防止溢出,并设置 [c] 垂直居中
    \begin{minipage}[c]{0.55\linewidth}
        \textbf{信息传输速率 $R$} (b/s):
        \[ R = r \cdot [H(x) - H(x/y)] \]
        \footnotesize{其中 $r$ 为符号速率 (符号/秒)。}
    \end{minipage}
    \hfill
    \begin{minipage}[c]{0.40\linewidth}
        \centering
        % [修改] 减小 TikZ 的水平间距,使其更紧凑
        \begin{tikzpicture}[scale=0.65, >=Stealth]
            % 定义间距变量
            \def\hdist{1.8} % 水平间距 (减小)
            \def\vdist{0.8} % 垂直间距

            % 左侧节点 (发送)
            \foreach \i in {1,2,3} {
                \coordinate (x\i) at (0, {-\i*\vdist});
                \filldraw[mainblue] (x\i) circle (2pt);
            }
            \node[left] at (x1) {\scriptsize $x_1$};
            \node[left] at (x2) {\scriptsize $x_2$};
            \node[left] at (x3) {\scriptsize $x_3$};
            \node[left=0.35cm] at (x2) {\scriptsize $P(x)$};

            % 右侧节点 (接收)
            \foreach \j in {1,2,3} {
                \coordinate (y\j) at (\hdist, {-\j*\vdist});
                \filldraw[alertred] (y\j) circle (2pt);
            }
            \node[right] at (y1) {\scriptsize $y_1$};
            \node[right] at (y2) {\scriptsize $y_2$};
            \node[right] at (y3) {\scriptsize $y_3$};
            % P(y) 标签省略或放近一点以节省空间

            % 连线 (示意图)
            \draw[->, gray!50] (x1) -- (y1);
            \draw[->, gray!50] (x1) -- (y2);
            \draw[->, gray!50] (x2) -- (y2);
            \draw[->, gray!50] (x3) -- (y2);
            \draw[->, gray!50] (x3) -- (y3);
            
            % 顶部标签
            \node[font=\tiny, gray] at (\hdist/2, -0.4) {$P(y|x)$};
        \end{tikzpicture}
    \end{minipage}

    \vspace{5pt}
    \textbf{信道容量 $C$}:指\textbf{信息传输速率}(或平均互信息)在\textbf{一切可能的信源概率分布 $P(x)$} 下的\textbf{最大值}。
    
    \begin{enumerate}
        \item \textbf{每秒容量 ($C_t$)}:最大信息传输速率。
        \[ C_t = \max_{P(x)} \{ R \} = \max_{P(x)} \{ r[H(x) - H(x/y)] \} \quad (\text{b/s}) \]
        \item \textbf{每符号容量 ($C$)}:每个符号能传输的最大平均信息量。
        \[ C = \max_{P(x)} [H(x) - H(x/y)] \quad (\text{b/符号}) \]
    \end{enumerate}
    \small{\textit{*若信道每秒传输符号数 $r$ 已知,则 $C_t = r \cdot C$。}}

\end{kbox}

\begin{kbox}{6. 连续信道容量 (Shannon Formula)}
    \textbf{香农公式}:即使在有噪声的信道中,只要传输速率小于 $C$,理论上就存在无误码传输。
    \[ C = B \log_2 \left( 1 + \frac{S}{N} \right) = B \log_2 \left( 1 + \frac{S}{n_0 B} \right) \quad (\text{b/s}) \]
    \begin{itemize}
        \item $B$:带宽 (Hz) \quad $S/N$:信噪比 (线性值)
        \item $n_0$:噪声单边功率谱密度 (W/Hz)
    \end{itemize}
    
    \tcbline
    
    \textbf{重要极限}:
    \[ \lim_{B \to \infty} C \approx 1.44 \frac{S}{n_0} \]
    这意味着:
    1. 增加带宽 $B$ 可以增加容量,但不能无限增加 (受限于 $S/n_0$)。
    2. 增加信号功率 $S$ 可以无上限地增加容量。
\end{kbox}