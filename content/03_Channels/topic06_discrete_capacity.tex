% file: content/03_Channels/topic06_discrete_capacity.tex

\begin{kbox}{5. 离散信道容量基础}
    
    \textbf{1. 互信息的两种物理视角}
    根据熵的定义,互信息 $I(X;Y)$ 可以从两个角度理解:
    
    % 使用直观理解盒子强调笔记中的概念
    \begin{intuitionbox}{辨析:损失熵 vs 噪声熵}
        \begin{itemize}
            \item \textbf{观察输入端 ($H(X) - H(X|Y)$)}:
            \begin{itemize}
                \item $H(X)$:信源原本的信息量(发送的干货)。
                \item \textbf{$H(X|Y)$ (损失熵/疑义度)}:接收到 $Y$ 后,对 $X$ 仍然存在的不确定性。代表在传输过程中\textbf{损失}掉的信息。
            \end{itemize}
            
            \item \textbf{观察输出端 ($H(Y) - H(Y|X)$)}:
            \begin{itemize}
                \item $H(Y)$:接收端收到的总信息量(包含干货和噪声)。
                \item \textbf{$H(Y|X)$ (噪声熵)}:给定发送 $X$ 的情况下,$Y$ 的不确定性。这完全是由信道\textbf{噪声}引起的随机性。
            \end{itemize}
        \end{itemize}
    \end{intuitionbox}

    % [修复] 使用 align* 环境并拆分长公式,防止溢出
    \textbf{2. 深入理解:逻辑连接与实际意义}
    
    \textbf{(1) 为什么两者相等?(快递包裹类比)}
    虽然发送和接收的总熵不同,但扣除各自的无效成分后,剩余部分守恒:
    
    \begin{align*}
        \text{互信息} &= \underbrace{\text{发送总量} - \text{路上丢件}}_{\text{输入端视角}} \\
                      &= \underbrace{\text{收到总重} - \text{填充垃圾}}_{\text{输出端视角}}
    \end{align*}
    这在数学上对应韦恩图(Venn Diagram)的交集部分。

    \textbf{(2) 工程指导意义}
    \begin{itemize}
        \item \textbf{针对 $H(X|Y)$ (损失)}:
        \begin{itemize}
            \item \textit{通信目标}:通过信道编码,使 $H(X|Y) \to 0$。
            \item \textit{保密目标}:加密通信希望 $H(X|Y)$ 极大化。
        \end{itemize}
        \item \textbf{针对 $H(Y|X)$ (噪声)}:
        \begin{itemize}
            \item 物理信道确定时,$H(Y|X)$ 通常固定。
            \item \textbf{容量策略}:既然减数(噪声)固定,要提高容量 $C$,只能调整发送策略使\textbf{被减数 $H(Y)$ 最大化}。
        \end{itemize}
    \end{itemize}

    \tcbline

    \textbf{3. 信道容量定义}
    \[ C = \max_{P(x)} I(X;Y) = \max_{P(x)} [H(Y) - H(Y|X)] \]
    \footnotesize{注:信道容量是针对\textbf{所有可能的信源分布 $P(x)$} 寻找最大互信息。}
\end{kbox}

\begin{examplebox}{推导实例:二进对称信道 (BSC) 的容量}
    \textbf{1. 模型定义}
    假设信道是对称的,误码率为 $\epsilon$ (或 $p$):
    \begin{itemize}
        \item 正确传输:$P(0|0) = P(1|1) = 1 - \epsilon$
        \item 错误传输:$P(1|0) = P(0|1) = \epsilon$
    \end{itemize}

    \begin{center}
    \begin{tikzpicture}[scale=0.8, >=Stealth]
        \node (x0) at (0, 1.5) {$0$};
        \node (x1) at (0, 0) {$1$};
        \node (y0) at (3, 1.5) {$0$};
        \node (y1) at (3, 0) {$1$};
        
        \draw[->, thick, mainblue] (x0) -- (y0) node[midway, above] {\scriptsize $1-\epsilon$};
        \draw[->, thick, mainblue] (x1) -- (y1) node[midway, below] {\scriptsize $1-\epsilon$};
        \draw[->, thick, alertred, dashed] (x0) -- (y1) node[pos=0.25, below] {\scriptsize $\epsilon$};
        \draw[->, thick, alertred, dashed] (x1) -- (y0) node[pos=0.25, above] {\scriptsize $\epsilon$};
        
        \node at (-0.5, 0.75) {Input $X$};
        \node at (3.5, 0.75) {Output $Y$};
    \end{tikzpicture}
    \end{center}

    \textbf{2. 计算噪声熵 $H(Y|X)$}
    \[ H(Y|X) = \sum_{x} P(x) H(Y|X=x) \]
    由于信道对称,无论发送 0 还是 1,其条件熵都是一样的。
    % [修复] 使用 align* 拆分过长的熵公式
    \begin{align*}
        H(Y|X=0) &= H(Y|X=1) \\
                 &= -[\epsilon \log_2 \epsilon + (1-\epsilon)\log_2(1-\epsilon)] \\
                 &\triangleq H(\epsilon)
    \end{align*}
    \textbf{关键点}:$H(Y|X) = H(\epsilon)$ 仅取决于信道特性,与信源分布 $P(x)$ \textbf{无关}。

    \tcbline

    \textbf{3. 计算输出熵 $H(Y)$ 并最大化}
    \[ C = \max_{P(x)} [H(Y) - H(\epsilon)] = \max_{P(x)} H(Y) - H(\epsilon) \]
    
    设信源概率 $P(X=0) = \alpha$,则 $P(X=1) = 1-\alpha$。由全概率公式:
    % [修复] 使用 align* 拆分过长的全概率公式
    \begin{align*}
        P(Y=0) &= P(0|0)\alpha + P(0|1)(1-\alpha) \\
               &= (1-\epsilon)\alpha + \epsilon(1-\alpha)
    \end{align*}
    
    \textbf{分析}:
    要使 $C$ 最大,需使 $H(Y)$ 最大。对于二进制变量,$H(Y)$ 的最大值为 1 bit,条件是输出概率相等,即 $P(Y=0)=P(Y=1)=0.5$。
    
    \textbf{4. 最终容量公式}
    当信源等概分布 ($P(X=0)=0.5$) 且信道对称时,输出也等概,此时达到容量:
    \[ \Large \boxed{ C = 1 - H(\epsilon) \quad (\text{b/symbol}) } \]
    
\end{examplebox}