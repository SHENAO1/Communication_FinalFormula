% file: content/03_Channels/topic06_discrete_capacity.tex

\begin{kbox}{5. 离散信道容量基础}
    
    \textbf{1. 互信息的两种物理视角}
    根据熵的定义,互信息 $I(X;Y)$ 可以从两个角度理解:
    
    % 使用直观理解盒子强调笔记中的概念
    \begin{intuitionbox}{辨析:损失熵 vs 噪声熵}
        \begin{itemize}
            \item \textbf{观察输入端 ($H(X) - H(X|Y)$)}:
            \begin{itemize}
                \item $H(X)$:信源原本的信息量(发送的干货)。
                \item \textbf{$H(X|Y)$ (损失熵/疑义度)}:接收到 $Y$ 后,对 $X$ 仍然存在的不确定性。代表在传输过程中\textbf{损失}掉的信息。
            \end{itemize}
            
            \item \textbf{观察输出端 ($H(Y) - H(Y|X)$)}:
            \begin{itemize}
                \item $H(Y)$:接收端收到的总信息量(包含干货和噪声)。
                \item \textbf{$H(Y|X)$ (噪声熵)}:给定发送 $X$ 的情况下,$Y$ 的不确定性。这完全是由信道\textbf{噪声}引起的随机性。
            \end{itemize}
        \end{itemize}
    \end{intuitionbox}

    % [修复] 使用 align* 环境并拆分长公式,防止溢出 (参考 image_9bc048.png)
    \textbf{2. 深入理解:逻辑连接与实际意义}
    
    \textbf{(1) 为什么两者相等?(快递包裹类比)}
    虽然发送和接收的总熵不同,但扣除各自的无效成分后,剩余部分守恒:
    
    \begin{align*}
        \text{互信息} &= \underbrace{\text{发送总量} - \text{路上丢件}}_{\text{输入端视角}} \\
                      &= \underbrace{\text{收到总重} - \text{填充垃圾}}_{\text{输出端视角}}
    \end{align*}
    这在数学上对应韦恩图(Venn Diagram)的交集部分。

    \textbf{(2) 工程指导意义}
    \begin{itemize}
        \item \textbf{针对 $H(X|Y)$ (损失)}:
        \begin{itemize}
            \item \textit{通信目标}:通过信道编码,使 $H(X|Y) \to 0$。
            \item \textit{保密目标}:加密通信希望 $H(X|Y)$ 极大化。
        \end{itemize}
        \item \textbf{针对 $H(Y|X)$ (噪声)}:
        \begin{itemize}
            \item 物理信道确定时,$H(Y|X)$ 通常固定。
            \item \textbf{容量策略}:既然减数(噪声)固定,要提高容量 $C$,只能调整发送策略使\textbf{被减数 $H(Y)$ 最大化}。
        \end{itemize}
    \end{itemize}

    \tcbline

    \textbf{3. 信道容量定义}
    \[ C = \max_{P(x)} I(X;Y) = \max_{P(x)} [H(Y) - H(Y|X)] \]
    \footnotesize{注:信道容量是针对\textbf{所有可能的信源分布 $P(x)$} 寻找最大互信息。}
\end{kbox}

\begin{examplebox}{推导实例:二进对称信道 (BSC) 的容量}
    % 
    \textbf{1. 模型定义与对称性含义}
    
    “离散二进对称信道”的含义如下:
    \begin{itemize}
        \item \textbf{正确传递概率相等}:
        \[ P(0|0) = P(1|1) = 1 - \epsilon \]
        \item \textbf{错误传递概率相等}:
        \[ P(0|1) = P(1|0) = \epsilon \]
    \end{itemize}
    且满足概率归一性:$P(1|1) + P(0|1) = 1$。

    \begin{center}
    \begin{tikzpicture}[scale=0.8, >=Stealth]
        \node (x0) at (0, 1.5) {$0$};
        \node (x1) at (0, 0) {$1$};
        \node (y0) at (3, 1.5) {$0$};
        \node (y1) at (3, 0) {$1$};
        \draw[->, thick, mainblue] (x0) -- (y0) node[midway, above] {\scriptsize $1-\epsilon$};
        \draw[->, thick, mainblue] (x1) -- (y1) node[midway, below] {\scriptsize $1-\epsilon$};
        \draw[->, thick, alertred, dashed] (x0) -- (y1) node[pos=0.25, below] {\scriptsize $\epsilon$};
        \draw[->, thick, alertred, dashed] (x1) -- (y0) node[pos=0.25, above] {\scriptsize $\epsilon$};
        \node at (-0.5, 0.75) {Input $X$};
        \node at (3.5, 0.75) {Output $Y$};
    \end{tikzpicture}
    \end{center}

    \textbf{2. 第一步:计算噪声熵 $H(Y|X)$}
    \[ H(Y|X) = \sum_{x} P(x) H(Y|X=x) \]
    由于信道对称,无论发0还是发1,其条件熵相同:
    \begin{align*}
        H(Y|X=0) &= H(Y|X=1) \\
                 &= -[\epsilon \log_2 \epsilon + (1-\epsilon)\log_2(1-\epsilon)] \\
                 &\triangleq H(\epsilon)
    \end{align*}
    \textbf{结论}:$H(Y|X)$ 仅由信道特性 $\epsilon$ 决定,与信源 $P(x)$ 无关。

    \tcbline

    % 
    \textbf{3. 第二步:输出概率推导}
    
    为了使 $C = \max [H(Y) - H(\epsilon)]$ 最大,需最大化 $H(Y)$。
    我们考察\textbf{输入等概}的情况,即 $P(X=0)=P(X=1)=\frac{1}{2}$。
    
    利用全概率公式,分别计算输出端收到 0 和 1 的概率:
    \begin{align*}
        P(Y=0) &= P(X=0)P(0|0) + P(X=1)P(0|1) \\
               &= \frac{1}{2}(1-\epsilon) + \frac{1}{2}(\epsilon) = \frac{1}{2} \\[1em]
        P(Y=1) &= P(X=1)P(1|1) + P(X=0)P(1|0) \\
               &= \frac{1}{2}(1-\epsilon) + \frac{1}{2}(\epsilon) = \frac{1}{2}
    \end{align*}
    
    \textbf{推导结论}:离散二进制对称信道,若\textbf{输入等概},则\textbf{输出也等概}。

    \tcbline
    
    % 
    \textbf{4. 第三步:验证转移概率与后验概率的关系}
    
    基于“输入等概($1/2$)”导致“输出等概($1/2$)”的条件,利用贝叶斯公式推导后验概率。
    
    \small
    \begin{align*}
        P(X=0|Y=0) &= \frac{P(Y=0|X=0) \cdot \cancel{0.5}}{\cancel{0.5}} = P(Y=0|X=0) \\
        P(X=1|Y=0) &= \frac{P(Y=0|X=1) \cdot \cancel{0.5}}{\cancel{0.5}} = P(Y=0|X=1) \\
        P(X=0|Y=1) &= \frac{P(Y=1|X=0) \cdot \cancel{0.5}}{\cancel{0.5}} = P(Y=1|X=0) \\
        P(X=1|Y=1) &= \frac{P(Y=1|X=1) \cdot \cancel{0.5}}{\cancel{0.5}} = P(Y=1|X=1)
    \end{align*}
    \normalsize

    % 
    \begin{itemize}
        \item \textbf{思考:分子上的 0.5 和分母上的 0.5 物理意义相同吗?}
        \begin{itemize}
            \item \textbf{分子上的 0.5}:代表\textbf{先验概率} $P(X)$(信源发送的统计特性)。
            \item \textbf{分母上的 0.5}:代表\textbf{输出概率} $P(Y)$(接收端收到的统计特性)。
            \item \textbf{结论}:两者\textbf{物理来源不同}。正是因为“信道对称”且“输入等概”,才使得经全概率公式计算出的 $P(Y)$ 恰好在数值上等于 $P(X)$,从而可以被约去。
        \end{itemize}
    \end{itemize}

    \vspace{5pt}
    \textbf{重要性质}:
    \textcolor{alertred}{在输入等概时,离散二进对称信道的转移概率等于后验概率。}
    
    {\Large
    \[ \boxed{ P(y_j | x_i) = P(x_i | y_j) } \]
    }

    \tcbline
    
    \textbf{5. 最终容量公式}
    当输出等概时,$H(Y)$ 达到最大值 1 bit。
    \[ C = 1 - H(\epsilon) \quad (\text{b/symbol}) \]
    
\end{examplebox}