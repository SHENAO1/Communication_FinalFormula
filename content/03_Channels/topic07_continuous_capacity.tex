% file: content/03_Channels/topic07_continuous_capacity.tex

\begin{kbox}{6. 连续信道容量 (Shannon Formula)}
    
    对于带宽受限、存在高斯白噪声的连续信道,其容量由\textbf{香农公式}给出:
    
    \[ C = B \log_2 \left( 1 + \frac{S}{N} \right) \quad (\text{b/s}) \]
    
    \begin{itemize}
        \item $B$:信道带宽 (Hz)。
        \item $S$:信号平均功率 (W)。
        \item $N = n_0 B$:噪声功率 (W),其中 $n_0$ 为单边噪声功率谱密度。
    \end{itemize}

    \tcbline
    
    \textbf{物理意义与权衡 (Trade-off)}:
    \begin{enumerate}
        \item \textbf{带宽与信噪比互换}:为了保持 $C$ 不变,可以降低信噪比 $S/N$,但必须增加带宽 $B$(扩频通信的基础)。
        \item \textbf{无误码传输}:只要信息传输速率 $R \le C$,理论上就存在某种编码方式,可以实现无误码传输。
    \end{enumerate}
\end{kbox}

\begin{intuitionbox}{极限情况分析:带宽 $B \to \infty$}
    当带宽趋于无穷大时,信道容量\textbf{不会}趋于无穷大,而是趋于一个定值。这是因为噪声功率 $N=n_0 B$ 也会随带宽增加而增加。
    
    \[ \lim_{B \to \infty} C = \lim_{B \to \infty} B \log_2 \left( 1 + \frac{S}{n_0 B} \right) \approx 1.44 \frac{S}{n_0} \]
    
    这表明:在给定信号功率 $S$ 和噪声密度 $n_0$ 的情况下,仅靠增加带宽所能获得的容量是有上限的。
\end{intuitionbox}