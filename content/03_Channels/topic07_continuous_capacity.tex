% file: content/03_Channels/topic07_continuous_capacity.tex

\begin{kbox}{6. 连续信道容量 (Continuous Channel Capacity)}

    \subsection*{6.1 无扰信道 vs. 有扰信道}
        
    在讨论香农公式前,需区分两种信道环境下的极限:
    
    \begin{itemize}
        \item \textbf{无扰信道 (奈奎斯特定理)}:
        对于带宽为 $B$ (Hz) 的理想无噪声信道,若信号采用 $M$ 进制传输,其极限信息速率为:
        \[ C_t = 2B \log_2 M \quad (\text{b/s}) \]
        \textit{注:这限制的是码元速率以避免码间串扰 (ISI),未考虑噪声影响。}
        
        \item \textbf{有扰信道 (香农公式)}:
        对于带宽为 $B$ (Hz)、存在加性高斯白噪声 (AWGN) 的信道,其无差错传输的最大信息速率为:
        \[ C_t = B \log_2 \left( 1 + \frac{S}{N} \right) \quad (\text{b/s}) \]
    \end{itemize}

    \tcbline

    \subsection*{6.2 香农公式 (Shannon Formula)}
    
    \[ C_t = B \log_2 \left( 1 + \frac{S}{n_0 B} \right) \]
    
    \textbf{核心三要素}:
    \begin{itemize}
        \item $B$:信道带宽 (Hz)。
        \item $S$:信号平均功率 (W)。
        \item $n_0$:噪声单边功率谱密度 (W/Hz),噪声功率 $N = n_0 B$。
    \end{itemize}
    
    \textbf{信道编码定理 (Channel Coding Theorem)}:
    \begin{itemize}
        \item 若信息传输速率 $R_b \le C_t$,理论上\textbf{总能找到}一种编码方式,实现无差错传输。
        \item 若 $R_b > C_t$,则\textbf{不可能}实现无差错传输。
    \end{itemize}
\end{kbox}

\begin{intuitionbox}{香农公式的“存在性”本质}
    \textbf{存在性定理 (Existence Theorem)}:
    香农公式仅证明了理想通信系统极限信息率的“存在性”,即指出了理论上的上限(Upper Bound)和无差错传输的可能性,但\textbf{未给出具体实现方法}(Constructive Method)。
    
    这意味着它告诉我们“可以做到”,但没告诉我们“怎么做”。现代通信技术(如Turbo码、LDPC码、Polar码)的发展正是为了不断逼近这一理论极限。
\end{intuitionbox}