% file: content/03_Channels/topic07_ex_shannon_analysis.tex

\begin{examplebox}{推导:香农极限 (Shannon Limit Derivation)}
    \textbf{问题}:当信道带宽 $B \to \infty$ 时,信道容量 $C_t$ 是否无限增加?
    
    \[ \lim_{B \to \infty} C_t = \lim_{B \to \infty} B \log_2 \left( 1 + \frac{S}{n_0 B} \right) \]
    
        
    \tcbline
    
    \textbf{方法一:重要极限法 (PPT思路)} 
    
    利用重要极限 $\lim_{z \to \infty} (1 + \frac{1}{z})^z = e$ 进行凑形。
    
    令 $x = \frac{n_0 B}{S}$。当 $B \to \infty$ 时, $x \to \infty$。且 $B = \frac{S}{n_0} x$。
    
    \begin{align*}
        \lim_{B \to \infty} C_t &= \lim_{x \to \infty} \frac{S}{n_0} x \cdot \log_2 \left( 1 + \frac{1}{x} \right) \\
        &= \frac{S}{n_0} \lim_{x \to \infty} \log_2 \left( 1 + \frac{1}{x} \right)^x \\
        &= \frac{S}{n_0} \log_2 e \approx 1.44 \frac{S}{n_0}
    \end{align*}

    \tcbline
    
    \textbf{方法二:变量代换与等价无穷小 (手写笔记思路)} 
    
    利用 $\lim_{u \to 0} \ln(1+u) \sim u$ 简化计算。
    
    令 $t = \frac{1}{B}$。当 $B \to \infty$ 时, $t \to 0$。原式变形为 $\frac{0}{0}$ 型极限:
    
    \begin{align*}
        \lim_{B \to \infty} C_t &= \lim_{t \to 0} \frac{\log_2 (1 + \frac{S}{n_0} t)}{t} \\
        \text{\small (换底公式)} \quad &= \lim_{t \to 0} \frac{\ln (1 + \frac{S}{n_0} t)}{t \cdot \ln 2} \\
        \text{\small (等价无穷小)} \quad &= \frac{1}{\ln 2} \lim_{t \to 0} \frac{\frac{S}{n_0} t}{t} \\
        &= \frac{1}{\ln 2} \frac{S}{n_0} \approx 1.44 \frac{S}{n_0}
    \end{align*}

    \tcbline
    
    \textbf{结论}:
    \begin{itemize}
        \item 当带宽 $B \to \infty$ 时,信道容量趋于定值 $1.44 \frac{S}{n_0}$ (即 $C_\infty$)。
        \item 这表明:在给定噪声功率谱密度的情况下,仅靠增加带宽,信道容量是有上限的。
    \end{itemize}
\end{examplebox}

\begin{kbox}{工程应用:三要素的互换与权衡}
        
    根据公式,给定信道容量 $C_t$,带宽 $B$、信噪比 $S/N$ 及传输时间三者可以互换:
    
    \begin{enumerate}
        \item \textbf{以带宽换信噪比 (宽带/扩频通信)}:
        \begin{itemize}
            \item \textbf{原理}:提高带宽 $B$,可以容忍更低的信噪比 $S/N$ 而保持 $C_t$ 不变。
            \item \textbf{应用}:深空通信(信号极其微弱)、CDMA、扩频通信。
            \item \textbf{直观}:将信号能量分散在很宽的频带上,使信号淹没在噪声中,接收端再通过扩频码提取。
        \end{itemize}
        
        \item \textbf{以信噪比换带宽 (窄带通信)}:
        \begin{itemize}
            \item \textbf{原理}:提高信噪比 $S/N$(增加发射功率),可以降低对带宽 $B$ 的需求。
            \item \textbf{应用}:有线载波电话、频带拥挤的场合。
            \item \textbf{直观}:使用高阶调制(如 256-QAM),在有限带宽内传输更多比特,但要求信号非常干净(高SNR)。
        \end{itemize}
        
        \item \textbf{以带宽换时间}:
        \begin{itemize}
            \item 信噪比不变,提高带宽 $B$ 可以增加容量 $C_t$,从而降低传输同样数据量所需的时间。
        \end{itemize}
    \end{enumerate}
\end{kbox}