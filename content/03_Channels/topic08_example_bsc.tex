% file: content/03_Channels/topic08_example_bsc.tex

\begin{examplebox}{例 3.3:BSC信道建模与容量计算}
    \textbf{题目描述}:
    设信源由两种符号 “0” 和 “1” 组成,符号传输速率为 $R_B = 1000$ 符号/s。
    且这两种符号的出现概率相等,均等于 $1/2$。
    信道为对称信道 (BSC),其传输的符号错误概率为 $P_e = 1/128$。
    
    \textbf{任务}:试画出此信道模型,并求此信道的容量 $C$ 和 $C_t$。
    
    \tcbline
    
    \textbf{解:1. 建立信道模型图}
    
    根据题意,输入概率 $P(x=0)=P(x=1)=\frac{1}{2}$。
    错误转移概率 $P(0|1) = P(1|0) = \frac{1}{128}$。
    正确转移概率 $P(0|0) = P(1|1) = 1 - \frac{1}{128} = \frac{127}{128}$。
    
    模型图如下:
    \begin{center}
        \begin{tikzpicture}[>=stealth, scale=1.2]
            % 定义节点
            \node (x0) at (0, 2) {$0$};
            \node (x1) at (0, 0) {$1$};
            \node (y0) at (4, 2) {$0$};
            \node (y1) at (4, 0) {$1$};
            
            % 添加输入概率标注
            \node[left] at (x0) {$P(x_0)=\frac{1}{2}$};
            \node[left] at (x1) {$P(x_1)=\frac{1}{2}$};
            
            % 绘制连线与概率
            % 直通 (正确)
            \draw[->, thick, mainblue] (x0) -- (y0) node[midway, above] {$\frac{127}{128}$};
            \draw[->, thick, mainblue] (x1) -- (y1) node[midway, below] {$\frac{127}{128}$};
            
            % 交叉 (错误)
            \draw[->, dashed, alertred] (x0) -- (y1) node[near end, above right, font=\footnotesize] {$\frac{1}{128}$};
            \draw[->, dashed, alertred] (x1) -- (y0) node[near end, below right, font=\footnotesize] {$\frac{1}{128}$};
            
            % 标题
            \node[above] at (0, 2.5) {输入 $X$};
            \node[above] at (4, 2.5) {输出 $Y$};
        \end{tikzpicture}
    \end{center}

    \tcbline

    \textbf{2. 求解输出概率分布 $P(y)$}
    
    由于输入等概且信道对称,输出也应等概。验证如下:
    \[
    \begin{aligned}
        P(y=0) &= P(x=0)P(y=0|x=0) + P(x=1)P(y=0|x=1) \\
               &= \frac{1}{2} \cdot \frac{127}{128} + \frac{1}{2} \cdot \frac{1}{128} \\
               &= \frac{1}{2} \left( \frac{127+1}{128} \right) = \frac{1}{2}
    \end{aligned}
    \]
    同理,$P(y=1) = 1/2$。
    
    \textbf{结论}:输出端熵达到最大值:
    \[ H(Y) = -\sum P(y)\log_2 P(y) = 1 \text{ bit/symbol} \]

    \tcbline
    
    \textbf{3. 求解信道容量 $C$ (bits/symbol)}
    
    根据信道容量定义:
    \[ C = \max_{P(x)} I(X;Y) = \max_{P(x)} [H(Y) - H(Y|X)] \]
    
    \textbf{(1) 计算条件熵 $H(Y|X)$}:
    对于BSC信道,条件熵只与转移概率有关,与输入分布无关:
    \[
    \begin{aligned}
        H(Y|X) &= \sum_{x} P(x) H(Y|X=x) \\
               &= - \sum_{i=0}^{1} \sum_{j=0}^{1} P(x_i) P(y_j|x_i) \log_2 P(y_j|x_i)
    \end{aligned}
    \]
    代入数值计算:
    \[
    \begin{aligned}
        H(Y|X) &= - \left( \frac{127}{128} \log_2 \frac{127}{128} + \frac{1}{128} \log_2 \frac{1}{128} \right) \\
               &\approx - (0.992 \times (-0.011) + 0.0078 \times (-7)) \\
               &\approx 0.0109 + 0.0546 \\
               &\approx 0.0659 \text{ bits/symbol}
    \end{aligned}
    \]
    
    \textbf{(2) 计算容量 $C$}:
    由于 $P(x)$ 已经是均匀分布,此时 $H(Y)$ 已取最大值 1,故:
    \[
    \begin{aligned}
        C &= H(Y)_{\max} - H(Y|X) \\
          &= 1 - 0.0659 \\
          &= 0.9341 \text{ bits/symbol}
    \end{aligned}
    \]

    \tcbline
    
    \textbf{4. 求解信道容量 $C_t$ (bits/s)}
    
    已知符号速率 $R_B = 1000$ Baud。
    \[
    \begin{aligned}
        C_t &= R_B \cdot C \\
            &= 1000 \times 0.9341 \\
            &= 934.1 \text{ bits/s}
    \end{aligned}
    \]
    
    \textit{注:这意味着在每秒传输的 1000 个符号中,扣除用于纠错的冗余度(由信道噪声引起的不确定性),实际有效的信息传输速率约为 934.1 bit/s。}
\end{examplebox}