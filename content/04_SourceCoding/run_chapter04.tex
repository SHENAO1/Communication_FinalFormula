% content/04_SourceCoding/run_chapter04.tex
\documentclass{../../commsummary}
\graphicspath{{./figures/}{../../figures/}}

% 引入必要的 TikZ 库用于绘图

\begin{document}

\section{信源编码 (Source Coding)}

% --- 0. 概述与波形编码基础 ---
% 0.1 概述:定义、目的与分类
% content/04_SourceCoding/topic00_intro.tex

\begin{kbox}{信源编码 vs. 信道编码}
    \begin{itemize}
        \item \textbf{定义与核心目的}
        \begin{itemize}
            \item \textbf{信源编码 (Source Coding)}:
            \begin{itemize}
                \item \textbf{目的}:提高通信的\textbf{有效性} (Efficiency)。
                \item \textbf{手段}:\textbf{减少冗余}。通过压缩算法(如哈夫曼、PCM、JPEG)去除原始信息中不必要的成分,降低数据率,节省带宽/存储。
                \item \textbf{核心指标}:压缩比、失真度。
            \end{itemize}
            \item \textbf{信道编码 (Channel Coding)}:
            \begin{itemize}
                \item \textbf{目的}:提高通信的\textbf{可靠性} (Reliability)。
                \item \textbf{手段}:\textbf{增加冗余}。人为添加监督位(校验位),使接收端能发现或纠正传输中的错误(如奇偶校验、CRC、卷积码)。
                \item \textbf{核心指标}:误码率、编码增益。
            \end{itemize}
        \end{itemize}
        
        \item \textbf{对比总结表}
        \begin{center}
        \footnotesize
        \begin{tabular}{|c|c|c|}
            \hline
            \textbf{维度} & \textbf{信源编码} & \textbf{信道编码} \\
            \hline
            \textbf{作用} & “瘦身” (压缩) & “穿甲” (保护) \\
            \hline
            \textbf{冗余操作} & \textbf{去除}冗余 & \textbf{增加}冗余 \\
            \hline
            \textbf{关键目标} & 有效性 (省带宽) & 可靠性 (抗干扰) \\
            \hline
        \end{tabular}
        \end{center}
    \end{itemize}
\end{kbox}

\begin{intuitionbox}{直观理解:冗余的减法与加法}
    \begin{itemize}
        \item \textbf{信源编码 (减法)}:类似于\textbf{写缩写}。
        \par 把 ``For Your Information'' 缩写为 ``FYI''。这去除了多余的字母,让书写和阅读更快(提高了有效性)。
        
        \item \textbf{信道编码 (加法)}:类似于\textbf{打电话时的确认}。
        \par 说 ``我是张三,弓长张,一二三的三''。额外多说的几句话就是“冗余”,目的是为了防止对方听错(提高了可靠性)。
    \end{itemize}
\end{intuitionbox}

% 0.2 波形编码的三个步骤:抽样、量化、编码
% content/04_SourceCoding/topic00_waveform.tex

\begin{kbox}{波形编码的三个步骤 (详解)}
    根据,波形编码(如PCM)将模拟信号转化为数字信号主要经过以下三步:

    \begin{center}
    % 使用 resizebox 确保整体宽度适应双栏
    \resizebox{\linewidth}{!}{
    \begin{tikzpicture}[
        scale=1, 
        every node/.style={transform shape},
        % 定义描述文本框样式
        desc/.style={
            align=left, 
            fill=gray!10, 
            text width=8.5cm, 
            rounded corners, 
            font=\small,
            inner sep=6pt,
            anchor=north
        }
    ]
        
        % --- 第一步:抽样 ---
        \begin{scope}[yshift=0cm]
            \draw[->] (-0.5,0) -- (7.5,0) node[right] {$t$};
            \draw[->] (0,-1.2) -- (0,1.8) node[right] {$m(t), m_s(t)$};
            \draw[gray, thick, dashed] plot[domain=0:7, samples=50] (\x, {sin(\x r * 1.5) + 0.2});
            
            \foreach \x in {0.5, 1.5, ..., 6.5} {
                \draw[blue, thick, -Latex] (\x,0) -- (\x, {sin(\x r * 1.5) + 0.2});
            }
            \node[blue, font=\bfseries] at (6, 1.8) {抽样信号};
            
            \node[desc, fill=blue!5] at (3.5, -1.6) {
                \textbf{1. 抽样 (Sampling)}\\
                $m_s(t)$:\textbf{时间上离散},但取值(幅度)仍然连续。\\
                称为:离散模拟信号。
            };
        \end{scope}

        % --- 第二步:量化 ---
        \begin{scope}[yshift=-6cm] 
            \draw[->] (-0.5,0) -- (7.5,0) node[right] {$t$};
            \draw[->] (0,-1.2) -- (0,1.8) node[right] {$m_q(t)$};
            \foreach \y in {-1, -0.5, 0, 0.5, 1, 1.5} {
                \draw[gray!30, dashed] (0,\y) -- (7,\y);
            }
            
            \foreach \x in {0.5, 1.5, ..., 6.5} {
                \pgfmathsetmacro{\val}{sin(\x r * 1.5) + 0.2}
                \pgfmathsetmacro{\quantized}{round(\val*2)/2} 
                \draw[magenta, line width=2pt] (\x,0) -- (\x, \quantized);
                \draw[magenta, fill=white] (\x, \quantized) circle (2pt);
            }
            \node[magenta, font=\bfseries] at (6, 1.8) {量化信号};

            \node[desc, fill=magenta!5] at (3.5, -1.6) {
                \textbf{2. 量化 (Quantization)}\\
                $m_q(t)$:\textbf{时间上离散},\textbf{取值(幅度)也离散}。\\
                此时脉冲的\textbf{长短不一},直接反映信号幅度大小。
            };
        \end{scope}

        % --- 第三步:编码 ---
        \begin{scope}[yshift=-12cm] 
            \draw[->] (-0.5,0) -- (7.5,0) node[right] {$t$};
            
            \foreach \x in {0.5, 1.5, ..., 6.5} {
                \node[above, font=\scriptsize] at (\x, 1.0) {011}; 
                \draw[mainblue, fill=mainblue] (\x-0.15, 0) rectangle (\x+0.15, 0.7);
            }
            \node[mainblue, font=\bfseries] at (6, 1.8) {编码信号};

            % --- 在此处加入了解释 ---
            \node[desc, fill=mainblue!5] at (3.5, -0.5) {
                \textbf{3. 编码 (Encoding)}\\
                将量化电平映射为二进制码组 (如 010, 011)。\\
                \textbf{注意}:此时脉冲\textbf{高度恒定},仅代表数字逻辑电平(如5V)。原始信号的幅度信息已转化为\textbf{二进制数值},不再由脉冲高度承载。
            };
        \end{scope}
    \end{tikzpicture}
    }
    \end{center}
\end{kbox}


% --- 1. 抽样 (Sampling) ---
% 1.1 抽样的分类 (自然抽样、平顶抽样等概念引入)
% content/04_SourceCoding/topic01_sampling_classification.tex

\begin{kbox}{抽样技术的分类体系}
    根据手写笔记整理,抽样技术通常可以从以下三个维度进行分类:

    \begin{center}
    \resizebox{\linewidth}{!}{
    \begin{tikzpicture}[
        grow=right,
        level 1/.style={sibling distance=3.5cm, level distance=3.5cm},
        level 2/.style={sibling distance=1.2cm, level distance=4.5cm}, % 稍微增加 level distance 以容纳更长的灰色文字
        edge from parent/.style={draw, thick, gray},
        root/.style={rectangle, rounded corners, draw=mainblue, fill=mainblue!10, text centered, font=\bfseries\large, inner sep=10pt},
        node1/.style={rectangle, rounded corners, draw=blue!80, fill=blue!5, text centered, font=\bfseries, minimum height=0.8cm},
        node2/.style={rectangle, rounded corners, draw=gray!60, fill=white, text centered, font=\small, minimum height=0.6cm},
        % 定义一个统一的 label 样式,方便复用
        note/.style={label={[font=\scriptsize, text=gray, align=left, xshift=0.1cm]right:#1}}
        ]

        % 根节点
        \node[root] {抽样}
            % 分支 3:按脉冲形状
            child {node[node1] {3. 按脉冲形状}
                child {node[node2, note={脉冲有宽度}] {实际抽样 (平顶/自然)}}
                child {node[node2, note={$\delta(t)$序列}] {理想抽样 (冲激)}}
            }
            % 分支 2:按时间间隔
            child {node[node1] {2. 按时间间隔}
                child {node[node2] {非均匀抽样 (随机)}}
                child {node[node2, note={$T_s = \text{const}$}] {均匀抽样 (等间隔)}}
            }
            % 分支 1:按信号频谱
            child {node[node1] {1. 按信号频谱}
                child {node[node2, note={对应带通信号}] {带通抽样}}
                child {node[node2, note={对应基带信号}] {低通抽样}}
            };
    \end{tikzpicture}
    }
    \end{center}

    \begin{itemize}
        \item \textbf{1. 信号类型 (频谱特性)}:
        \begin{itemize}
            \item \textbf{低通抽样}:针对基带信号,满足 $f_s \ge 2f_H$ 。
            \item \textbf{带通抽样}:针对频谱在 $[f_L, f_H]$ 的信号,采样率可低于 $2f_H$。
        \end{itemize}
        
        \item \textbf{2. 抽样间隔 (时间特性)}:
        \begin{itemize}
            \item \textbf{均匀抽样}:抽样时刻 $t_n = nT_s$,间隔恒定。通信原理主要讨论此类。
            \item \textbf{非均匀抽样}:间隔随机或变化。
        \end{itemize}

        \item \textbf{3. 脉冲序列 (物理实现)}:
        \begin{itemize}
            \item \textbf{理想抽样}:使用冲激脉冲序列 $\delta_T(t)$,便于理论分析。
            \item \textbf{实际抽样}:使用具有一定宽度的脉冲(如矩形脉冲),会引入孔径失真(如平顶抽样)。
        \end{itemize}
    \end{itemize}
\end{kbox}

% 1.2 [核心] 理想抽样推导与奈奎斯特低通抽样定理
% (替代了原来的 topic01_sampling_derivation 和 topic01_sampling)
% --- Part 1: 数学基础回顾 ---
\begin{intuitionbox}{数学预备:从傅里叶级数到冲激串变换}
    理解抽样定理频谱搬移的关键,在于掌握\textbf{周期信号的频谱特性}。
    
    \subsubsection*{1. 傅里叶级数 (Fourier Series, FS) 回顾}
    对于周期为 $T_0$ 的周期信号 $f(t)$,角频率 $\omega_0 = \frac{2\pi}{T_0}$ (或基频 $f_0 = \frac{1}{T_0}$)。
    
    \begin{itemize}
        \item \textbf{三角形式}:
        \begin{equation}
            f(t) = a_0 + \sum_{n=1}^{\infty} \left[ a_n \cos(n\omega_0 t) + b_n \sin(n\omega_0 t) \right]
        \end{equation}
        其中系数为:
        \begin{equation}
        \begin{cases}
            \displaystyle a_0 = \frac{1}{T_0}\int_{T_0} f(t)dt \\[10pt]
            \displaystyle a_n = \frac{2}{T_0}\int_{T_0} f(t)\cos(n\omega_0 t)dt \\[10pt]
            \displaystyle b_n = \frac{2}{T_0}\int_{T_0} f(t)\sin(n\omega_0 t)dt
        \end{cases}
        \end{equation}

        \item \textbf{指数形式}:
        \begin{equation}
            f(t) = \sum_{n=-\infty}^{\infty} F_n e^{j n \omega_0 t}, \quad F_n = \frac{1}{T_0} \int_{-T_0/2}^{T_0/2} f(t) e^{-j n \omega_0 t} dt
        \end{equation}
    \end{itemize}

    \subsubsection*{2. 周期单位冲激串的傅里叶变换}
    设周期单位冲激串为 $\delta_T(t) = \sum_{n=-\infty}^{\infty} \delta(t - nT_s)$。
    
    \begin{itemize}
        \item \textbf{时域}:周期为 $T_s$ 的冲激串。
        \item \textbf{频域}:周期为 $f_s$ 的冲激串,且幅度加权为 $f_s$。
        \begin{equation}
            \Delta_T(f) = \mathcal{F}[\delta_T(t)] = \frac{1}{T_s} \sum_{n=-\infty}^{\infty} \delta(f - n f_s)
        \end{equation}
    \end{itemize}
\end{intuitionbox}

% --- Part 2: 理想抽样定理图解与推导 ---
\begin{figure*}[t]
% [修改说明] 这里添加了 [breakable=false],强制禁止跨页,修复遮挡问题
\begin{kbox}[breakable=false]{理想抽样过程:波形与频谱对照}
    理想抽样可以看作是模拟信号 $m(t)$ 与单位冲激脉冲序列 $\delta_T(t)$ 在时域的\textbf{乘积}。
    根据卷积定理:\textbf{时域相乘,对应频域卷积}。

    \centering
    \begin{tikzpicture}[>=Latex, xscale=1.1, yscale=1]
        % --- 定义颜色 ---
        \definecolor{myblue}{RGB}{0, 114, 189} 
        \definecolor{mygreen}{RGB}{119, 172, 48} 
        \tikzstyle{signal} = [thick, myblue]
        \tikzstyle{spectrum} = [thick, myblue, fill=myblue!10]
        \tikzstyle{axis} = [->, gray, thin]
        \tikzstyle{labeltext} = [font=\small]
        
        % --- 布局参数 ---
        \def\yRowA{4.2}   \def\yRowB{0}     \def\yRowC{-5.0}
        \def\xColL{-5}    \def\xColR{5}

        % =================================================
        % 第一行:模拟信号
        % =================================================
        \begin{scope}[shift={(\xColL, \yRowA)}]
            \draw[axis] (-2.5,0) -- (2.5,0) node[right] {$t$};
            \draw[axis] (0,-0.5) -- (0,1.5) node[above] {$m(t)$};
            \draw[signal] plot[domain=-2:2, samples=100] (\x, {0.8*cos(50*\x) + 0.3*cos(120*\x)});
            \node[labeltext] at (0, -0.8) {(a) 模拟信号};
        \end{scope}

        \draw[<->, thick, gray!50] (\xColL+4, \yRowA+0.5) -- (\xColR-4, \yRowA+0.5);
        \begin{scope}[shift={(\xColR, \yRowA)}]
            \draw[axis] (-2.5,0) -- (2.5,0) node[right] {$f$};
            \draw[axis] (0,-0.5) -- (0,1.5) node[above] {$|M(f)|$};
            \draw[spectrum] (-1.2,0) -- (0,1.2) -- (1.2,0) -- cycle;
            \node[above right, font=\small] at (0, 1.2) {$A$};
            \node[below] at (-1.2,0) {$-f_H$};
            \node[below] at (1.2,0) {$f_H$};
            \node[labeltext] at (0, -0.8) {(b) 信号频谱};
        \end{scope}

        % =================================================
        % 运算符号
        % =================================================
        \def\opY{{(\yRowA+\yRowB)/2 + 0.5}} 
        \node[font=\bfseries\huge, text=mygreen] at (\xColL, \opY) {$\times$};
        \node[align=center, font=\bfseries\small, text=mygreen] at (\xColL+1.0, \opY) {时域\\相乘};
        \node[font=\bfseries\huge, text=mygreen] at (\xColR, \opY) {$*$};
        \node[align=center, font=\bfseries\small, text=mygreen] at (\xColR+1.0, \opY) {频域\\卷积};

        % =================================================
        % 第二行:冲激序列
        % =================================================
        \begin{scope}[shift={(\xColL, \yRowB)}]
            \draw[axis] (-2.5,0) -- (2.5,0) node[right] {$t$};
            \draw[axis] (0,-0.5) -- (0,1.5) node[above] {$\delta_T(t)$};
            \foreach \x in {-2,-1.5,...,2} { \draw[->, thick, black] (\x,0) -- (\x,1); }
            \node[below, text=red] at (0.5,0) {$T_s$};
            \node[labeltext] at (0, -0.8) {(c) 冲激序列};
        \end{scope}
        
        \draw[<->, thick, gray!50] (\xColL+4, \yRowB+0.5) -- (\xColR-4, \yRowB+0.5);
        \begin{scope}[shift={(\xColR, \yRowB)}]
            \draw[axis] (-2.5,0) -- (2.5,0) node[right] {$f$};
            \draw[axis] (0,-0.5) -- (0,1.5) node[above] {$\delta_T(f)$};
            \foreach \x in {-2,-1,0,1,2} { \draw[->, thick, black] (\x,0) -- (\x,1); }
            \node[above right, font=\small] at (0, 1) {$f_s$};
            \node[labeltext] at (0, -0.8) {(d) 脉冲频谱};
        \end{scope}

        % =================================================
        % 等号
        % =================================================
        \draw[double, double distance=2pt, gray] (\xColL, {\yRowB - 1.2}) -- (\xColL, {\yRowB - 1.8});
        \draw[double, double distance=2pt, gray] (\xColR, {\yRowB - 1.2}) -- (\xColR, {\yRowB - 1.8});

        % =================================================
        % 第三行:已抽样信号
        % =================================================
        \begin{scope}[shift={(\xColL, \yRowC)}]
            \draw[axis] (-2.5,0) -- (2.5,0) node[right] {$t$};
            \draw[axis] (0,-0.5) -- (0,2.0) node[above] {$m_s(t)$};
            \draw[dashed, gray!80] plot[domain=-2.2:2.2, samples=100] (\x, {0.8*cos(50*\x) + 0.3*cos(120*\x)});
            \foreach \x in {-2,-1.5,...,2} {
                \draw[-, thick, myblue] (\x,0) -- (\x, {0.8*cos(50*\x) + 0.3*cos(120*\x)});
                \fill[myblue] (\x, {0.8*cos(50*\x) + 0.3*cos(120*\x)}) circle (1.8pt);
            }
            \node[labeltext] at (0, -0.8) {(e) 抽样信号波形};
        \end{scope}
        
        \draw[<->, thick, gray!50] (\xColL+4, \yRowC+0.5) -- (\xColR-4, \yRowC+0.5);
        \begin{scope}[shift={(\xColR, \yRowC)}]
            \draw[axis] (-2.5,0) -- (2.5,0) node[right] {$f$};
            \draw[axis] (0,-0.5) -- (0,2.4) node[above] {$M_s(f)$};
            \foreach \k in {-2,-1,0,1,2} {
                \draw[spectrum] (\k-0.4, 0) -- (\k, 1.2) -- (\k+0.4, 0) -- cycle;
            }
            \node[above right, font=\small] at (0, 1.5) {$Af_s$};
            % 滤波器标注
            \draw[thick, dashed, red] (-0.55, -0.1) rectangle (0.55, 1.45);
            \node[red, font=\scriptsize] (filterLabel) at (-1.3, 1.9) {低通滤波};
            \draw[->, red, dashed, thick] (filterLabel) -- (-0.55, 1.45);
            % 坐标轴
            \node[below] at (0,0) {$0$};
            \node[below] at (1,0) {$f_s$}; \node[below] at (-1,0) {$-f_s$};
            % 条件框
            \node[draw, rounded corners, fill=yellow!10, font=\bfseries\small, text=red] at (1.8, 1.5) {$f_s \ge 2f_H$};
            \node[labeltext] at (0, -0.8) {(f) 周期延拓频谱};
        \end{scope}
    \end{tikzpicture}
    
    \tcblower
    
    \textbf{核心数学推导}:
    \begin{itemize}
        \item \textbf{时域表达式}:
        \begin{equation}
            m_s(t) = m(t) \cdot \delta_T(t) = m(t) \sum_{n=-\infty}^{\infty} \delta(t - nT_s)
        \end{equation}
        \item \textbf{频域表达式}(应用卷积定理):
        \begin{equation}
            \begin{aligned}
                M_s(f) &= M(f) * \left[ \frac{1}{T_s} \sum_{n=-\infty}^{\infty} \delta(f - n f_s) \right] \\
                       &= \frac{1}{T_s} \sum_{n=-\infty}^{\infty} M(f - n f_s)
            \end{aligned}
        \end{equation}
    \end{itemize}
\end{kbox}
\end{figure*}

% --- Part 3: 定理定义提取 ---
% [注] 这里的用法仍然兼容,因为 [2][] 允许省略可选参数
\begin{kbox}{低通抽样定理 (Nyquist Sampling Theorem)}
    \textbf{定理内容}:对于最高频率为 $f_H$ 的模拟信号,无失真恢复条件为:
    \begin{itemize}
        \item \textbf{抽样速率条件}:$\boxed{f_s \ge 2f_H}$
        \item \textbf{抽样间隔条件}:$\boxed{T_s \le \frac{1}{2f_H}}$
    \end{itemize}

    \textbf{应用实例}:电话信号 $f_H \approx 3.4\text{kHz}$,理论 $f_s \ge 6.8\text{kHz}$,工程标准取 $f_s = 8\text{kHz}$。
\end{kbox}

% 1.3 抽样信号的恢复 (内插公式)
% content/04_SourceCoding/topic01_sampling_recovery.tex

\begin{kbox}{信号的无失真恢复:从框图到公式}
    抽样定理不仅规定了“怎么抽”($f_s \ge 2f_H$),更通过重建原理告诉我们“怎么复原”。
    
    \subsubsection*{1. 抽样与恢复原理框图}
    信号的恢复过程是将抽样信号 $m_s(t)$ 通过一个理想低通滤波器 $H_L(f)$。
    
    \begin{center}
    \resizebox{\linewidth}{!}{
    \begin{tikzpicture}[node distance=2.0cm, auto, >=Latex]
        % 样式定义
        \tikzstyle{block} = [draw, rectangle, minimum height=1.2cm, minimum width=2.0cm, align=center, thick, draw=mainblue, fill=subbg]
        \tikzstyle{input} = [coordinate]
        \tikzstyle{output} = [coordinate]

        % 节点
        \node [input] (input) {};
        \node [block, right=1.2cm of input] (multiplier) {$\times$\\ \footnotesize 乘法器};
        \node [input, below=0.8cm of multiplier] (pulse) {};
        \node [block, right=1.5cm of multiplier] (lpf) {理想低通\\ $H_L(f)$};
        \node [output, right=1.2cm of lpf] (output) {};

        % 连线
        \draw [->, thick] (input) -- node [above] {$m(t)$} (multiplier);
        \draw [->, thick] (pulse) -- node [right] {$\delta_T(t)$} (multiplier);
        \draw [->, thick] (multiplier) -- node [above] {$m_s(t)$} (lpf);
        \draw [->, thick] (lpf) -- node [above] {$m(t)$} (output);
    \end{tikzpicture}
    }
    \end{center}

    \begin{itemize}
        \item \textbf{频域关系}:抽样信号频谱 $M_s(f)$ 是 $M(f)$ 的周期延拓。
        \begin{equation}
            M_s(f) = \frac{1}{T_s} \sum_{n=-\infty}^{\infty} M(f - n f_s)
        \end{equation}
        \item \textbf{恢复条件}:利用理想低通滤波器 $H_L(f)$ 滤出基带分量。
        \begin{equation}
            H_L(f) = \begin{cases} 
                T_s, & |f| \le f_H \\
                0, & |f| > f_H 
            \end{cases}
            \quad (\text{理想低通})
        \end{equation}
    \end{itemize}

        % --- 2. 滤波器增益的直观理解 ---
        \begin{intuitionbox}{思考:为什么滤波器增益是 $T_s$ (即 $1/f_s$)?}
            \textbf{问题}:一般的理想低通滤波器增益通常为1,但为何重建时 $H_L(f)$ 的通带增益必须是 $T_s$?
            
            \textbf{笔记解析}:
            \begin{itemize}
                \item 在理想抽样后的频谱 $M_s(f)$ 中,包含了幅度因子 $f_s$(即 $1/T_s$)。
                \item 为了\textcolor{alertred}{恢复原始信号的真实幅值},滤波器的增益必须抵消这个因子。
            \end{itemize}
            
            \textbf{结论}:
            \begin{equation}
                H_L(f) = \begin{cases} 
                    T_s, & |f| \le f_H \\
                    0, & |f| > f_H 
                \end{cases}
                \quad \text{其中 } T_s = \frac{1}{f_s}
            \end{equation}
        \end{intuitionbox}


    \subsubsection*{2. 重建滤波器的时域推导}
    \textbf{目标}:求解理想低通滤波器 $H_L(f)$ 的时域冲击响应 $h_L(t)$。
    
    重建在频域表示为 $M_s(f) \cdot H_L(f)$,对应时域卷积。我们需要找到频域门函数对应的时域信号。推导过程严格如下:

    \textbf{Step 1: 引入基础变换对}
    设 $g_{2\tau}(t)$ 为时域宽度为 $2\tau$ 的门函数。已知其傅里叶变换为:
    \begin{equation}
        g_{2\tau}(t) \longleftrightarrow 2\tau \Sa(\omega \tau)
    \end{equation}
    其中 $\omega = 2\pi f$。

    \textbf{Step 2: 利用对偶性 (Duality)}
    根据傅里叶变换的对偶性,并整理系数,可以得到 $Sa$ 函数的变换对:
    \begin{equation}
        \Sa(\tau t) \longleftrightarrow \frac{\pi}{\tau} g_{2\tau}(\omega)
    \end{equation}
    移项整理,将幅度系数移至左侧:
    \begin{equation}
        \frac{\tau}{\pi} \Sa(\tau t) \longleftrightarrow g_{2\tau}(\omega)
    \end{equation}

    \textbf{Step 3: 频域变量代换与参数匹配}
    将频域变量由 $\omega$ 转换为 $f$(注意 $\omega$ 域的宽度 $2\tau$ 对应 $f$ 域的宽度为 $\frac{2\tau}{2\pi} = \frac{\tau}{\pi}$):
    \begin{equation}
        \frac{\tau}{\pi} \Sa(\tau t) \longleftrightarrow g_{\frac{\tau}{\pi}}(f)
    \end{equation}
    
    此时,我们需要对应到理想低通滤波器 $H_L(f)$。$H_L(f)$ 是一个截止频率为 $f_H$ 的门函数,其总带宽宽度为 $2f_H$。
    
    \begin{equation}
        \boxed{ \frac{\tau}{\pi} = 2f_H } \quad \Longrightarrow \quad \tau = 2\pi f_H
    \end{equation}

    \textbf{Step 4: 代入得出结论}
    将 $\tau = 2\pi f_H$ 代入 Step 3 的左侧公式:
    \begin{equation}
        2f_H \Sa(2\pi f_H t) \longleftrightarrow g_{2f_H}(f)
    \end{equation}
    这正是截止频率为 $f_H$ 的理想低通滤波器 $H_L(f)$。
    
    因此,时域冲击响应为:
    \begin{equation}
        h_L(t) = 2f_H \Sa(2\pi f_H t)
    \end{equation}
    或者写作 Sinc 形式($\Sa(x) = \text{sinc}(x/\pi)$):
    \begin{equation}
        h_L(t) = 2f_H \text{sinc}(2 f_H t)
    \end{equation}

    \textbf{问题}:为什么频域的理想矩形滤波器,时域对应 Sinc 插值?
    
    根据手写笔记,我们利用傅里叶变换的\textbf{对偶性}和\textbf{尺度变换}性质进行详细推导。

    \textbf{Step 1: 基础变换对 (门函数 $\leftrightarrow$ Sa函数)}
    设 $g_{2\tau}(t)$ 为宽度为 $2\tau$ 的门函数(矩形脉冲)。
    \begin{equation}
        g_{2\tau}(t) \longleftrightarrow 2\tau \Sa(\omega \tau), \quad \text{其中 } \Sa(x) = \frac{\sin x}{x}
    \end{equation}
    
    \textbf{Step 2: 利用对偶性}
    利用对偶性 $X(t) \leftrightarrow 2\pi x(-\omega)$,并考虑到门函数是偶函数:
    \begin{equation}
        2\tau \Sa(t \tau) \longleftrightarrow 2\pi g_{2\tau}(\omega)
    \end{equation}
    
    \textbf{Step 3: 尺度变换与代入}
    我们的目标是求 $H_L(f)$ 的时域形式。$H_L(f)$ 是频域上宽度为 $2f_H$ 的门函数。
    \begin{itemize}
        \item 令频域门宽参数对应截止频率:$\tau \rightarrow \pi f_H$ (因 $\omega=2\pi f$)。
        \item 或者更直接地,对 $H_L(f) = \text{rect}(f/2f_H)$ 进行逆变换:
    \end{itemize}
    \begin{equation}
        h_L(t) = \int_{-f_H}^{f_H} 1 \cdot e^{j2\pi f t} df = \frac{e^{j2\pi f_H t} - e^{-j2\pi f_H t}}{j2\pi t}
    \end{equation}
    化简得到结论:
    \begin{equation}
        h_L(t) = \frac{\sin(2\pi f_H t)}{\pi t} = 2f_H \Sa(2\pi f_H t)
    \end{equation}
    
    当 $f_s = 2f_H$ 且取增益 $T_s = 1/f_s$ 时:
    \begin{equation}
        h(t) = T_s \cdot 2f_H \Sa(2\pi f_H t) = \Sa\left(\frac{\pi t}{T_s}\right) = \text{sinc}\left(\frac{t}{T_s}\right)
    \end{equation}

    \subsubsection*{3. 时域内插公式与几何解释}
    根据卷积定理,$m(t) = m_s(t) * h(t)$。
    将 $m_s(t) = \sum m(nT_s)\delta(t-nT_s)$ 代入,得到 \textbf{Whittaker–Shannon 插值公式}:
    
    \begin{equation}
        \boxed{ m(t) = \sum_{n=-\infty}^{\infty} m(nT_s) \cdot \text{sinc}\left( \frac{t - nT_s}{T_s} \right) }
    \end{equation}

    \textbf{图解}:原信号是无穷多个\textbf{移位 Sinc 函数}的线性叠加。

    \vspace{0.5em}
    \centering
    % === Sinc 插值叠加图 (复现原有代码) ===
    \resizebox{\linewidth}{!}{
    \begin{tikzpicture}[>=Latex]
        % --- 样式定义 ---
        \definecolor{myblue}{RGB}{0, 114, 189}
        \definecolor{myred}{RGB}{217, 83, 25}
        \tikzstyle{stem} = [thick, myblue, ->]
        \tikzstyle{sinc} = [thin, gray!70, samples=100, smooth]
        \tikzstyle{sum} = [line width=1.5pt, myred, samples=100, smooth] 
        
        % --- 坐标轴 ---
        \draw[->, thick] (-3.8, 0) -- (3.8, 0) node[right] {$t$};
        \draw[->, thick] (0, -0.5) -- (0, 1.5) node[above] {$m(t)$};

        % --- 定义参数 ---
        \def\valA{0.2}  % n = -2
        \def\valB{0.8}  % n = -1
        \def\valC{1.0}  % n = 0
        \def\valD{0.8}  % n = 1
        \def\valE{0.2}  % n = 2
        
        % --- 辅助函数 ---
        \tikzset{
            declare function={
                sinc(\t) = (abs(\t) < 0.001) ? 1 : sin(180*\t)/(3.14159*\t);
            }
        }

        % --- 1. 绘制 Sinc 分量 ---
        \foreach \n/\val in {-2/\valA, -1/\valB, 0/\valC, 1/\valD, 2/\valE} {
            \draw[sinc] plot[domain=-3.5:3.5] (\x, {\val * sinc(\x - \n)});
        }

        % --- 2. 绘制合成曲线 ---
        \draw[sum] plot[domain=-3.5:3.5] (\x, {
            \valA * sinc(\x + 2) + \valB * sinc(\x + 1) +
            \valC * sinc(\x) + \valD * sinc(\x - 1) + \valE * sinc(\x - 2)
        });

        % --- 3. 绘制抽样点 ---
        \foreach \n/\val in {-2/\valA, -1/\valB, 0/\valC, 1/\valD, 2/\valE} {
            \draw[stem] (\n, 0) -- (\n, \val);
            \fill[myblue] (\n, \val) circle (2pt);
            
            \ifnum\n=0 \node[below, font=\small] at (\n, 0) {$0$};
            \else \ifnum\n=1 \node[below, font=\small] at (\n, 0) {$T_s$};
            \else \ifnum\n=-1 \node[below, font=\small] at (\n, 0) {$-T_s$};
            \else \node[below, font=\small] at (\n, 0) {$\n T_s$};
            \fi \fi \fi
        }

        % --- 标注 ---
        \node[myred, right, font=\bfseries] at (1.5, 1.2) {重构信号};
        \node[gray, right, font=\footnotesize] at (2.4, 0.3) {Sinc分量};
        \draw[->, thin, gray] (2.4, 0.3) -- (1.8, 0.2);
    \end{tikzpicture}
    }


    
\end{kbox}

% 1.4 [新增] 实际抽样:平顶抽样与孔径效应
% (这是基于你手写笔记新增的详细推导部分,建议放在恢复之后,引入实际工程问题)
% content/topic02_sampling_flattop.tex
% 
% 重点:平顶抽样 (Flat-top Sampling) 与孔径效应 (Aperture Effect)

% 确保在导言区定义了 sinc
\providecommand{\sinc}{\operatorname{sinc}}

% --- Part 1: 系统模型图解 ---
\begin{kbox}{平顶抽样 (Flat-top Sampling) 系统模型}
    与理想抽样不同,实际抽样输出的脉冲具有一定的宽度 $\tau$。这通常通过\textbf{脉冲保持电路 (Zero-order Hold)} 来实现。
    
    \vspace{5pt}
    \centering
    \begin{tikzpicture}[auto, >=Latex, node distance=1.5cm]
        % 样式定义
        \tikzstyle{block} = [draw, rectangle, fill=blue!5, minimum height=2em, minimum width=3em, rounded corners]
        \tikzstyle{sum} = [draw, circle, inner sep=1pt]
        \tikzstyle{input} = [coordinate]
        \tikzstyle{output} = [coordinate]

        % 节点定义
        \node [input] (input) {};
        \node [sum, right=1cm of input] (mixer) {$\times$};
        \node [above=0.6cm of mixer, text=red] (pulse) {$\delta_{T_s}(t)$};
        \node [block, right=1.5cm of mixer, align=center] (hold) {\textbf{脉冲保持}\\ $h_{Hold}(t)$};
        \node [output, right=1.5cm of hold] (output) {};

        % 连线与标注
        \draw [->, thick] (input) -- node {$m(t)$} (mixer);
        \draw [->, thick] (pulse) -- (mixer);
        \draw [->, thick] (mixer) -- node {$m_s(t)$} (hold);
        \draw [->, thick] (hold) -- node {$m_H(t)$} (output);
        
        % 备注
        \node [below=0.2cm of mixer, align=center, font=\scriptsize, gray] {理想抽样\\(冲激串)};
        \node [below=0.2cm of hold, align=center, font=\scriptsize, gray] {线性系统\\(门电路)};
        \node [below=0.2cm of output, align=center, font=\scriptsize, gray] {平顶脉冲\\序列};
    \end{tikzpicture}
\end{kbox}

% --- Part 2: 时频域详细对照表 (核心笔记还原) ---
\begin{kbox}{理想抽样 vs 平顶抽样:时频域对照表}
    \centering
    \renewcommand{\arraystretch}{1.5}
    % [修改] 使用 resizebox 强制将表格缩放到栏宽
    \resizebox{\linewidth}{!}{
        \begin{tabular}{|c|l|l|}
            \hline
            \rowcolor{subbg} 
            \textbf{阶段} & \textbf{时域表达式 (Time)} & \textbf{频域表达式 (Freq)} \\ 
            \hline
            
            % 第一行:理想抽样
            \textbf{1. 理想抽样} & 
            $\begin{aligned} m_s(t) &= m(t) \cdot \delta_T(t) \\ &= \sum m(nT_s)\delta(t-nT_s) \end{aligned}$ & 
            $\begin{aligned} M_s(f) &= M(f) * \Delta_T(f) \\ &= f_s \sum M(f-nf_s) \end{aligned}$ \\ 
            \hline
            
            % 第二行:保持滤波器
            \textbf{2. 脉冲保持} & 
            $h_{Hold}(t) = \text{rect}(t/\tau)$ & 
            $H_{Hold}(f) = A\tau \sinc(f\tau) e^{-j\pi f \tau}$ \\ 
            \hline
            
            % 第三行:系统输出
            \textbf{3. 平顶输出} & 
            $\begin{aligned} m_H(t) &= m_s(t) * h_{Hold}(t) \\ &= \sum m(nT_s) h(t-nT_s) \end{aligned}$ & 
            $\begin{aligned} M_H(f) &= M_s(f) \cdot H_{Hold}(f) \\ &\text{(包络受 Sinc 加权)} \end{aligned}$ \\ 
            \hline
        \end{tabular}
    }
\end{kbox}

% --- Part 3: 直观理解与混淆辨析 ---
\begin{intuitionbox}{核心辨析:脉冲保持是“单脉冲”还是“周期序列”?}
    初学者常误认为保持电路本身是周期性的。根据笔记强调:
    
    \begin{itemize}
        \item \textbf{输入信号 $m_s(t)$}:是\textbf{周期性}的冲激序列(理想抽样结果)。
        \item \textbf{系统响应 $h(t)$}:是脉冲保持电路的\textbf{单位冲激响应}。它是一个\textbf{孤立的、单个}矩形门信号(Single Gate),而不是周期门信号。
    \end{itemize}
    
    \textbf{物理图像}:
    卷积 $m_H(t) = m_s(t) * h(t)$ 相当于把“单个印章” $h(t)$,按照 $m_s(t)$ 指定的位置(每隔 $T_s$)和力度(幅度)盖在纸上,从而形成了一串平顶脉冲。
\end{intuitionbox}

% --- Part 4: 详细数学推导 ---
\begin{kbox}{平顶抽样谱与孔径效应推导}

    \subsubsection*{1. 保持电路 $h_{Hold}(t)$ 的定义}
    设 $h_{Hold}(t)$ 为宽度 $\tau$,幅度 $A$ 的矩形脉冲。为了保证因果性(从 $t=0$ 开始),其表达式为:
    \begin{equation}
        h_{Hold}(t) = A \cdot g_\tau \left( t - \frac{\tau}{2} \right) = \begin{cases} A, & 0 \le t \le \tau \\ 0, & \text{其他} \end{cases}
    \end{equation}

    \subsubsection*{2. 频谱 $H_{Hold}(f)$ 的推导}
    利用傅里叶变换的时移性质 $\mathcal{F}\{x(t-t_0)\} = X(f)e^{-j2\pi f t_0}$:
    矩形脉冲 $g_\tau(t) \leftrightarrow \tau \sinc(f\tau)$。
    
    \begin{equation}
    \begin{split}
        H_{Hold}(f) &= A \cdot \left[ \tau \sinc(f\tau) \right] \cdot e^{-j2\pi f \cdot \frac{\tau}{2}} \\
                    &= A\tau \cdot \sinc(f\tau) \cdot e^{-j\pi f \tau}
    \end{split}
    \end{equation}
    \textbf{注}:此处 $\sinc(x) = \frac{\sin(\pi x)}{\pi x}$。相移因子 $e^{-j\pi f \tau}$ 仅代表线性相位延迟,不影响幅度谱形状。

    \subsubsection*{3. 输出信号频谱 $M_H(f)$}
    将 $M_s(f)$ 和 $H_{Hold}(f)$ 相乘:
    \begin{equation}
        M_H(f) = \left[ f_s \sum_{n=-\infty}^{+\infty} M(f-nf_s) \right] \cdot \left[ A\tau \sinc(f\tau) e^{-j\pi f \tau} \right]
    \end{equation}
    
    只关注幅度谱(忽略相位延迟):
    \begin{equation}
        \boxed{ |M_H(f)| = \underbrace{\frac{A\tau}{T_s}}_{\text{增益}} \cdot \left| \sum_{n=-\infty}^{+\infty} M(f-nf_s) \right| \cdot \underbrace{|\sinc(f\tau)|}_{\text{孔径失真}} }
    \end{equation}

    \tcblower
    
    \subsubsection*{4. 孔径效应 (Aperture Effect)}
    由于引入了 $\sinc(f\tau)$ 项,信号频谱在高频处会发生衰减(不仅是高频噪声,\textbf{有用信号的高频分量也会被衰减})。
    
    \begin{center}
    \begin{tikzpicture}
        \begin{axis}[
            % --- 尺寸调整 ---
            width=0.95\linewidth, 
            height=6cm, 
            % --- 坐标轴设置 ---
            axis lines=middle, 
            xmin=-0.2, xmax=2.8, 
            ymin=0, ymax=1.4, 
            % --- 标签内容 ---
            xlabel={频率 $f$},
            ylabel={幅度 $|H(f)|$},
            % --- 刻度设置 ---
            xtick={1, 2},
            xticklabels={$\frac{1}{\tau}$, $\frac{2}{\tau}$},
            ytick={1},
            yticklabels={$A\tau$},
            % --- 标签位置微调 ---
            every axis x label/.style={at={(current axis.right of origin)}, anchor=west},
            every axis y label/.style={at={(current axis.above origin)}, anchor=south},
            % --- 绘图数据 ---
            samples=200, 
            domain=0.01:2.6,
            clip=false 
        ]
            % 1. 绘制 Sinc 函数曲线 (主蓝色)
            \addplot [thick, mainblue] {abs(sin(deg(pi*x))/(pi*x))};

            % 2. 绘制理想保持线 (灰色虚线)
            \draw[dashed, gray, thick] (axis cs:0, 1) -- (axis cs:2.6, 1);
            % 文字放在线的上方,且靠右
            \node[gray, font=\small] at (axis cs: 2.0, 1.1) {理想保持 (无衰减)};
            
            % 3. 绘制衰减指示 (红色双向箭头)
            \draw[<->, red, thick] (axis cs:0.6, 0.98) -- (axis cs:0.6, 0.48);
            % 文字放在箭头右侧
            \node[red, align=left, font=\small, anchor=west] at (axis cs: 0.65, 0.75) {高频\\严重衰减};

            % 4. 补充零点提示
            \node[font=\scriptsize, fill=white, inner sep=1pt] at (axis cs:1, 0.15) {零点};
        \end{axis}
    \end{tikzpicture}
    \end{center}
    
    \begin{itemize}
        \item \textbf{现象}:抽样脉冲宽度 $\tau$ 越大,Sinc 函数主瓣越窄,高频衰减越严重。
        \item \textbf{对策}:在接收端使用\textbf{均衡器} $H_{eq}(f) = \frac{1}{\sinc(f\tau)}$ 进行补偿。
    \end{itemize}
\end{kbox} % 注意:确保你保存的文件名是这个,或者改为 topic01_sampling_flattop 以保持编号一致


% --- 以下旧文件建议检查是否重复,若已在 ideal 中涵盖则注释掉 ---
% % content/04_SourceCoding/topic01_sampling.tex

\begin{kbox}{抽样定理 (Sampling Theorem)}
    \begin{itemize}
        \item \textbf{低通模拟信号抽样定理} \pptpage{12}
        \begin{itemize}
            \item \textbf{条件}:为了无失真恢复原信号,抽样速率 $f_s$ 需满足:
            \begin{equation}
                f_s \ge 2f_H \quad \text{且} \quad T_s \le \frac{1}{2f_H}
            \end{equation}
            其中 $f_H$ 为信号最高频率。
            \item \textbf{奈奎斯特速率}:$f_s = 2f_H$
            \item \textbf{奈奎斯特间隔}:$T_s = \frac{1}{2f_H}$
        \end{itemize}
        
        \item \textbf{理想抽样 (Ideal Sampling)} \pptpage{13}
        \begin{itemize}
            \item \textbf{时域}:利用单位冲激序列 $\delta_T(t)$ 进行乘积:
            % --- 修正:使用 split 环境将长公式换行对齐 ---
            \begin{equation}
            \begin{split}
                m_s(t) &= m(t)\sum_{n=-\infty}^{\infty}\delta(t-nT_s) \\
                       &= \sum_{n=-\infty}^{\infty}m(nT_s)\delta(t-nT_s)
            \end{split}
            \end{equation}
          
            \item \textbf{频域}:频谱以 $f_s$ 为周期搬移:
            \begin{equation}
                M_s(f) = \frac{1}{T_s}\sum_{n=-\infty}^{\infty}M(f-nf_s)
            \end{equation}
        \end{itemize}

        \item \textbf{平顶抽样与孔径失真} \pptpage{26}
        \begin{itemize}
            \item \textbf{产生}:理想抽样脉冲经过形状为 $h(t)$ 的保持电路。
            
            \item \textbf{频域关系}:
            \begin{equation}
                M_H(f) = M_s(f) \cdot H(f)
            \end{equation}
            \item \textbf{孔径失真}:由 $H(f) = T_s \text{Sa}(\pi f T_s)$ (矩形脉冲) 引起的高频衰减,需在接收端使用 \textbf{修正滤波器} 进行补偿。
        \end{itemize}
    \end{itemize}
\end{kbox}          % 原抽样定理 (已融入 topic01_sampling_ideal)
% % content/04_SourceCoding/topic01_sampling_intuition.tex

\begin{intuitionbox}{抽样定理的直观解释:操场跑步与照相机}
    有些初学者难以理解为什么“离散采样会导致频域周期延拓”,以及为什么“采样频率必须大于两倍”。我们可以用“操场跑步拍照”的例子来直观解释。

    \subsubsection*{1. 为什么时域离散 $\to$ 频域周期延拓?}
    \begin{itemize}
        \item \textbf{场景设定}:
        \begin{itemize}
            \item \textbf{信号相位 $\leftrightarrow$ 操场跑道}:假设操场是一个圆圈,人在上面跑。
            \item \textbf{信号幅度/位置 $\leftrightarrow$ 人所在的位置}:你看到的照片里人在哪里。
            \item \textbf{采样 $\leftrightarrow$ 拍照}:你手里有一台相机,每分钟按下一次快门。
        \end{itemize}
        
        \item \textbf{无法区分的圈数 (混叠)}:
        假设甲的速度是 \textbf{0.3 圈/分钟}。每过一分钟你拍一张照,甲相对于上一张向前移动了 0.3 圈。
        但是,如果乙的速度是 \textbf{1.3 圈/分钟},每过一分钟他跑了一整圈多 0.3 圈,照片里他的位置和甲\textbf{完全一样}。
        同理,2.3 圈/分钟、3.3 圈/分钟的人,拍出来的照片位置都重叠了。
    \end{itemize}

    \begin{center}
    \begin{tikzpicture}[scale=0.9, >=Latex]
        % 跑道
        \draw[thick, gray!50] (0,0) circle (2.0cm);
        
        % 辅助线
        \draw[gray, dashed] (0,0) -- (90:2.0) node[above, text=black] {\small 起点 (0)};
        \node[text=gray] at (0,0) {位置重叠};
        
        % 甲:0.3圈
        \draw[->, line width=1.5pt, mainblue] (90:2.1) arc (90:-18:2.1);
        \node[mainblue, right] at (45:2.2) {甲: 0.3圈};
        
        % 乙:1.3圈
        \draw[->, line width=1.5pt, red] (90:1.8) arc (90:-270:1.8) coordinate (aux);
        \draw[line width=1.5pt, red] (aux) arc (90:-18:1.8);
        \node[red, left] at (180:1.8) {乙: 1.3圈};
        
        % 终点
        \filldraw [black] (-18:2.0) circle (3pt);
        \node[right, font=\small, yshift=-0.3cm] at (-18:2.0) {拍照时刻位置};
    \end{tikzpicture}
    \end{center}

    \subsubsection*{2. 为什么采样频率必须大于两倍 ($f_s > 2f_H$)?}
    采样定理要求 $f_s > 2f_H$,本质是因为频率不仅有大小,还有\textbf{方向}(正负频率)。照片只能记录“位置”,大脑在重建信号时,默认物体走的是\textbf{近道}(移动不超过半圈)。
    
    \vspace{0.5em}
    \textbf{场景分析}:假设某人真实速度为 \textbf{顺时针 1.3 圈/分钟} ($f_H=1.3$)。

    \vspace{1em}
    \begin{center}
    % 使用 scale 调整整体大小,防止垂直方向过长
    \begin{tikzpicture}[
        scale=0.85, transform shape,
        clock/.style={circle, draw=black!60, very thick, minimum size=3.5cm},
        hand/.style={-Latex, line width=2pt, line cap=round},
        dot/.style={fill=red, circle, inner sep=2.5pt},
        textbox/.style={text width=9cm, align=center, font=\small, anchor=north}, % 文字居中
        title/.style={font=\bfseries\large}
    ]
        % 定义垂直间距 (根据时钟大小和文字高度调整)
        \def\vdist{7.5}

        % ====== 情况一:采样太慢 (混叠) ======
        \begin{scope}[yshift=0cm]
            % 时钟
            \node[clock] (c1) at (0,0) {};
            \node[above] at (c1.north) {12点};
            \node[below] at (c1.south) {6点};
            
            % 路径
            \draw[hand, mainblue!40, dashed] (90:1.4) arc (90:-144:1.4); 
            \node[mainblue!60, font=\scriptsize] at (0,0.5) {真实: 0.65圈};
            
            \draw[hand, red] (90:1.75) arc (90:216:1.75);
            \node[red, font=\footnotesize, left] at (150:1.9) {观测: -0.35圈};
            \node[dot] at (216:1.75) {};

            % 下方文字
            \node[textbox] at (0, -2.2) {
                \textbf{\textcolor{red}{1. 采样太慢 ($f_s < 2f_H$)}}\\
                $f_s=2$ (每0.5min拍一张)\\
                \vspace{0.2cm}
                \textbf{混叠错误}:真实跑了\textbf{过半圈} (0.65),但因为拍照间隔太长,
                被误判为\textbf{倒着跑}小圈 (-0.35)。\\
                结果:$1.3 \to -0.7$ (方向错了)
            };
        \end{scope}

        % ====== 情况二:临界采样 (方向丢失) ======
        \begin{scope}[yshift=-\vdist cm]
            \node[clock] (c2) at (0,0) {};
            \node[above] at (c2.north) {12点};
            \node[below] at (c2.south) {6点};
            
            \draw[hand, mainblue, dashed] (91:1.5) arc (91:-89:1.5);
            \draw[hand, mainblue, dashed] (89:1.5) arc (89:269:1.5);
            \node[dot] at (270:1.75) {}; 
            
            \node[textbox] at (0, -2.2) {
                \textbf{\textcolor{orange}{2. 临界采样 ($f_s = 2f_H$)}}\\
                $f_s=2.6$ (每0.38min拍一张)\\
                \vspace{0.2cm}
                \textbf{方向丢失}:刚好跑了半圈 (到达6点)。\\
                此时顺时针跑半圈和逆时针跑半圈落点一样,\textbf{分不清方向}。
            };
        \end{scope}

        % ====== 情况三:采样足够快 (无失真) ======
        \begin{scope}[yshift=-2*\vdist cm]
            \node[clock] (c3) at (0,0) {};
            \node[above] at (c3.north) {12点};
            \node[below] at (c3.south) {6点};
            
            \draw[hand, green!50!black] (90:1.5) arc (90:-27:1.5);
            \node[green!50!black, font=\footnotesize] at (30:0.8) {< 0.5圈};
            \node[dot] at (-27:1.75) {};
            
            \node[textbox] at (0, -2.2) {
                \textbf{\textcolor{green!40!black}{3. 采样足够快 ($f_s > 2f_H$)}}\\
                $f_s=4$ (每0.25min拍一张)\\
                \vspace{0.2cm}
                \textbf{唯一确定}:两次拍照之间移动\textbf{少于半圈}。\\
                根据“近道原则”,可以唯一确定他是顺时针跑过来的。
            };
        \end{scope}
    \end{tikzpicture}
    \end{center}

    \begin{quote}
        \textbf{总结}:为了不让正频率(顺时针)和负频率(逆时针)发生混淆,采样频率必须足够快,保证两次拍照之间物体移动\textbf{不超过半圈}。这就是 $f_s > 2f_{max}$ 的本质。
    \end{quote}
\end{intuitionbox}% 原直观理解 (已融入 topic01_sampling_ideal 的 intuitionbox)


% --- 2. 量化 (Quantization) ---
% 2.1 量化理论基础 (均匀量化、非均匀量化)
% content/04_SourceCoding/topic02_quantization.tex

\begin{kbox}{量化 (Quantization)}
    \begin{itemize}
        \item \textbf{均匀量化 (Uniform Quantization)} \pptpage{32}
        \begin{itemize}
            \item \textbf{量化间隔}:$\Delta v = \frac{b-a}{M}$
            \item \textbf{量化噪声功率} (重要结论):
            \begin{equation}
                N_q = \frac{(\Delta v)^2}{12}
            \end{equation}
            \item \textbf{信号量噪比 ($S/N_q$)} \pptpage{34}:
            % 使用 multline 确保长公式能够适应窄栏
            \begin{multline}
                (S/N_q)_{\text{dB}} \approx  4.8 + 6n + 10\lg(S/x_{max}^2)
            \end{multline}
            \textbf{结论}:编码位数 $n$ 每增加 1 bit,信噪比提高约 \textbf{6dB}。
        \end{itemize}

        \item \textbf{非均匀量化 (Non-uniform)} \pptpage{38}
        \begin{itemize}
            \item \textbf{原理}:先压缩 (Compress),再均匀量化,最后扩张 (Expand)。目的:提高\textbf{小信号}的量噪比。
            \item \textbf{A律压缩特性} (A-Law, 欧/中标准, $A=87.6$) \pptpage{44}:
            % 微调格式,确保分数在窄栏中不溢出
            \begin{equation}
                y = \begin{cases} 
                \frac{Ax}{1+\ln A}, & 0 < x \le \frac{1}{A} \\[6pt] 
                \frac{1+\ln(Ax)}{1+\ln A}, & \frac{1}{A} \le x \le 1 
                \end{cases}
            \end{equation}
            \item \textbf{$\mu$律压缩特性} (美/日标准, $\mu=255$):
            \begin{equation}
                y = \frac{\ln(1+\mu x)}{\ln(1+\mu)}, \quad 0 \le x \le 1
            \end{equation}
            \item \textbf{改善度}:$[Q]_{\text{dB}} = 20\lg y'$,即取决于压缩曲线斜率。
        \end{itemize}
    \end{itemize}
\end{kbox} 


% --- 3. 编码 (Coding / PCM) ---
% 3.1 脉冲编码调制 (PCM原理与系统框图)
% content/04_SourceCoding/topic03_pcm.tex

\begin{kbox}{脉冲编码调制 (PCM) 系统原理}
    \begin{itemize}
        \item \textbf{PCM原理框图}
        \begin{center}
        \resizebox{\linewidth}{!}{
        \begin{tikzpicture}[
            node distance=1.5cm,
            blk/.style={rectangle, draw=blue!80!black, thick, fill=blue!5, minimum height=1cm, align=center, font=\small},
            txt/.style={font=\small},
            arrow/.style={-Latex, thick, blue!80!black}
        ]
            % --- 发端 (A/D) ---
            \node[blk] (src) {模拟信源};
            
            % [修改] 显著增加间距到 2.2cm,确保 x(t) 远离红色虚线
            \node[blk, right=2.2cm of src] (lpf) {预滤波器\\$(0, f_H)$};
            \node[blk, right=0.8cm of lpf] (samp) {抽样};
            \node[blk, right=0.8cm of samp] (enc) {波形编码器:\\量化、编码};
            
            % 信号标注 (此时位于长箭头的中间,完全在虚线框外)
            \draw[arrow] (src) -- node[above]{$x(t)$} (lpf);
            \draw[arrow] (lpf) -- (samp);
            \draw[arrow] (samp) -- node[above]{$x(n)$} (enc);

            % 发端虚线框
            \node[draw=red, dashed, thick, inner sep=10pt, fit=(lpf) (samp) (enc), label={[red]above:发端 (A/D)}] (sender) {};
            
            % 黄色注释:抗混叠
            \node[fill=yellow!30, above=1.0cm of lpf, font=\footnotesize] (note1) {抗混叠、频带失真};
            \draw[green!60!black, ->] (note1) -- (lpf.north);

            % --- 信道 ---
            \node[blk, below=1.2cm of enc] (chn) {数字信道};
            \draw[arrow, double] (enc) -- (chn);

            % --- 收端 (D/A) ---
            \node[blk, below=1.2cm of chn] (dec) {波形解码器};
            \node[blk, left=0.8cm of dec] (rec) {重建滤波器:\\$x/\sin x$、低通};
            
            % [修改] 显著增加间距到 2.2cm
            \node[blk, left=2.2cm of rec] (dest) {模拟终端};

            % 收端信号
            \draw[arrow, double] (chn) -- (dec);
            \draw[arrow] (dec) -- node[above]{$\hat{x}(n)$} (rec);
            \draw[arrow] (rec) -- node[above]{$\hat{x}(t)$} (dest);

            % 收端虚线框
            \node[draw=red, dashed, thick, inner sep=10pt, fit=(dec) (rec), label={[red]below:收端 (D/A)}] (receiver) {};

            % 黄色注释:频率补偿
            \node[fill=yellow!30, below=1.0cm of rec, font=\footnotesize] (note2) {频率补偿 (孔径失真)};
            \draw[green!60!black, ->] (note2) -- (rec.south);

        \end{tikzpicture}
        }
        \end{center}

        \item \textbf{关键模块功能}
        \begin{itemize}
            \item \textbf{预滤波器}:限制带宽至 $f_H$,防止抽样\textbf{混叠}。
            \item \textbf{重建滤波器}:包含 $x/\sin x$ 滤波器(补偿平顶抽样引起的\textbf{孔径失真})和低通滤波器。
        \end{itemize}

        \item \textbf{基本参数计算} \pptpage{72}
        设模拟信号最高频率为 $f_H$,抽样率为 $f_s$,编码位数为 $N$。
        \begin{itemize}
            \item \textbf{信息传输速率 (比特率)}:
            \begin{equation}
                R_b = f_s \cdot N \ge 2f_H \cdot N
            \end{equation}
            \item \textbf{传输带宽} (第一零点带宽):
            \begin{equation}
                B = R_b = f_s \cdot N \quad (\text{NRZ矩形脉冲})
            \end{equation}
        \end{itemize}

        \item \textbf{典型应用:数字电话}
        \begin{itemize}
            \item \textbf{参数设定}:
            \begin{itemize}
                \item 话音带宽 $B_{voice} \approx 3.4\text{kHz} \to$ 取抽样率 $f_s = 8000\text{Hz}$。
                \item 量化位数 $N=8$ (采用A律13折线编码)。
            \end{itemize}
            \item \textbf{系统比特率}:
            \[ R_b = 8000 \times 8 = \mathbf{64\text{kbps}} \]
            \item \textbf{E1载波 (PCM 30/32路)}:
            \[ R_{E1} = 64\text{kbps} \times 32 = 2.048\text{Mbps} \]
        \end{itemize}
    \end{itemize}
\end{kbox}

\end{document}