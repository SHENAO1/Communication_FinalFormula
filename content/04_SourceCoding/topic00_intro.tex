% content/04_SourceCoding/topic00_intro.tex

\begin{kbox}{信源编码概述}
    \begin{itemize}
        \item \textbf{信源编码的两大作用}
        \begin{enumerate}
            \item \textbf{压缩编码}:减少冗余,提高有效性。
            \item \textbf{模/数转换 (A/D)}:将模拟信号转换为数字信号。
        \end{enumerate}

        \item \textbf{为什么要数字化?}
        \begin{itemize}
            \item 数字通信系统已成为通信的发展方向。
            \item 自然界的许多信息(如声音、图像)经传感器感知后都是模拟量,必须经处理才能进入数字系统。
        \end{itemize}

        \item \textbf{A/D转换 (数字化编码) 的技术分类}
        \begin{itemize}
            \item \textbf{波形编码} (Waveform Coding):如 PCM, DPCM, $\Delta$M。特点是还原度高,但码率较高。
            \item \textbf{参量编码} (Parametric Coding):提取特征参数(如声码器)。
            \item \textbf{混合编码} (Hybrid Coding):结合上述两者。
        \end{itemize}
    \end{itemize}
\end{kbox}

\begin{intuitionbox}{模拟信号数字化传输的三个环节}
    根据,数字化传输包含以下全过程:
    
    \begin{center}
    % --- 修改说明:使用了 resizebox 确保图片自动缩放以适应栏宽 ---
    \resizebox{\linewidth}{!}{
    \begin{tikzpicture}[
        % --- 修改说明:将水平间距(第二个参数)从 0.8cm 减小到 0.45cm,缩短箭头 ---
        node distance=1.2cm and 0.45cm,
        box/.style={rectangle, draw=mainblue, fill=white, thick, minimum height=0.8cm, minimum width=1.5cm, align=center, font=\small},
        arrow/.style={-Latex, thick, mainblue},
        labeltext/.style={font=\scriptsize\color{gray}, align=center}
    ]
        % 节点
        \node[box, fill=orange!20] (src) {模拟\\信号源};
        \node[box, fill=red!10, right=of src] (ad) {A/D\\变换};
        \node[box, fill=green!10, right=of ad] (sys) {数字\\通信系统};
        \node[box, fill=red!10, right=of sys] (da) {D/A\\变换};
        \node[coordinate, right=of da] (end) {};

        % 连线
        \draw[arrow] (src) -- (ad);
        \draw[arrow] (ad) -- (sys);
        \draw[arrow] (sys) -- (da);
        \draw[arrow] (da) -- (end);

        % 下方标注 (模拟/数字序列)
        \node[labeltext, below=0.2cm of src] (l1) {模拟随机\\信号};
        \node[labeltext, below=0.2cm of ad] (l2) {数字随机\\序列};
        \node[labeltext, below=0.2cm of sys] (l3) {数字随机\\序列};
        \node[labeltext, below=0.2cm of da] (l4) {模拟随机\\信号};

        \draw[dashed, gray, ->] (l1) -- (src);
        \draw[dashed, gray, ->] (l2) -- (ad);
        \draw[dashed, gray, ->] (l3) -- (sys);
        \draw[dashed, gray, ->] (l4) -- (da);

    \end{tikzpicture}
    }
    \end{center}

    \textbf{波形编码的三步曲}:
    \begin{equation*}
        \text{模拟信号} \xrightarrow{\textbf{抽样}} \text{抽样信号} \xrightarrow{\textbf{量化}} \text{量化信号} \xrightarrow{\textbf{编码}} \text{数字信号}
    \end{equation*}
\end{intuitionbox}