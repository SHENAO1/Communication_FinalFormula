% content/04_SourceCoding/topic00_intro.tex

\begin{kbox}{信源编码 vs. 信道编码}
    \begin{itemize}
        \item \textbf{定义与核心目的}
        \begin{itemize}
            \item \textbf{信源编码 (Source Coding)}:
            \begin{itemize}
                \item \textbf{目的}:提高通信的\textbf{有效性} (Efficiency)。
                \item \textbf{手段}:\textbf{减少冗余}。通过压缩算法(如哈夫曼、PCM、JPEG)去除原始信息中不必要的成分,降低数据率,节省带宽/存储。
                \item \textbf{核心指标}:压缩比、失真度。
            \end{itemize}
            \item \textbf{信道编码 (Channel Coding)}:
            \begin{itemize}
                \item \textbf{目的}:提高通信的\textbf{可靠性} (Reliability)。
                \item \textbf{手段}:\textbf{增加冗余}。人为添加监督位(校验位),使接收端能发现或纠正传输中的错误(如奇偶校验、CRC、卷积码)。
                \item \textbf{核心指标}:误码率、编码增益。
            \end{itemize}
        \end{itemize}
        
        \item \textbf{对比总结表}
        \begin{center}
        \footnotesize
        \begin{tabular}{|c|c|c|}
            \hline
            \textbf{维度} & \textbf{信源编码} & \textbf{信道编码} \\
            \hline
            \textbf{作用} & “瘦身” (压缩) & “穿甲” (保护) \\
            \hline
            \textbf{冗余操作} & \textbf{去除}冗余 & \textbf{增加}冗余 \\
            \hline
            \textbf{关键目标} & 有效性 (省带宽) & 可靠性 (抗干扰) \\
            \hline
        \end{tabular}
        \end{center}
    \end{itemize}
\end{kbox}

\begin{intuitionbox}{直观理解:冗余的减法与加法}
    \begin{itemize}
        \item \textbf{信源编码 (减法)}:类似于\textbf{写缩写}。
        \par 把 ``For Your Information'' 缩写为 ``FYI''。这去除了多余的字母,让书写和阅读更快(提高了有效性)。
        
        \item \textbf{信道编码 (加法)}:类似于\textbf{打电话时的确认}。
        \par 说 ``我是张三,弓长张,一二三的三''。额外多说的几句话就是“冗余”,目的是为了防止对方听错(提高了可靠性)。
    \end{itemize}
\end{intuitionbox}