% content/04_SourceCoding/topic00_waveform.tex

\begin{kbox}{波形编码的三个步骤 (详解)}
    根据,波形编码(如PCM)将模拟信号转化为数字信号主要经过以下三步:

    \begin{center}
    % 使用 resizebox 确保整体宽度适应双栏
    \resizebox{\linewidth}{!}{
    \begin{tikzpicture}[
        scale=1, 
        every node/.style={transform shape},
        % 定义描述文本框样式
        desc/.style={
            align=left, 
            fill=gray!10, 
            text width=8.5cm, 
            rounded corners, 
            font=\small,
            inner sep=6pt,
            anchor=north
        }
    ]
        
        % --- 第一步:抽样 ---
        \begin{scope}[yshift=0cm]
            \draw[->] (-0.5,0) -- (7.5,0) node[right] {$t$};
            \draw[->] (0,-1.2) -- (0,1.8) node[right] {$m(t), m_s(t)$};
            \draw[gray, thick, dashed] plot[domain=0:7, samples=50] (\x, {sin(\x r * 1.5) + 0.2});
            
            \foreach \x in {0.5, 1.5, ..., 6.5} {
                \draw[blue, thick, -Latex] (\x,0) -- (\x, {sin(\x r * 1.5) + 0.2});
            }
            \node[blue, font=\bfseries] at (6, 1.8) {抽样信号};
            
            \node[desc, fill=blue!5] at (3.5, -1.6) {
                \textbf{1. 抽样 (Sampling)}\\
                $m_s(t)$:\textbf{时间上离散},但取值(幅度)仍然连续。\\
                称为:离散模拟信号。
            };
        \end{scope}

        % --- 第二步:量化 ---
        \begin{scope}[yshift=-6cm] 
            \draw[->] (-0.5,0) -- (7.5,0) node[right] {$t$};
            \draw[->] (0,-1.2) -- (0,1.8) node[right] {$m_q(t)$};
            \foreach \y in {-1, -0.5, 0, 0.5, 1, 1.5} {
                \draw[gray!30, dashed] (0,\y) -- (7,\y);
            }
            
            \foreach \x in {0.5, 1.5, ..., 6.5} {
                \pgfmathsetmacro{\val}{sin(\x r * 1.5) + 0.2}
                \pgfmathsetmacro{\quantized}{round(\val*2)/2} 
                \draw[magenta, line width=2pt] (\x,0) -- (\x, \quantized);
                \draw[magenta, fill=white] (\x, \quantized) circle (2pt);
            }
            \node[magenta, font=\bfseries] at (6, 1.8) {量化信号};

            \node[desc, fill=magenta!5] at (3.5, -1.6) {
                \textbf{2. 量化 (Quantization)}\\
                $m_q(t)$:\textbf{时间上离散},\textbf{取值(幅度)也离散}。\\
                此时脉冲的\textbf{长短不一},直接反映信号幅度大小。
            };
        \end{scope}

        % --- 第三步:编码 ---
        \begin{scope}[yshift=-12cm] 
            \draw[->] (-0.5,0) -- (7.5,0) node[right] {$t$};
            
            \foreach \x in {0.5, 1.5, ..., 6.5} {
                \node[above, font=\scriptsize] at (\x, 1.0) {011}; 
                \draw[mainblue, fill=mainblue] (\x-0.15, 0) rectangle (\x+0.15, 0.7);
            }
            \node[mainblue, font=\bfseries] at (6, 1.8) {编码信号};

            % --- 在此处加入了解释 ---
            \node[desc, fill=mainblue!5] at (3.5, -0.5) {
                \textbf{3. 编码 (Encoding)}\\
                将量化电平映射为二进制码组 (如 010, 011)。\\
                \textbf{注意}:此时脉冲\textbf{高度恒定},仅代表数字逻辑电平(如5V)。原始信号的幅度信息已转化为\textbf{二进制数值},不再由脉冲高度承载。
            };
        \end{scope}
    \end{tikzpicture}
    }
    \end{center}
\end{kbox}