% content/04_SourceCoding/topic01_sampling_classification.tex

\begin{kbox}{抽样技术的分类体系}
    根据手写笔记整理,抽样技术通常可以从以下三个维度进行分类:

    \begin{center}
    \resizebox{\linewidth}{!}{
    \begin{tikzpicture}[
        grow=right,
        level 1/.style={sibling distance=3.5cm, level distance=3.5cm},
        level 2/.style={sibling distance=1.2cm, level distance=4.5cm}, % 稍微增加 level distance 以容纳更长的灰色文字
        edge from parent/.style={draw, thick, gray},
        root/.style={rectangle, rounded corners, draw=mainblue, fill=mainblue!10, text centered, font=\bfseries\large, inner sep=10pt},
        node1/.style={rectangle, rounded corners, draw=blue!80, fill=blue!5, text centered, font=\bfseries, minimum height=0.8cm},
        node2/.style={rectangle, rounded corners, draw=gray!60, fill=white, text centered, font=\small, minimum height=0.6cm},
        % 定义一个统一的 label 样式,方便复用
        note/.style={label={[font=\scriptsize, text=gray, align=left, xshift=0.1cm]right:#1}}
        ]

        % 根节点
        \node[root] {抽样}
            % 分支 3:按脉冲形状
            child {node[node1] {3. 按脉冲形状}
                child {node[node2, note={脉冲有宽度}] {实际抽样 (平顶/自然)}}
                child {node[node2, note={$\delta(t)$序列}] {理想抽样 (冲激)}}
            }
            % 分支 2:按时间间隔
            child {node[node1] {2. 按时间间隔}
                child {node[node2] {非均匀抽样 (随机)}}
                child {node[node2, note={$T_s = \text{const}$}] {均匀抽样 (等间隔)}}
            }
            % 分支 1:按信号频谱
            child {node[node1] {1. 按信号频谱}
                child {node[node2, note={对应带通信号}] {带通抽样}}
                child {node[node2, note={对应基带信号}] {低通抽样}}
            };
    \end{tikzpicture}
    }
    \end{center}

    \begin{itemize}
        \item \textbf{1. 信号类型 (频谱特性)}:
        \begin{itemize}
            \item \textbf{低通抽样}:针对基带信号,满足 $f_s \ge 2f_H$ 。
            \item \textbf{带通抽样}:针对频谱在 $[f_L, f_H]$ 的信号,采样率可低于 $2f_H$。
        \end{itemize}
        
        \item \textbf{2. 抽样间隔 (时间特性)}:
        \begin{itemize}
            \item \textbf{均匀抽样}:抽样时刻 $t_n = nT_s$,间隔恒定。通信原理主要讨论此类。
            \item \textbf{非均匀抽样}:间隔随机或变化。
        \end{itemize}

        \item \textbf{3. 脉冲序列 (物理实现)}:
        \begin{itemize}
            \item \textbf{理想抽样}:使用冲激脉冲序列 $\delta_T(t)$,便于理论分析。
            \item \textbf{实际抽样}:使用具有一定宽度的脉冲(如矩形脉冲),会引入孔径失真(如平顶抽样)。
        \end{itemize}
    \end{itemize}
\end{kbox}