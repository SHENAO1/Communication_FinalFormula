% content/04_SourceCoding/topic01_sampling_derivation.tex

% --- Part 1: 数学基础回顾 ---
\begin{intuitionbox}{数学预备:从傅里叶级数到冲激串变换}
    理解抽样定理频谱搬移的关键,在于掌握\textbf{周期信号的频谱特性}。
    
    \subsubsection*{1. 傅里叶级数 (Fourier Series, FS) 回顾}
    对于周期为 $T_0$ 的周期信号 $f(t)$,角频率 $\omega_0 = \frac{2\pi}{T_0}$ (或基频 $f_0 = \frac{1}{T_0}$)。
    
    \begin{itemize}
        \item \textbf{三角形式}:将信号分解为直流分量与不同频率的正弦/余弦波之和。
        \begin{equation}
            f(t) = a_0 + \sum_{n=1}^{\infty} \left[ a_n \cos(n\omega_0 t) + b_n \sin(n\omega_0 t) \right]
        \end{equation}
        
        % [修改1] 使用 cases 环境大括号包裹系数,并使用 displaystyle 优化积分显示
        其中系数为:
        \begin{equation}
        \begin{cases}
            \displaystyle a_0 = \frac{1}{T_0}\int_{T_0} f(t)dt \\[10pt]
            \displaystyle a_n = \frac{2}{T_0}\int_{T_0} f(t)\cos(n\omega_0 t)dt \\[10pt]
            \displaystyle b_n = \frac{2}{T_0}\int_{T_0} f(t)\sin(n\omega_0 t)dt
        \end{cases}
        \end{equation}

        \item \textbf{指数形式}:利用欧拉公式 $e^{j\theta} = \cos\theta + j\sin\theta$,形式更为简洁,便于推导。
        \begin{equation}
            f(t) = \sum_{n=-\infty}^{\infty} F_n e^{j n \omega_0 t}
        \end{equation}
        \textbf{傅里叶系数} $F_n$ (频谱幅度) 计算公式:
        \begin{equation}
            F_n = \frac{1}{T_0} \int_{-T_0/2}^{T_0/2} f(t) e^{-j n \omega_0 t} dt
        \end{equation}
    \end{itemize}

    \subsubsection*{2. 周期单位冲激串的傅里叶变换 (关键推导)}
    设周期单位冲激串为 $\delta_T(t) = \sum_{n=-\infty}^{\infty} \delta(t - nT_s)$,周期为 $T_s$,抽样角频率 $\omega_s = \frac{2\pi}{T_s}$。
    
    \textbf{第一步:求傅里叶系数 $F_n$}
    在一个周期 $[-T_s/2, T_s/2]$ 内,$\delta_T(t)$ 仅在 $t=0$ 处有一个 $\delta(t)$。
    \begin{equation}
        F_n = \frac{1}{T_s} \int_{-T_s/2}^{T_s/2} \delta(t) e^{-j n \omega_s t} dt \overset{\text{筛分性质}}{=} \frac{1}{T_s} \cdot 1 \cdot e^{0} = \frac{1}{T_s}
    \end{equation}
    可见,时域冲激串的傅里叶系数是常数 $\frac{1}{T_s}$。
    代入指数形式级数:
    \begin{equation}
        \delta_T(t) = \frac{1}{T_s} \sum_{n=-\infty}^{\infty} e^{j n \omega_s t}
    \end{equation}

    \textbf{第二步:求傅里叶变换 (FT)}
    对上式两边取傅里叶变换 $\mathcal{F}\{\cdot\}$。利用频移性质 $\mathcal{F}\{e^{j\omega_0 t}\} = 2\pi \delta(\omega - \omega_0)$。
    
    \begin{itemize}
        \item \textbf{角频率 $\omega$ 域结果}:
        % [修改2] 使用 split 环境进行多行公式对齐,防止超出单栏宽度
        \begin{equation}
        \begin{split}
            \mathcal{F}[\delta_T(t)] &= \frac{1}{T_s} \sum_{n=-\infty}^{\infty} 2\pi \delta(\omega - n\omega_s) \\
                                     &= \frac{2\pi}{T_s} \sum_{n=-\infty}^{\infty} \delta(\omega - n\omega_s)
        \end{split}
        \end{equation}
        
        \item \textbf{频率 $f$ 域结果} (常用,注意系数变化):
        利用 $\delta(ax) = \frac{1}{|a|}\delta(x)$,这里 $\omega = 2\pi f$,则 $\delta(\omega) = \delta(2\pi f) = \frac{1}{2\pi}\delta(f)$。
        \begin{equation}
            \Delta_T(f) = \mathcal{F}[\delta_T(t)] = \frac{1}{T_s} \sum_{n=-\infty}^{\infty} \delta(f - n f_s)
        \end{equation}
        \textcolor{textred}{\textbf{结论}}:时域是周期为 $T_s$ 的冲激串,频域是周期为 $f_s$ 的冲激串,且幅度加权为 $f_s$ (即 $1/T_s$)。
    \end{itemize}
\end{intuitionbox}

% --- Part 2: 理想抽样定理图解与推导 ---
% [修改3] 使用 figure* 环境 (带星号) 实现跨双栏显示
\begin{figure*}[t]
\begin{kbox}{理想抽样过程:波形与频谱对照}
    理想抽样可以看作是模拟信号 $m(t)$ 与单位冲激脉冲序列 $\delta_T(t)$ 在时域的\textbf{乘积}。
    根据卷积定理:\textbf{时域相乘,对应频域卷积}。

    \centering
    % [调整] 将 xscale 调整为 1.1 以利用全宽空间
    \begin{tikzpicture}[>=Latex, xscale=1.1, yscale=1]
        % --- 颜色与样式定义 ---
        \definecolor{myblue}{RGB}{0, 114, 189} 
        \definecolor{mygreen}{RGB}{119, 172, 48} 
        \tikzstyle{signal} = [thick, myblue]
        \tikzstyle{spectrum} = [thick, myblue, fill=myblue!10]
        \tikzstyle{axis} = [->, gray, thin]
        \tikzstyle{labeltext} = [font=\small]
        
        % --- 布局参数 ---
        \def\yRowA{4.2}   % 第一行 Y 坐标
        \def\yRowB{0}     % 第二行 Y 坐标
        \def\yRowC{-5.0}  % 第三行 Y 坐标
        \def\xColL{-5}    % 左列 X 坐标
        \def\xColR{5}     % 右列 X 坐标

        % =================================================
        % 第一行:模拟信号
        % =================================================
        \begin{scope}[shift={(\xColL, \yRowA)}]
            \draw[axis] (-2.5,0) -- (2.5,0) node[right] {$t$};
            \draw[axis] (0,-0.5) -- (0,1.5) node[above] {$m(t)$};
            \draw[signal] plot[domain=-2:2, samples=100] (\x, {0.8*cos(50*\x) + 0.3*cos(120*\x)});
            \node[labeltext] at (0, -0.8) {(a) 模拟信号};
        \end{scope}

        \draw[<->, thick, gray!50] (\xColL+4, \yRowA+0.5) -- (\xColR-4, \yRowA+0.5);

        \begin{scope}[shift={(\xColR, \yRowA)}]
            \draw[axis] (-2.5,0) -- (2.5,0) node[right] {$f$};
            \draw[axis] (0,-0.5) -- (0,1.5) node[above] {$|M(f)|$};
            \draw[spectrum] (-1.2,0) -- (0,1.2) -- (1.2,0) -- cycle;
            % [新增] 频谱幅值 A,位置在顶点右上方
            \node[above right, font=\small] at (0, 1.2) {$A$}; 
            \node[below] at (-1.2,0) {$-f_H$};
            \node[below] at (1.2,0) {$f_H$};
            \node[labeltext] at (0, -0.8) {(b) 信号频谱};
        \end{scope}

        % =================================================
        % 运算符号 (乘法与卷积)
        % =================================================
        \def\opY{{(\yRowA+\yRowB)/2 + 0.5}} 

        % 左侧:乘法
        \node[font=\bfseries\huge, text=mygreen] at (\xColL, \opY) {$\times$};
        \node[align=center, font=\bfseries\small, text=mygreen] at (\xColL+1.0, \opY) {时域\\相乘};
        % 右侧:卷积
        \node[font=\bfseries\huge, text=mygreen] at (\xColR, \opY) {$*$};
        \node[align=center, font=\bfseries\small, text=mygreen] at (\xColR+1.0, \opY) {频域\\卷积};

        % =================================================
        % 第二行:冲激序列
        % =================================================
        \begin{scope}[shift={(\xColL, \yRowB)}]
            \draw[axis] (-2.5,0) -- (2.5,0) node[right] {$t$};
            \draw[axis] (0,-0.5) -- (0,1.5) node[above] {$\delta_T(t)$};
            \foreach \x in {-2,-1.5,...,2} {
                \draw[->, thick, black] (\x,0) -- (\x,1);
            }
            \node[below, text=red] at (0.5,0) {$T_s$};
            \node[below, text=red] at (-0.5,0) {$-T_s$};
            \node[labeltext] at (0, -0.8) {(c) 冲激序列};
        \end{scope}
        
        \draw[<->, thick, gray!50] (\xColL+4, \yRowB+0.5) -- (\xColR-4, \yRowB+0.5);

        \begin{scope}[shift={(\xColR, \yRowB)}]
            \draw[axis] (-2.5,0) -- (2.5,0) node[right] {$f$};
            \draw[axis] (0,-0.5) -- (0,1.5) node[above] {$\delta_T(f)$};
            \foreach \x in {-2,-1,0,1,2} {
                \draw[->, thick, black] (\x,0) -- (\x,1);
            }
            % [新增] 冲激强度 fs,位置在中心箭头顶点右上方
            \node[above right, font=\small] at (0, 1) {$f_s$};
            \draw[<->, red, dashed] (0, 0.6) -- (1, 0.6) node[midway, above, font=\scriptsize] {$f_s$};
            \node[below, text=red] at (1,0) {$f_s$};
            \node[labeltext] at (0, -0.8) {(d) 脉冲频谱};
        \end{scope}

        % =================================================
        % 等号
        % =================================================
        \draw[double, double distance=2pt, gray] (\xColL, {\yRowB - 1.2}) -- (\xColL, {\yRowB - 1.8});
        \draw[double, double distance=2pt, gray] (\xColR, {\yRowB - 1.2}) -- (\xColR, {\yRowB - 1.8});

        % =================================================
        % 第三行:已抽样信号 (修改后:Y轴拉高)
        % =================================================
        \begin{scope}[shift={(\xColL, \yRowC)}]
            \draw[axis] (-2.5,0) -- (2.5,0) node[right] {$t$};
            % [修改] Y轴高度由 1.6 改为 2.0,增加视觉空间
            \draw[axis] (0,-0.5) -- (0,2.0) node[above] {$m_s(t)$};
            
            \draw[dashed, gray!80] plot[domain=-2.2:2.2, samples=100] (\x, {0.8*cos(50*\x) + 0.3*cos(120*\x)});
            \foreach \x in {-2,-1.5,...,2} {
                \draw[-, thick, myblue] (\x,0) -- (\x, {0.8*cos(50*\x) + 0.3*cos(120*\x)});
                \fill[myblue] (\x, {0.8*cos(50*\x) + 0.3*cos(120*\x)}) circle (1.8pt);
            }
            \node[right, font=\footnotesize, fill=white, inner sep=1pt] at (0.2, 1.3) {$m(nT_s)$};
            \node[labeltext] at (0, -0.8) {(e) 抽样信号波形};
        \end{scope}
        
        \draw[<->, thick, gray!50] (\xColL+4, \yRowC+0.5) -- (\xColR-4, \yRowC+0.5);

        % =================================================
        % 第三行右侧:(f) 周期延拓频谱 (修改后:Y轴大幅拉高防遮挡)
        % =================================================
        \begin{scope}[shift={(\xColR, \yRowC)}]
            \draw[axis] (-2.5,0) -- (2.5,0) node[right] {$f$};
            % [修改] Y轴高度由 1.6 改为 2.4,确保轴标签 M_s(f) 高于“低通滤波”文字
            \draw[axis] (0,-0.5) -- (0,2.4) node[above] {$M_s(f)$};
            
            % 绘制频谱三角形
            \foreach \k in {-2,-1,0,1,2} {
                \draw[spectrum] (\k-0.4, 0) -- (\k, 1.2) -- (\k+0.4, 0) -- cycle;
            }
            % 抽样后频谱幅值 Afs
            \node[above right, font=\small] at (0, 1.5) {$Af_s$};
            
            % --- 滤波器标注 ---
            \draw[thick, dashed, red] (-0.55, -0.1) rectangle (0.55, 1.45);
            
            % 低通滤波文字 (y=1.9),现在位于轴标签 (y>2.4) 下方,互不遮挡
            \node[red, font=\scriptsize] (filterLabel) at (-1.3, 1.9) {低通滤波};
            \draw[->, red, dashed, thick] (filterLabel) -- (-0.55, 1.45);

            % 坐标轴刻度与标注
            \node[below] at (0,0) {$0$};
            \node[below] at (1,0) {$f_s$};
            \node[below] at (-1,0) {$-f_s$};
            \node[below, font=\scriptsize] at (0.4,0) {$f_H$};
            
            % 采样定理条件框
            \node[draw, rounded corners, fill=yellow!10, font=\bfseries\small, text=red] at (1.8, 1.5) {$f_s \ge 2f_H$};
            \node[labeltext] at (0, -0.8) {(f) 周期延拓频谱};
        \end{scope}

    \end{tikzpicture}
    
    \tcblower
    
    \textbf{核心数学推导}:
    \begin{itemize}
        \item \textbf{时域表达式}:
        \begin{equation}
            m_s(t) = m(t) \cdot \delta_T(t) = m(t) \sum_{n=-\infty}^{\infty} \delta(t - nT_s)
        \end{equation}
        \item \textbf{频域表达式}(应用卷积定理 $M_s(f) = M(f) * \Delta_T(f)$):
        \begin{equation}
            \begin{aligned}
                M_s(f) &= M(f) * \left[ \frac{1}{T_s} \sum_{n=-\infty}^{\infty} \delta(f - n f_s) \right] \\
                       &= \frac{1}{T_s} \sum_{n=-\infty}^{\infty} M(f) * \delta(f - n f_s) \\
                       &= \frac{1}{T_s} \sum_{n=-\infty}^{\infty} M(f - n f_s)
                       &= \frac{1}{T_s} \sum_{n=-\infty}^{\infty} M(f - n / T_s)
            \end{aligned}
        \end{equation}
    \end{itemize}
    \textbf{物理意义}:抽样后信号的频谱 $M_s(f)$ 是原信号频谱 $M(f)$ 以 $f_s$ 为周期进行\textbf{无限周期延拓},且幅度缩减为原来的 $1/T_s$。
\end{kbox}
\end{figure*}

% --- Part 3: 定理定义提取 ---
\begin{kbox}{低通抽样定理 (Nyquist Sampling Theorem)}
    
    \textbf{定理内容}:
    对于最高频率小于 $f_H$ 的模拟信号 $m(t)$,若要由其等间隔的抽样值唯一确定(无失真恢复)原信号,抽样间隔 $T_s$ 或抽样速率 $f_s$ 应满足以下条件:

    \begin{itemize}
        \item \textbf{抽样速率条件}:
        \begin{equation}
            \boxed{f_s \ge 2f_H}
        \end{equation}
        即抽样频率必须大于等于信号最高频率的2倍。
        
        \item \textbf{抽样间隔条件}:
        \begin{equation}
            \boxed{T_s \le \frac{1}{2f_H}}
        \end{equation}
    \end{itemize}

    \textbf{关键术语定义}:
    \begin{enumerate}
        \item \textbf{奈奎斯特速率 (Nyquist Rate)}:允许的\textbf{最低}抽样速率。
        \begin{equation}
            f_s = 2f_H
        \end{equation}
        \item \textbf{奈奎斯特间隔 (Nyquist Interval)}:允许的\textbf{最大}抽样间隔。
        \begin{equation}
            T_s = \frac{1}{2f_H}
        \end{equation}
    \end{enumerate}

    \tcblower % 分割线,下方放例子
    
    \textbf{典型应用实例}:
    对于典型电话信号,其最高频率通常限制为 $f_H = 3400\text{Hz}$。
    根据定理,无失真恢复所需的:
    \begin{itemize}
        \item 理论最低抽样速率 (奈奎斯特速率):$f_s = 2 \times 3400 = 6800\text{Hz}$。
        \item \textbf{实际工程标准}:为了留有保护频带 (Guard Band) 便于滤波,通常取 $f_s = \mathbf{8000\text{Hz}}$。
    \end{itemize}
\end{kbox}
% =========================================================
% [新增] Part 4: 实际抽样——平顶抽样与孔径效应
% =========================================================

\begin{intuitionbox}{核心辨析:脉冲保持是“单脉冲”还是“周期序列”?}
    在推导平顶抽样公式时,初学者常混淆 $h(t)$ 的定义。我们需要从线性系统(卷积)的角度来理解。
    
    \begin{itemize}
        \item \textbf{输入信号 $m_s(t)$}:是\textbf{周期性}的冲激序列(理想抽样结果)。
        \[ m_s(t) = \sum m(nT_s) \delta(t - nT_s) \]
        \item \textbf{系统响应 $h(t)$}:是脉冲保持电路的\textbf{单位冲激响应}。它是一个\textbf{孤立的、单个}矩形门信号(Single Gate)。
        \[
        h(t) = \begin{cases} A, & 0 \le t \le \tau \\ 0, & \text{其他} \end{cases}
        \]
    \end{itemize}
    
    \textbf{卷积的物理图像}:
    卷积 $m_H(t) = m_s(t) * h(t)$ 相当于把“单个印章” $h(t)$,按照 $m_s(t)$ 指定的位置(每隔 $T_s$)和力度(幅度)盖在纸上,从而形成了一串平顶脉冲。
\end{intuitionbox}

\begin{kbox}{平顶抽样 (Flat-top Sampling) 的数学推导}
    平顶抽样产生模型为:理想抽样信号 $m_s(t)$ 通过一个冲激响应为 $h(t)$ 的线性系统(零阶保持器)。
    
    \begin{equation}
        m_H(t) = m_s(t) * h(t) \quad \overset{\mathcal{F}}{\longleftrightarrow} \quad M_H(f) = M_s(f) \cdot H(f)
    \end{equation}

    \subsubsection*{1. 保持电路频率响应 $H(f)$ 的推导}
    我们需要求单个矩形脉冲 $h(t)$ 的傅里叶变换:
    \begin{equation}
        H(f) = \int_{0}^{\tau} A \cdot e^{-j2\pi ft} dt = \frac{A}{-j2\pi f} (e^{-j2\pi f\tau} - 1)
    \end{equation}
    利用欧拉公式技巧,提取公因子 $e^{-j\pi f\tau}$:
    \begin{equation}
        \begin{split}
            H(f) &= \frac{A}{j2\pi f} e^{-j\pi f\tau} (e^{j\pi f\tau} - e^{-j\pi f\tau}) \\
                 &= A\tau \frac{\sin(\pi f\tau)}{\pi f\tau} e^{-j\pi f\tau}
        \end{split}
    \end{equation}
    最终得到:
    \begin{equation}
        \boxed{ H(f) = A\tau \cdot \text{Sa}(\pi f\tau) \cdot e^{-j\pi f\tau} }
    \end{equation}
    
    \subsubsection*{2. 频域影响:孔径效应 (Aperture Effect)}
    将 $H(f)$ 代入 $M_H(f)$,幅度谱为:
    \begin{equation}
        |M_H(f)| = \underbrace{|M_s(f)|}_{\text{周期延拓}} \cdot \underbrace{A\tau |\text{Sa}(\pi f\tau)|}_{\text{孔径失真加权}}
    \end{equation}
    
    \begin{center}
    \resizebox{0.95\linewidth}{!}{
    \begin{tikzpicture}
        \begin{axis}[
            width=10cm, height=5cm,
            xlabel={频率 $f$},
            ylabel={幅度增益 $|H(f)|$},
            axis lines=middle,
            xmin=-0.5, xmax=3.5,
            ymin=0, ymax=1.2,
            xtick={1, 2, 3},
            xticklabels={$\frac{1}{\tau}$, $\frac{2}{\tau}$, $\frac{3}{\tau}$},
            ytick={1},
            yticklabels={$A\tau$},
            every axis x label/.style={at={(current axis.right of origin)},anchor=west},
            every axis y label/.style={at={(current axis.above origin)},anchor=south},
            samples=200, domain=0.01:3.2
        ]
            % [修改点] 这里原来是 myblue,改为 mainblue (模板自带颜色)
            \addplot [thick, mainblue] {abs(sin(deg(pi*x))/(pi*x))};
            
            % 理想抽样对比
            \draw[dashed, gray] (axis cs:0, 1) -- (axis cs:3.2, 1);
            \node[gray, font=\small] at (axis cs: 2.5, 1.1) {理想抽样 (平坦)};
            
            % 失真指示
            \draw[<->, red, thick] (axis cs:0.5, 0.95) -- (axis cs:0.5, 0.63);
            \node[red, align=left, font=\small] at (axis cs: 1.2, 0.7) {高频分量衰减\\(孔径失真)};
        \end{axis}
    \end{tikzpicture}
    }
    \end{center}
    
    \textbf{结论}:由于 $Sa$ 函数随频率增加而下降,信号的高频分量被衰减。抽样脉冲宽度 $\tau$ 越大,频谱零点 $1/\tau$ 越靠近零频,失真越严重。
    \\
    \textbf{解决方法}:在接收端使用均衡器(提升高频)进行补偿。
\end{kbox}