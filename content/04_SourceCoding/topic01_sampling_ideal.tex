% --- Part 1: 数学基础回顾 ---
\begin{intuitionbox}{数学预备:从傅里叶级数到冲激串变换}
    理解抽样定理频谱搬移的关键,在于掌握\textbf{周期信号的频谱特性}。
    
    \subsubsection*{1. 傅里叶级数 (Fourier Series, FS) 回顾}
    对于周期为 $T_0$ 的周期信号 $f(t)$,角频率 $\omega_0 = \frac{2\pi}{T_0}$ (或基频 $f_0 = \frac{1}{T_0}$)。
    
    \begin{itemize}
        \item \textbf{三角形式}:
        \begin{equation}
            f(t) = a_0 + \sum_{n=1}^{\infty} \left[ a_n \cos(n\omega_0 t) + b_n \sin(n\omega_0 t) \right]
        \end{equation}
        其中系数为:
        \begin{equation}
        \begin{cases}
            \displaystyle a_0 = \frac{1}{T_0}\int_{T_0} f(t)dt \\[10pt]
            \displaystyle a_n = \frac{2}{T_0}\int_{T_0} f(t)\cos(n\omega_0 t)dt \\[10pt]
            \displaystyle b_n = \frac{2}{T_0}\int_{T_0} f(t)\sin(n\omega_0 t)dt
        \end{cases}
        \end{equation}

        \item \textbf{指数形式}:
        \begin{equation}
            f(t) = \sum_{n=-\infty}^{\infty} F_n e^{j n \omega_0 t}, \quad F_n = \frac{1}{T_0} \int_{-T_0/2}^{T_0/2} f(t) e^{-j n \omega_0 t} dt
        \end{equation}
    \end{itemize}

    \subsubsection*{2. 周期单位冲激串的傅里叶变换}
    设周期单位冲激串为 $\delta_T(t) = \sum_{n=-\infty}^{\infty} \delta(t - nT_s)$。
    
    \begin{itemize}
        \item \textbf{时域}:周期为 $T_s$ 的冲激串。
        \item \textbf{频域}:周期为 $f_s$ 的冲激串,且幅度加权为 $f_s$。
        \begin{equation}
            \Delta_T(f) = \mathcal{F}[\delta_T(t)] = \frac{1}{T_s} \sum_{n=-\infty}^{\infty} \delta(f - n f_s)
        \end{equation}
    \end{itemize}
\end{intuitionbox}

% --- Part 2: 理想抽样定理图解与推导 ---
\begin{figure*}[t]
% [修改说明] 这里添加了 [breakable=false],强制禁止跨页,修复遮挡问题
\begin{kbox}[breakable=false]{理想抽样过程:波形与频谱对照}
    理想抽样可以看作是模拟信号 $m(t)$ 与单位冲激脉冲序列 $\delta_T(t)$ 在时域的\textbf{乘积}。
    根据卷积定理:\textbf{时域相乘,对应频域卷积}。

    \centering
    \begin{tikzpicture}[>=Latex, xscale=1.1, yscale=1]
        % --- 定义颜色 ---
        \definecolor{myblue}{RGB}{0, 114, 189} 
        \definecolor{mygreen}{RGB}{119, 172, 48} 
        \tikzstyle{signal} = [thick, myblue]
        \tikzstyle{spectrum} = [thick, myblue, fill=myblue!10]
        \tikzstyle{axis} = [->, gray, thin]
        \tikzstyle{labeltext} = [font=\small]
        
        % --- 布局参数 ---
        \def\yRowA{4.2}   \def\yRowB{0}     \def\yRowC{-5.0}
        \def\xColL{-5}    \def\xColR{5}

        % =================================================
        % 第一行:模拟信号
        % =================================================
        \begin{scope}[shift={(\xColL, \yRowA)}]
            \draw[axis] (-2.5,0) -- (2.5,0) node[right] {$t$};
            \draw[axis] (0,-0.5) -- (0,1.5) node[above] {$m(t)$};
            \draw[signal] plot[domain=-2:2, samples=100] (\x, {0.8*cos(50*\x) + 0.3*cos(120*\x)});
            \node[labeltext] at (0, -0.8) {(a) 模拟信号};
        \end{scope}

        \draw[<->, thick, gray!50] (\xColL+4, \yRowA+0.5) -- (\xColR-4, \yRowA+0.5);
        \begin{scope}[shift={(\xColR, \yRowA)}]
            \draw[axis] (-2.5,0) -- (2.5,0) node[right] {$f$};
            \draw[axis] (0,-0.5) -- (0,1.5) node[above] {$|M(f)|$};
            \draw[spectrum] (-1.2,0) -- (0,1.2) -- (1.2,0) -- cycle;
            \node[above right, font=\small] at (0, 1.2) {$A$};
            \node[below] at (-1.2,0) {$-f_H$};
            \node[below] at (1.2,0) {$f_H$};
            \node[labeltext] at (0, -0.8) {(b) 信号频谱};
        \end{scope}

        % =================================================
        % 运算符号
        % =================================================
        \def\opY{{(\yRowA+\yRowB)/2 + 0.5}} 
        \node[font=\bfseries\huge, text=mygreen] at (\xColL, \opY) {$\times$};
        \node[align=center, font=\bfseries\small, text=mygreen] at (\xColL+1.0, \opY) {时域\\相乘};
        \node[font=\bfseries\huge, text=mygreen] at (\xColR, \opY) {$*$};
        \node[align=center, font=\bfseries\small, text=mygreen] at (\xColR+1.0, \opY) {频域\\卷积};

        % =================================================
        % 第二行:冲激序列
        % =================================================
        \begin{scope}[shift={(\xColL, \yRowB)}]
            \draw[axis] (-2.5,0) -- (2.5,0) node[right] {$t$};
            \draw[axis] (0,-0.5) -- (0,1.5) node[above] {$\delta_T(t)$};
            \foreach \x in {-2,-1.5,...,2} { \draw[->, thick, black] (\x,0) -- (\x,1); }
            \node[below, text=red] at (0.5,0) {$T_s$};
            \node[labeltext] at (0, -0.8) {(c) 冲激序列};
        \end{scope}
        
        \draw[<->, thick, gray!50] (\xColL+4, \yRowB+0.5) -- (\xColR-4, \yRowB+0.5);
        \begin{scope}[shift={(\xColR, \yRowB)}]
            \draw[axis] (-2.5,0) -- (2.5,0) node[right] {$f$};
            \draw[axis] (0,-0.5) -- (0,1.5) node[above] {$\delta_T(f)$};
            \foreach \x in {-2,-1,0,1,2} { \draw[->, thick, black] (\x,0) -- (\x,1); }
            \node[above right, font=\small] at (0, 1) {$f_s$};
            \node[labeltext] at (0, -0.8) {(d) 脉冲频谱};
        \end{scope}

        % =================================================
        % 等号
        % =================================================
        \draw[double, double distance=2pt, gray] (\xColL, {\yRowB - 1.2}) -- (\xColL, {\yRowB - 1.8});
        \draw[double, double distance=2pt, gray] (\xColR, {\yRowB - 1.2}) -- (\xColR, {\yRowB - 1.8});

        % =================================================
        % 第三行:已抽样信号
        % =================================================
        \begin{scope}[shift={(\xColL, \yRowC)}]
            \draw[axis] (-2.5,0) -- (2.5,0) node[right] {$t$};
            \draw[axis] (0,-0.5) -- (0,2.0) node[above] {$m_s(t)$};
            \draw[dashed, gray!80] plot[domain=-2.2:2.2, samples=100] (\x, {0.8*cos(50*\x) + 0.3*cos(120*\x)});
            \foreach \x in {-2,-1.5,...,2} {
                \draw[-, thick, myblue] (\x,0) -- (\x, {0.8*cos(50*\x) + 0.3*cos(120*\x)});
                \fill[myblue] (\x, {0.8*cos(50*\x) + 0.3*cos(120*\x)}) circle (1.8pt);
            }
            \node[labeltext] at (0, -0.8) {(e) 抽样信号波形};
        \end{scope}
        
        \draw[<->, thick, gray!50] (\xColL+4, \yRowC+0.5) -- (\xColR-4, \yRowC+0.5);
        \begin{scope}[shift={(\xColR, \yRowC)}]
            \draw[axis] (-2.5,0) -- (2.5,0) node[right] {$f$};
            \draw[axis] (0,-0.5) -- (0,2.4) node[above] {$M_s(f)$};
            \foreach \k in {-2,-1,0,1,2} {
                \draw[spectrum] (\k-0.4, 0) -- (\k, 1.2) -- (\k+0.4, 0) -- cycle;
            }
            \node[above right, font=\small] at (0, 1.5) {$Af_s$};
            % 滤波器标注
            \draw[thick, dashed, red] (-0.55, -0.1) rectangle (0.55, 1.45);
            \node[red, font=\scriptsize] (filterLabel) at (-1.3, 1.9) {低通滤波};
            \draw[->, red, dashed, thick] (filterLabel) -- (-0.55, 1.45);
            % 坐标轴
            \node[below] at (0,0) {$0$};
            \node[below] at (1,0) {$f_s$}; \node[below] at (-1,0) {$-f_s$};
            % 条件框
            \node[draw, rounded corners, fill=yellow!10, font=\bfseries\small, text=red] at (1.8, 1.5) {$f_s \ge 2f_H$};
            \node[labeltext] at (0, -0.8) {(f) 周期延拓频谱};
        \end{scope}
    \end{tikzpicture}
    
    \tcblower
    
    \textbf{核心数学推导}:
    \begin{itemize}
        \item \textbf{时域表达式}:
        \begin{equation}
            m_s(t) = m(t) \cdot \delta_T(t) = m(t) \sum_{n=-\infty}^{\infty} \delta(t - nT_s)
        \end{equation}
        \item \textbf{频域表达式}(应用卷积定理):
        \begin{equation}
            \begin{aligned}
                M_s(f) &= M(f) * \left[ \frac{1}{T_s} \sum_{n=-\infty}^{\infty} \delta(f - n f_s) \right] \\
                       &= \frac{1}{T_s} \sum_{n=-\infty}^{\infty} M(f - n f_s)
            \end{aligned}
        \end{equation}
    \end{itemize}
\end{kbox}
\end{figure*}

% --- Part 3: 定理定义提取 ---
% [注] 这里的用法仍然兼容,因为 [2][] 允许省略可选参数
\begin{kbox}{低通抽样定理 (Nyquist Sampling Theorem)}
    \textbf{定理内容}:对于最高频率为 $f_H$ 的模拟信号,无失真恢复条件为:
    \begin{itemize}
        \item \textbf{抽样速率条件}:$\boxed{f_s \ge 2f_H}$
        \item \textbf{抽样间隔条件}:$\boxed{T_s \le \frac{1}{2f_H}}$
    \end{itemize}

    \textbf{应用实例}:电话信号 $f_H \approx 3.4\text{kHz}$,理论 $f_s \ge 6.8\text{kHz}$,工程标准取 $f_s = 8\text{kHz}$。
\end{kbox}