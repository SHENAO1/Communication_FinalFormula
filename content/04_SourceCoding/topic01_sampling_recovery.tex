% content/04_SourceCoding/topic01_sampling_recovery.tex

\begin{kbox}{信号的无失真恢复:从框图到公式}
    抽样定理不仅规定了“怎么抽”($f_s \ge 2f_H$),更通过重建原理告诉我们“怎么复原”。
    
    \subsubsection*{1. 抽样与恢复原理框图}
    信号的恢复过程是将抽样信号 $m_s(t)$ 通过一个理想低通滤波器 $H_L(f)$。
    
    \begin{center}
    \resizebox{\linewidth}{!}{
    \begin{tikzpicture}[node distance=2.0cm, auto, >=Latex]
        % 样式定义
        \tikzstyle{block} = [draw, rectangle, minimum height=1.2cm, minimum width=2.0cm, align=center, thick, draw=mainblue, fill=subbg]
        \tikzstyle{input} = [coordinate]
        \tikzstyle{output} = [coordinate]

        % 节点
        \node [input] (input) {};
        \node [block, right=1.2cm of input] (multiplier) {$\times$\\ \footnotesize 乘法器};
        \node [input, below=0.8cm of multiplier] (pulse) {};
        \node [block, right=1.5cm of multiplier] (lpf) {理想低通\\ $H_L(f)$};
        \node [output, right=1.2cm of lpf] (output) {};

        % 连线
        \draw [->, thick] (input) -- node [above] {$m(t)$} (multiplier);
        \draw [->, thick] (pulse) -- node [right] {$\delta_T(t)$} (multiplier);
        \draw [->, thick] (multiplier) -- node [above] {$m_s(t)$} (lpf);
        \draw [->, thick] (lpf) -- node [above] {$m(t)$} (output);
    \end{tikzpicture}
    }
    \end{center}

    \begin{itemize}
        \item \textbf{频域关系}:抽样信号频谱 $M_s(f)$ 是 $M(f)$ 的周期延拓。
        \begin{equation}
            M_s(f) = \frac{1}{T_s} \sum_{n=-\infty}^{\infty} M(f - n f_s)
        \end{equation}
        \item \textbf{恢复条件}:利用理想低通滤波器 $H_L(f)$ 滤出基带分量。
        \begin{equation}
            H_L(f) = \begin{cases} 
                T_s, & |f| \le f_H \\
                0, & |f| > f_H 
            \end{cases}
            \quad (\text{理想低通})
        \end{equation}
    \end{itemize}

        % --- 2. 滤波器增益的直观理解 ---
        \begin{intuitionbox}{思考:为什么滤波器增益是 $T_s$ (即 $1/f_s$)?}
            \textbf{问题}:一般的理想低通滤波器增益通常为1,但为何重建时 $H_L(f)$ 的通带增益必须是 $T_s$?
            
            \textbf{笔记解析}:
            \begin{itemize}
                \item 在理想抽样后的频谱 $M_s(f)$ 中,包含了幅度因子 $f_s$(即 $1/T_s$)。
                \item 为了\textcolor{alertred}{恢复原始信号的真实幅值},滤波器的增益必须抵消这个因子。
            \end{itemize}
            
            \textbf{结论}:
            \begin{equation}
                H_L(f) = \begin{cases} 
                    T_s, & |f| \le f_H \\
                    0, & |f| > f_H 
                \end{cases}
                \quad \text{其中 } T_s = \frac{1}{f_s}
            \end{equation}
        \end{intuitionbox}


    \subsubsection*{2. 重建滤波器的时域推导}
    \textbf{目标}:求解理想低通滤波器 $H_L(f)$ 的时域冲击响应 $h_L(t)$。
    
    重建在频域表示为 $M_s(f) \cdot H_L(f)$,对应时域卷积。我们需要找到频域门函数对应的时域信号。推导过程严格如下:

    \textbf{Step 1: 引入基础变换对}
    设 $g_{2\tau}(t)$ 为时域宽度为 $2\tau$ 的门函数。已知其傅里叶变换为:
    \begin{equation}
        g_{2\tau}(t) \longleftrightarrow 2\tau \Sa(\omega \tau)
    \end{equation}
    其中 $\omega = 2\pi f$。

    \textbf{Step 2: 利用对偶性 (Duality)}
    根据傅里叶变换的对偶性,并整理系数,可以得到 $Sa$ 函数的变换对:
    \begin{equation}
        \Sa(\tau t) \longleftrightarrow \frac{\pi}{\tau} g_{2\tau}(\omega)
    \end{equation}
    移项整理,将幅度系数移至左侧:
    \begin{equation}
        \frac{\tau}{\pi} \Sa(\tau t) \longleftrightarrow g_{2\tau}(\omega)
    \end{equation}

    \textbf{Step 3: 频域变量代换与参数匹配}
    将频域变量由 $\omega$ 转换为 $f$(注意 $\omega$ 域的宽度 $2\tau$ 对应 $f$ 域的宽度为 $\frac{2\tau}{2\pi} = \frac{\tau}{\pi}$):
    \begin{equation}
        \frac{\tau}{\pi} \Sa(\tau t) \longleftrightarrow g_{\frac{\tau}{\pi}}(f)
    \end{equation}
    
    此时,我们需要对应到理想低通滤波器 $H_L(f)$。$H_L(f)$ 是一个截止频率为 $f_H$ 的门函数,其总带宽宽度为 $2f_H$。
    
    \begin{equation}
        \boxed{ \frac{\tau}{\pi} = 2f_H } \quad \Longrightarrow \quad \tau = 2\pi f_H
    \end{equation}

    \textbf{Step 4: 代入得出结论}
    将 $\tau = 2\pi f_H$ 代入 Step 3 的左侧公式:
    \begin{equation}
        2f_H \Sa(2\pi f_H t) \longleftrightarrow g_{2f_H}(f)
    \end{equation}
    这正是截止频率为 $f_H$ 的理想低通滤波器 $H_L(f)$。
    
    因此,时域冲击响应为:
    \begin{equation}
        h_L(t) = 2f_H \Sa(2\pi f_H t)
    \end{equation}
    或者写作 Sinc 形式($\Sa(x) = \text{sinc}(x/\pi)$):
    \begin{equation}
        h_L(t) = 2f_H \text{sinc}(2 f_H t)
    \end{equation}

    \textbf{问题}:为什么频域的理想矩形滤波器,时域对应 Sinc 插值?
    
    根据手写笔记,我们利用傅里叶变换的\textbf{对偶性}和\textbf{尺度变换}性质进行详细推导。

    \textbf{Step 1: 基础变换对 (门函数 $\leftrightarrow$ Sa函数)}
    设 $g_{2\tau}(t)$ 为宽度为 $2\tau$ 的门函数(矩形脉冲)。
    \begin{equation}
        g_{2\tau}(t) \longleftrightarrow 2\tau \Sa(\omega \tau), \quad \text{其中 } \Sa(x) = \frac{\sin x}{x}
    \end{equation}
    
    \textbf{Step 2: 利用对偶性}
    利用对偶性 $X(t) \leftrightarrow 2\pi x(-\omega)$,并考虑到门函数是偶函数:
    \begin{equation}
        2\tau \Sa(t \tau) \longleftrightarrow 2\pi g_{2\tau}(\omega)
    \end{equation}
    
    \textbf{Step 3: 尺度变换与代入}
    我们的目标是求 $H_L(f)$ 的时域形式。$H_L(f)$ 是频域上宽度为 $2f_H$ 的门函数。
    \begin{itemize}
        \item 令频域门宽参数对应截止频率:$\tau \rightarrow \pi f_H$ (因 $\omega=2\pi f$)。
        \item 或者更直接地,对 $H_L(f) = \text{rect}(f/2f_H)$ 进行逆变换:
    \end{itemize}
    \begin{equation}
        h_L(t) = \int_{-f_H}^{f_H} 1 \cdot e^{j2\pi f t} df = \frac{e^{j2\pi f_H t} - e^{-j2\pi f_H t}}{j2\pi t}
    \end{equation}
    化简得到结论:
    \begin{equation}
        h_L(t) = \frac{\sin(2\pi f_H t)}{\pi t} = 2f_H \Sa(2\pi f_H t)
    \end{equation}
    
    当 $f_s = 2f_H$ 且取增益 $T_s = 1/f_s$ 时:
    \begin{equation}
        h(t) = T_s \cdot 2f_H \Sa(2\pi f_H t) = \Sa\left(\frac{\pi t}{T_s}\right) = \text{sinc}\left(\frac{t}{T_s}\right)
    \end{equation}

    \subsubsection*{3. 时域内插公式与几何解释}
    根据卷积定理,$m(t) = m_s(t) * h(t)$。
    将 $m_s(t) = \sum m(nT_s)\delta(t-nT_s)$ 代入,得到 \textbf{Whittaker–Shannon 插值公式}:
    
    \begin{equation}
        \boxed{ m(t) = \sum_{n=-\infty}^{\infty} m(nT_s) \cdot \text{sinc}\left( \frac{t - nT_s}{T_s} \right) }
    \end{equation}

    \textbf{图解}:原信号是无穷多个\textbf{移位 Sinc 函数}的线性叠加。

    \vspace{0.5em}
    \centering
    % === Sinc 插值叠加图 (复现原有代码) ===
    \resizebox{\linewidth}{!}{
    \begin{tikzpicture}[>=Latex]
        % --- 样式定义 ---
        \definecolor{myblue}{RGB}{0, 114, 189}
        \definecolor{myred}{RGB}{217, 83, 25}
        \tikzstyle{stem} = [thick, myblue, ->]
        \tikzstyle{sinc} = [thin, gray!70, samples=100, smooth]
        \tikzstyle{sum} = [line width=1.5pt, myred, samples=100, smooth] 
        
        % --- 坐标轴 ---
        \draw[->, thick] (-3.8, 0) -- (3.8, 0) node[right] {$t$};
        \draw[->, thick] (0, -0.5) -- (0, 1.5) node[above] {$m(t)$};

        % --- 定义参数 ---
        \def\valA{0.2}  % n = -2
        \def\valB{0.8}  % n = -1
        \def\valC{1.0}  % n = 0
        \def\valD{0.8}  % n = 1
        \def\valE{0.2}  % n = 2
        
        % --- 辅助函数 ---
        \tikzset{
            declare function={
                sinc(\t) = (abs(\t) < 0.001) ? 1 : sin(180*\t)/(3.14159*\t);
            }
        }

        % --- 1. 绘制 Sinc 分量 ---
        \foreach \n/\val in {-2/\valA, -1/\valB, 0/\valC, 1/\valD, 2/\valE} {
            \draw[sinc] plot[domain=-3.5:3.5] (\x, {\val * sinc(\x - \n)});
        }

        % --- 2. 绘制合成曲线 ---
        \draw[sum] plot[domain=-3.5:3.5] (\x, {
            \valA * sinc(\x + 2) + \valB * sinc(\x + 1) +
            \valC * sinc(\x) + \valD * sinc(\x - 1) + \valE * sinc(\x - 2)
        });

        % --- 3. 绘制抽样点 ---
        \foreach \n/\val in {-2/\valA, -1/\valB, 0/\valC, 1/\valD, 2/\valE} {
            \draw[stem] (\n, 0) -- (\n, \val);
            \fill[myblue] (\n, \val) circle (2pt);
            
            \ifnum\n=0 \node[below, font=\small] at (\n, 0) {$0$};
            \else \ifnum\n=1 \node[below, font=\small] at (\n, 0) {$T_s$};
            \else \ifnum\n=-1 \node[below, font=\small] at (\n, 0) {$-T_s$};
            \else \node[below, font=\small] at (\n, 0) {$\n T_s$};
            \fi \fi \fi
        }

        % --- 标注 ---
        \node[myred, right, font=\bfseries] at (1.5, 1.2) {重构信号};
        \node[gray, right, font=\footnotesize] at (2.4, 0.3) {Sinc分量};
        \draw[->, thin, gray] (2.4, 0.3) -- (1.8, 0.2);
    \end{tikzpicture}
    }


    
\end{kbox}