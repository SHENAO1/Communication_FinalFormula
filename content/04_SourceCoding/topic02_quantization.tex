% content/04_SourceCoding/topic02_quantization.tex

\begin{kbox}{量化 (Quantization)}
    \begin{itemize}
        \item \textbf{均匀量化 (Uniform Quantization)} \pptpage{32}
        \begin{itemize}
            \item \textbf{量化间隔}:$\Delta v = \frac{b-a}{M}$
            \item \textbf{量化噪声功率} (重要结论):
            \begin{equation}
                N_q = \frac{(\Delta v)^2}{12}
            \end{equation}
            \item \textbf{信号量噪比 ($S/N_q$)} \pptpage{34}:
            % 使用 multline 确保长公式能够适应窄栏
            \begin{multline}
                (S/N_q)_{\text{dB}} \approx  4.8 + 6n + 10\lg(S/x_{max}^2)
            \end{multline}
            \textbf{结论}:编码位数 $n$ 每增加 1 bit,信噪比提高约 \textbf{6dB}。
        \end{itemize}

        \item \textbf{非均匀量化 (Non-uniform)} \pptpage{38}
        \begin{itemize}
            \item \textbf{原理}:先压缩 (Compress),再均匀量化,最后扩张 (Expand)。目的:提高\textbf{小信号}的量噪比。
            \item \textbf{A律压缩特性} (A-Law, 欧/中标准, $A=87.6$) \pptpage{44}:
            % 微调格式,确保分数在窄栏中不溢出
            \begin{equation}
                y = \begin{cases} 
                \frac{Ax}{1+\ln A}, & 0 < x \le \frac{1}{A} \\[6pt] 
                \frac{1+\ln(Ax)}{1+\ln A}, & \frac{1}{A} \le x \le 1 
                \end{cases}
            \end{equation}
            \item \textbf{$\mu$律压缩特性} (美/日标准, $\mu=255$):
            \begin{equation}
                y = \frac{\ln(1+\mu x)}{\ln(1+\mu)}, \quad 0 \le x \le 1
            \end{equation}
            \item \textbf{改善度}:$[Q]_{\text{dB}} = 20\lg y'$,即取决于压缩曲线斜率。
        \end{itemize}
    \end{itemize}
\end{kbox} 