% content/05_Baseband/topic01_codes.tex

\subsection{数字基带信号的表示与编码}

\begin{kbox}{基带信号一般表达式 (随机脉冲序列) \pptpage{15}}
    \begin{equation*}
        s(t) = \sum_{n=-\infty}^{\infty} a_n g(t - nT_B)
    \end{equation*}
    \begin{itemize}
        \item $a_n$:第 $n$ 个码元的电平取值(统计独立的随机量)
        \item $g(t)$:单个脉冲波形;$T_B$:码元持续时间
    \end{itemize}
\end{kbox}

% 【修改点1】添加垂直间距,解决挨得太近的问题
\vspace{1em} 

\begin{enumerate}
    \item \textbf{差分编码与译码(相对码)} \pptpage{13}
    \begin{itemize}
        \item \textbf{编码}:$b_n = a_n \oplus b_{n-1}$ (克服相位模糊)
        \item \textbf{译码}:$a_n = b_n \oplus b_{n-1}$
    \end{itemize}
    
    \item \textbf{多电平(四电平)定义} \pptpage{14}
    \begin{itemize}
        \item 映射关系:$00 \to +3E,\; 01 \to +E,\; 10 \to -E,\; 11 \to -3E$
    \end{itemize}
\end{enumerate}