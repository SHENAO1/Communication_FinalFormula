% content/05_Baseband/topic03_isi.tex

\subsection{码间干扰 (ISI) 与奈奎斯特准则}

\begin{kbox}{接收抽样点信号分解 \pptpage{41-43}}
    % 【修改点3】分三行对齐,解决长公式和下括号导致的溢出
    \[
    \begin{aligned}
        r(kT_s + t_0) = & \underbrace{a_k h(t_0)}_{\text{有用信号}} \\
        & + \underbrace{\sum_{n \neq k} a_n h(kT_s + t_0 - nT_s)}_{\text{码间干扰 ISI}} \\
        & + \underbrace{n_R(kT_s + t_0)}_{\text{噪声}}
    \end{aligned}
    \]
    其中 $h(t) = \mathcal{F}^{-1}\{G_T(\omega)C(\omega)G_R(\omega)\}$ 为系统总响应。
\end{kbox}

\subsubsection*{1. 奈奎斯特第一准则 (无ISI准则)}
\begin{itemize}
    \item \textbf{时域条件} \pptpage{50}:
    \[ h(mT_s) = \begin{cases} 1, & m=0 \\ 0, & m \neq 0 \end{cases} \]
    \item \textbf{频域条件} \pptpage{52, 55}:
    \[ \sum_{i=-\infty}^{\infty} H\left(\omega + \frac{2\pi i}{T_s}\right) = T_s, \quad |\omega| \leq \frac{\pi}{T_s} \]
\end{itemize}

\subsubsection*{2. 系统带宽指标}
\begin{enumerate}
    \item \textbf{理想低通系统} \pptpage{58-59}:
    \begin{itemize}
        \item 奈奎斯特带宽:$B = \frac{1}{2T_B} = f_N$
        \item 最高频带利用率:$\eta = R_B/B = 2$ (Baud/Hz)
    \end{itemize}
    \item \textbf{余弦滚降系统} \pptpage{61-62}:
    \begin{itemize}
        \item 带宽:$B = (1+\alpha)f_N$ ($\alpha$ 为滚降系数)
        \item 利用率:$\eta = \frac{2}{1+\alpha}$ (Baud/Hz)
    \end{itemize}
\end{enumerate}